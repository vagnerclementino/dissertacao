\documentclass[msc,proposal]{ppgccufmg} % ou [msc] para dissertações
										% de mestrado. Para propostas ou
										% projetos, usar [phd,project],
										% [msc,proposal], etc.
\usepackage[brazil]{babel} % ajusta vários detalhes para
						   % documentos escritos em português,
						   % principalmente hifenização
\usepackage[T1]{fontenc}   % permite a hifenização de
						   % palavras acentuadas
\usepackage[utf8x]{inputenc} % ou [utf8x] para quem prefere
							 % usar a codificação UTF-8
\usepackage{graphicx} % define o comando \includegraphics
					  % para a inclusão de figuras
\usepackage[square]{natbib} % permite citações naturalmente
							% integradas ao texto
\begin{document}
\ppgccufmg{
title={Uma linguagem para modelagem conceitual em XBRL},
author={Vagner Clementino},
university={Universidade Federal de Minas Gerais},
course={Ciência da Computação},
address={Belo Horizonte},
date={2015-05},
advisor={Rodolfo Rezende},
abstract={Resumo}{resumo},
}
\chapter{Introdução}
\label{ch:intro}
No cenário atual, onde os sistemas de informação tem se tornado grandes e complexos, vemos a existência de um ecossistema de sistemas, também conhecidos como \textit{System of System} (SoS) \cite{Nakagawa:2013:SAF:2489850.2489853}. Organizações de grande porte, como os governos nacionais, precisam projetar sistemas de sistema ao invés vez de sistemas isolados a fim de enfrentar desafios tais como: $(i)$ colaboração entre organizações financiadas e geridas de forma independente; $(ii)$ migração para um ambiente orientado a serviços (SOA\footnote{Service Oriented Architecture }); $(iii)$ processos de testes e verificação da conformidade para sistemas de sistemas. Neste contexto, surge a necessidade do desenvolvimento de \textit{abordagens, técnicas e tecnologias} para a interação e evolução dos SoS.
     
     

\chapter{Revisão da Literatura}
\label{ch:revisao}


\chapter{Objetivos}
\label{ch:objetivos}


\chapter{Metodologia}
\label{ch:metodologia}

% Incluindo bibliografia:
\ppgccbibliography{./bib/bibliografia}
% apêndices, se houver
%\begin{appendices}
%\chapter{Um apêndice}
%...conteúdo do apêndice...
%\chapter{Outro apêndice}
%...conteúdo do apêndice...
%\end{appendices}
% anexos, se houver
\begin{attachments}
\chapter{Cronograma}
\label{att:cronograma}
\end{attachments}
\end{document}