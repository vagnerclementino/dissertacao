\documentclass[msc,proposal]{ppgccufmg} % ou [msc] para dissertações
										% de mestrado. Para propostas ou
										% projetos, usar [phd,project],
										% [msc,proposal], etc.
\usepackage[brazil]{babel} % ajusta vários detalhes para
						   % documentos escritos em português,
						   % principalmente hifenização
\usepackage[T1]{fontenc}   % permite a hifenização de
						   % palavras acentuadas
\usepackage[utf8x]{inputenc} % ou [utf8x] para quem prefere
							 % usar a codificação UTF-8
\usepackage{graphicx} % define o comando \includegraphics
					  % para a inclusão de figuras
\usepackage[square]{natbib} % permite citações naturalmente
							% integradas ao texto
\usepackage[a4paper,
portuguese,
bookmarks=true,
bookmarksnumbered=true,
linktocpage,
colorlinks=true,
citecolor=black,
urlcolor=black,
linkcolor=black,
filecolor=black,
]{hyperref}

\begin{document}
\ppgccufmg{
title={Uma linguagem para modelagem conceitual em XBRL},
author={Vagner Clementino},
university={Universidade Federal de Minas Gerais},
course={Ciência da Computação},
address={Belo Horizonte},
date={2015-05},
advisor={Rodolfo Rezende},
abstract={Resumo}{resumo},
}
\chapter{Introdução}
\label{ch:Contexto}
No cenário atual, onde os sistemas de informação tem se tornado grandes e complexos, vemos a existência de um ecossistema de sistemas, também conhecidos como \textit{System of System} (SoS) \cite{Nakagawa:2013:SAF:2489850.2489853}. Organizações de grande porte, como os governos nacionais, precisam projetar sistemas de sistema ao invés vez de sistemas isolados a fim de enfrentar desafios tais como: $(i)$ colaboração entre organizações financiadas e geridas de forma independente; $(ii)$ migração para um ambiente orientado a serviços (SOA\footnote{Service Oriented Architecture }); $(iii)$ processos de testes e verificação da conformidade para sistemas de sistemas. Neste contexto, surge a necessidade do desenvolvimento de \textit{abordagens, técnicas e tecnologias} para a interação e evolução dos SoS.
     
No tocante a interoperabilidade de dados diversos padrões vêm sendo propostos. Mais recentemente o formato JSON \cite{RFC4627} vêm crescendo em popularidade, contudo, o intercâmbio de dados é realizado primordialmente através da XML (Extensible Markup Language\footnote{\url{http://www.w3.org/XML/}}) e seus derivados. Na área medica, o padrão HL7 V3 messagem\footnote{\url{http://www.hl7.org/}} vêm sendo largamente adotado para troca de mensagem entre sistemas médicos \cite{Andrikopoulos:2013:TEO:2491845.2491890}{}. No contexto dos Sistemas de Informação Geográficas (SIG's) a GML (Geography Markup Language\footnote{\url{http://www.opengeospatial.org/standards/gml}}) consolidou-se com o principal instrumento para interoperabilidade de dados geográficos \cite{gmlpaper}{}. Na área financeira diversas linguagens de marcação vem vêm sendo utilizadas para o intercâmbio de informações na Internet: \textit{Open Financial Exchange}\footnote{\url{http://www.ofx.net/}} (OFX), \textit{Eletronic Business Using XML}\footnote{\url{http://www.ebxml.org}} (ebXML), \textit{Financial Information eXchange}\footnote{\url{https://fixspec.com/}} (FIX), \textit{Market Data Definition Language}\footnote{\url{http://www.mddl.org}}, dentre outras \cite{xbrl_conceitos_aplicacoes}{}.

Com o advento da Web as linguagens de marcação cresceram em importância sobretudo devido a necessidade de se adicionar significado a informações sendo transferidas. Contudo, o padrão HTML (Hypertext
Markup Language) foi desenvolvido com objetivo de descrever \textit{como} a informação deve ser apresentada e não possui qualquer compromisso com o significado da informação. Nos meados da década de 1980,  a Organização para Padronização Internacional (ISO) propôs uma metalinguagem padrão a fim de para etiquetar informações com conteúdo semântico. Esta linguagem foi denominada como \textit{Standard Generalized Markup Language} (SGML) \cite{smith1988sgml}.

A linguagem SGML, embora fosse capaz de definir diferentes tipos de marcação, a sua flexibilidade trouxe como preço a complexidade. O conceito era correto, todavia, necessitava precisava de ser mais simples. Com este objetivo em mente isso objetivo em mente, um pequeno grupo de trabalho e um maior número de interessados começou a trabalhar na
meados dos anos 1990 em um subconjunto de SGML conhecido como \textit{Extensible Markup Language} (XML). A primeira versão foi publicada em 1996 e, dois anos mais tarde, o World Wide Web Consortium\footnote{\url{http://www.w3.org}} (W3C) publicou uma versão revisada \cite{Fawcett:2012:BX:2408362}{}.

Conforme exposto, a XML foi especificada a partir da SGML, na tentativa de se resolver as limitações da HTML
e da SGML. Neste sentido, um documento em XML pode ser publicado na Web, interpretado por pessoas
ou processado por aplicações. Apropriando-se desta características da XML, diversos linguagens vêm sendo propostas com a finalidade de troca de informações.

A XBRL (\textit{eXstensible Business Reporting Language}) é uma linguagem para divulgação e intercâmbio de informações financeiras baseada em XML (\cite{xbrl_conceitos_aplicacoes}). O padrão vem sendo adotado por diversas instituições e empresas em todo mundo com o suporte de um consórcio global\footnote{\url{www.xbrl.org}} com mais de 650 membros que incentivam a criação de jurisdições locais. Atualmente o consórcio conta com 24 jurisdições, sendo que em países como  Estados Unidos, Grã-Bretanha e Austrália, a XBRL já é a linguagem oficial para entrega de relatórios à órgãos de governo. A figura \ref{fig:world_map} exibe os países que estão promovendo a adotação da XBRL.

\begin{figure}[hbtp]
\centering
\includegraphics[width=.75\textwidth]{img/world-map.png}
\caption{O uso da XBRL no mundo}
\label{fig:world_map}
\end{figure}

Os estudos para definição da XBRL iniciaram em 1998 nos Estados Unidos pelo contador Charles Hoffman com apoio \textit{Institute of Certified Public Accountants} (AICPA). O objetivo era utilizar a XML padronizar a divulgação de informações financeiras em formato eletrônico. A primeira versão do padrão, a XBRL 1.0, foi lançada em 2000, sendo a versão 2.0 lançada em dezembro de 2001. No mês de dezembro de 2003, foi lançada a versão 2.1 (\cite{hoffman_2006}), corrigindo algumas deficiências detectadas na versão anterior. A versão 2.1 se mantêm como a versão mais atual e estável da XBRL.

A linguagem XBRL define a estrutura básica dos documentos de instância, que são aqueles que portam os dados, e possibilita ainda a especificação de taxonomias que podem ser criadas para acomodar particularidades de cada organização por meio da introdução de novos elementos, denominados conceitos. Neste sentido, a linguagem possui elementos que facilitam a sua extensão e, por consequência, a adoção em diversos contextos.

Apesar de sua crescente adoção, falta à XBRL uma notação que facilite a sua modelagem e a comunicação entre os diferentes \textit{stakeholders}. A necessidade de um modelo abstrato para a XBRL foi manifestada pelo próprio \textit{XBRL Consortium}. Em 2010 (\cite{xbrl_preserve_promote_particite}), o consórcio declarou que a criação de um modelo conceitual é uma das seis iniciativas que darão suporte a contínua adoção do padrão. Com o objetivo de preencher esta lacuna propõe neste documento o desenvolvimento de uma linguagem conceitual para a XBRL. A seguir descreve-se como o documento está estruturado.

No Capítulo \ref{ch:justificativa} discute-se as justificas para adoção da XBRL bem como da criação de uma linguagem conceitual. O Capítulo \ref{ch:revisao} apresenta-se a revisão da literatura no tocante a criação de ontologias em diversos campos dos conhecimento, especialmente para área financeira e contábil. No Capítulo \ref{ch:metodologia} é discutida a metologia a ser aplicada. No Apêndice \ref{attch:cronograma} é exibido o cronograma do trabalho.

\chapter{Justificativa}
\label{ch:justificativa}

Nas organizações as informações contábeis e financeiras são armazenadas em diversos formatos (planilhas eletrôncias, documentos de texto, bancos de dados relacionais e etc). Não obstante, se faz necessário a transformação destas informações em um formato único a fim de facilitar a sua recuperação bem como a sua transmissão para outros sistemas. O processo de transformação e redirecionamento da informação internamente na organização, entre a organização e sua filiais ou mesmo entre a empresa e os governos. Este fluxo de informação, especialmente para a geração de relatórios, é exibido na figura \ref{fig:fluxo_dados}.

\begin{figure}[hbtp]
\centering
\includegraphics[width=.75\textwidth]{img/fluxo_informacoes.png}
\caption{Fluxo de dados financeiros em uma organização. Adaptado de (\cite{bergeron2004essentials}).}
\label{fig:fluxo_dados}
\end{figure}

Se pensarmos em uma organização que necessitem receber informações de diversos locais, como por exemplo o governo federal de uma país que solicite a prestação de contas de estados e municípios, onde cada ente possui seu próprio sistema para registro dos fatos financeiros. Neste contexto, haverá a necessidade de se criar diferentes interfaces para a conversão de formatos e padrões de contabilização. A figura \ref{} ilustra este cenário.


\begin{figure}[hbtp]
\centering
\includegraphics[width=.75\textwidth]{img/interfaces.png}
\caption{Diferentes interfaces para a interoperabilidade entre sistemas. Adaptado de (\cite{bergeron2004essentials}).}
\label{fig:fluxo_dados}
\end{figure}

A solução utilizada para minimizar estes problemas é adoção de uma linguagem de marcação que facilite o intercâmbio e apresentação na Internet, bem como proporcione o armazenamento em qualquer base de dados. Neste sentido a XBRL vêm sendo adotado como padrão em diversos países. O processo de troca de informações financeiras é simplificado pela XBRL tendo em vista que a transformação da informação original é realizada uma única vez para o formato XBRL. Posteriormente a informação poderá ser reutilizada e/ou distribuída automaticamente para diversos outros formatos. A figura \ref{fig:fluxo_info_xbrl} exibe a simplificação na troca de informação com a adoção do XBRL.

\begin{figure}[hbtp]
\centering
\includegraphics[width=.75\textwidth]{img/fluxo_info_xbrl.png}
\caption{Melhoria no fluxo de informação com XBRL. Adaptado de (\cite{hoffman2001xbrl}).}
\label{fig:fluxo_info_xbrl}
\end{figure}

Apesar de sua importância e crescente adoção, o \textit{XBRL Consortium} sentiu a necessidade de definir um modelo abstrato da XBRL (\cite{xbrl_preserve_promote_particite}) que facilitasse a compreensão da linguagem pelos profissionais da tecnologia da informação.Em 2012 foi proposto o \textit{XBRL Abstract Model 2.0}\footnote{\url{http://www.xbrl.org/Specification/abstractmodel-primary/PWD-2012-06-06/abstractmodel-primary-pwd-2012-06-06.html}} que consiste basicamente de uma versão estendida da UML com objetivo de captura aspectos semânticos da XBRL. 

Um modelo abstrato deverá remover do seu escopo qualquer questão relativa à implementação do objeto modelado. Neste sentido, no caso de XBRL, o modelo abstrato deveria remover todas as referências a XML, o que não ocorreu no XBRL Abstract Model. Ademais, apesar da UML ser largamente utilizada entre os profissionais de Tecnologia da Informação, o seu uso para comunicação com os demais stakeholders muitas vezes não é o ideal (\cite{peixoto2008comparison}).

Neste contexto se faz necessário a proposição de linguagem conceitual para a XBRL com as seguintes características:
 \begin{itemize}
 	\item Não contenha questões relativa ao XML
 	\item Possibilite o desenho de aplicações que utilizem a XBRL
 	\item Facilite a comunicação entre os diversos stakeholders envolvidos no domínio da XBRL.
 \end{itemize}
 
No próximo capítulos iremos revisar a literatura relativo à definição de ontologias, especialmente no domínio contábil/financeiro.

\chapter{Revisão da Literatura}
\label{ch:revisao}

\chapter{Metodologia}
\label{ch:metodologia}

% Incluindo bibliografia:
\ppgccbibliography{./bib/bibliografia}
% apêndices, se houver
%\begin{appendices}
%\chapter{Um apêndice}
%...conteúdo do apêndice...
%\chapter{Outro apêndice}
%...conteúdo do apêndice...
%\end{appendices}
% anexos, se houver
\begin{attachments}
\chapter{Cronograma}
\label{attch:cronograma}
\end{attachments}
\end{document}