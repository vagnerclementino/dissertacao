\documentclass{article}
\usepackage{graphicx,url}
\usepackage[brazil]{babel}
\usepackage[utf8]{inputenc}
\usepackage{enumerate}
\usepackage{tabularx}
\usepackage{multirow}
\usepackage{amsmath}
\usepackage[table,xcdraw]{xcolor}

\title{Rascunho Proposta de Dissertação:\\
Recomendação de Demandas Utilizando Rede Neural
}
\author{Vagner Clementino \\
       \url{vagnercs@dcc.ufmg.br}}
\date{\today}


\begin{document}

\maketitle

\section{Introdução}
\label{sec:intro}

\section{Justificativa}
\label{sec:justificativa}

\begin{itemize}
\item Redução da troca de contexto (context switch)
\item Aumento da produtividade
\item Trabalhos de recomendação de novas demandas utilizam análise textual
  através de técnicas de Recuperação da Informação (Information Retrieve - IR)

\end{itemize}

\section{Revisão da Literatura}
\label{sec:revisao}

Em \cite{5741246} é proposto um processo denominado PASM (Process for Arranging
Software Maintenance Requests) que propõe lidar com tarefas de manutenção como
projetos de software. Para tanto, utilizou-se técnicas de análise de
agrupamento (clustering) a fim de melhor compreender e comparar as demandas de
manutenção. Os resultados demostraram que depois de adotar PASM os
desenvolvedores têm dedicado mais tempo para análise e validação e menos tempo
para as tarefas de execução e de codificação.

Em \cite{101186} é proposto um plugin para a ferramenta de Controle de Demanda -
Issue Tracking System (ITS) Bugzilla \footnote{\url{https://www.bugzilla.org/}}
que recomenda novos bugs para um desenvolvedor baseado no bug que ele esteja
tratando atualmente. O plugin sugere novos bugs com base em técnicas de
Recuperação de Informação\cite{baeza1999modern}.

\section{Metologia}
\label{sec:medotologia}

\subsection{Revisão Sistemática da Literatura}
\label{sec:slr}

Será realizada uma Revisão Sistemática da Literatura seguindo as diretrizes por
\cite{keele2007guidelines} a fim de responder as seguintes questões de
pesquisa:


\begin{itemize}
  \item \textbf{$Q1$}: Quais são as metodologias/técnicas utilizar para
    recomendação de novas demandas?
  \item \textbf{$Q2$}: Qual procedimento utilizado para avaliar a
    técnica/metodologia proposta?
  \item \textbf{$Q3$}: Em qual tipo de projeto (comercial ou código-aberto) a
    metodologia/técnica proposto foi aplicada?
\end{itemize}





\subsection{Prova de Conceito}
\label{sec:prova-de-conceito}

Como prova de conceito da técnica proposta será construída uma ferramenta que
possibilite a recomendação de novas demandas a serem tratadas baseada em uma ``demanda-semente''.


\subsection{Avaliação}
\label{sec:avaliacao}

Será realizada um \textit{Quasi-Experimento} \cite{wohlin2012experimentation}
com o desenvolvedores da Empresa de Informática de Belo Horizonte - PRODABEL
visando avaliar as sugestões proposta neste trabalho comparada com as demandas
sugeridas pela gerência imediata e por um outro programador do mesmo setor. O
participante deverá informar qual ``lista de sugestão'' aumentaria sua
produtividade e reduziria a troca de contexto.


\section{Cronograma e Próximas Atividades}
\label{sec:cronograma}


\bibliographystyle{unsrt}%Used BibTeX style is unsrt
\bibliography{rascunho}

\end{document}
