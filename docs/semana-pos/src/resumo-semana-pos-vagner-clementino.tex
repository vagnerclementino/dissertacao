% !TeX encoding   = UTF-8
\documentclass[12pt]{article}

\usepackage{sbc-template}

\usepackage{graphicx,url}
\usepackage[brazil]{babel}
\usepackage[utf8]{inputenc}
\usepackage{graphicx}   %Package para figuras
\usepackage{enumerate}
\usepackage{tabularx}
\usepackage{multirow}
\usepackage[table,xcdraw]{xcolor}
\usepackage{todonotes}


\sloppy

\title{Um estudo de ferramentas de \\
	Suporte de Problemas de Software}

\author{Vagner Clementino\inst{1}}

\address{Departamento de Ciência da Computação\\
        Universidade Federal de Minas Gerais (UFMG)\\
  \email{vagnercs@dcc.ufmg.br}
}

\date{Maio de 2016}
\begin{document}

\maketitle
%Sugiro um parágrafo para contexto e motivação, um para o problema e breve estado da arte (ex: o atual estado da arte não considera...) e outro para os objetivos e resultados (esperados e/ou alcançados até o momento).

\todo[inline]{BEGIN: Parágrafo sobre Contexto}
Dentro do ciclo de vida de um produto de software o processo de manutenção tem
papel fundamental. Devido ao seu alto custo, em alguns casos chegando em 60\%
do custo total do software \cite{kaur2015review}, este processo deve ter sua
importância considerada tanto pela comunidade científica quanto pela indústria.
\todo[inline]{END: Parágrafo sobre Contexto}

\todo[inline]{BEGIN: Parágrafo sobre Motivação}
\todo[inline]{END: Parágrafo sobre Motivação}


\todo[inline]{BEGIN: Parágrafo sobre Problema}
\todo[inline]{END: Parágrafo sobre Problema}


\todo[inline]{BEGIN: Parágrafo sobre Estado da Arte}
\todo[inline]{END: Parágrafo sobre Estado da Arte}


\todo[inline]{BEGIN: Parágrafo sobre Resultados}
\todo[inline]{END: Parágrafo sobre Resultados}



\bibliographystyle{sbc}
\bibliography{../bib/resumo-semana-pos-vagner-clementino}

\end{document}
