\documentclass[msc,proposal,hidelot,hideabstract]{ppgccufmg} % ou [msc] para dissertações
                                        % de mestrado. Para propostas ou
                                        % projetos, usar [phd,project],
                                        % [msc,proposal], etc.
\usepackage[brazil]{babel} % ajusta vários detalhes para
                           % documentos escritos em português,
                           % principalmente hifenização
\usepackage[T1]{fontenc}   % permite a hifenização de
                           % palavras acentuadas
\usepackage[utf8x]{inputenc} % ou [utf8x] para quem prefere
                             % usar a codificação UTF-8
\usepackage{graphicx} % define o comando \includegraphics
                      % para a inclusão de Figuras
%\usepackage[square]{natbib} % permite citações naturalmente
                            % integradas ao texto
\usepackage[a4paper,
portuguese,
bookmarks=true,
bookmarksnumbered=true,
linktocpage,
colorlinks=true,
citecolor=black,
urlcolor=black,
linkcolor=black,
filecolor=black,
]{hyperref}
\usepackage[table,xcdraw]{xcolor}

\begin{document}
\ppgccufmg{
title={Uma Ferramenta para\\
Recomendação de \\
Demandas Similares},
author={Vagner Clementino dos Santos},
university={Universidade Federal de Minas Gerais},
course={Ciência da Computação},
address={Belo Horizonte},
date={2015-11},
advisor={Rodolfo F. Resende},
abstract={Resumo}{resumo},
}
\chapter{Introdução}
\label{ch:intro}
Dentro do ciclo de vida do produto de software o processo de manutenção tem
papel fundamental. Apesar de não ter merecido tanta atenção quanto a parte de
projeto e desenvolvimento de software, no últimos anos o processo de manter o
software vem ganhando relevância devido, primordialmente, à seu custo
associado.

No Capítulo \ref{ch:justificativa} discute-se as justificativas para adoção da
XBRL bem como da definição das diretrizes. O Capítulo \ref{ch:revisao}
apresenta-se a revisão da literatura a adoção da XBRL e dos trabalhos que
propões diretrizes para construção de ferramentas nas Relatórios de Negócio. No
Capítulo \ref{ch:metodologia} é discutida a metologia a ser aplicada. No
Capítulo \ref{ch:conclusao_trab_futuros}, especialmente na Tabela \ref{tab:cronograma} é exibido o cronograma do trabalho.

\chapter{Justificativa}
\label{ch:justificativa}

\begin{itemize}
\item Redução da troca de contexto (context switch)
\item Aumento da produtividade
\item Trabalhos de recomendação de novas demandas utilizam análise textual através de técnicas de Recuperação da Informação (Information Retrieve - IR)
\end{itemize}

Dois problemas podem afetar a produtividade do desenvolvedor, e por conseguinte
toda a equipe de manutenção, são as demandas duplicadas e a troca de
contexto. No primeiro caso, temos um mesmo problema sendo reportado por
diversos usuários em demandas distintas. No segundo, a dificuldade analisar
duas demandas totalmente distintas uma da outra. Diversas trabalhos vêm sendo
realizados no intuito de remover bug duplicados da base de dados dos Sistemas
de Controle de Demandas (Issue Tracking System - ITS).
Com relação ao problema da troca de contexto, existem na literatura trabalhos
com objetivo de recomendar demandas similares, evitando a repetição de algumas
tarefas e por consequência a produtividade do desenvolvedor.
O problema de recomendar demandas similares pode ser definido formalmente da
seguinte forma:

Seja I o conjunto de demandas de manutenção para um sistema S qualquer de
cardinalidade n. Seja i
um demanda pertencente ao conjunto I e que foi atribuída para um desenvolver
d. Pede-se que seja encontrado um subconjunto J de I, de tamanho $k << n$, tal
que para todo j no subconjunto J seja similar a i em um grau maior ou igual a s.


\chapter{Revisão da Literatura}
\label{ch:revisao}

Em \cite{5741246} é proposto um processo denominado PASM (Process for Arranging
Software Maintenance Requests) que propõe lidar com tarefas de manutenção como
projetos de software. Para tanto, utilizou-se técnicas de análise de
agrupamento (clustering) a fim de melhor compreender e comparar as demandas de
manutenção. Os resultados demostraram que depois de adotar PASM os
desenvolvedores têm dedicado mais tempo para análise e validação e menos tempo
para as tarefas de execução e de codificação.

Em \cite{101186} é proposto um plugin para a ferramenta de Controle de Demanda -
Issue Tracking System (ITS) Bugzilla \footnote{\url{https://www.bugzilla.org/}}
que recomenda novos bugs para um desenvolvedor baseado no bug que ele esteja
tratando atualmente. O plugin sugere novos bugs com base em técnicas de
Recuperação de Informação\cite{baeza1999modern}.


\chapter{Metodologia}
\label{ch:metodologia}

O processo de desenvolvimento deste trabalho pode ser dividido em quatro partes
principais: $I$\textit{ - Revisão Sistemática da Literatura}; $II$\textit{ -
  Pequisa com Usuário (Survey)}; $III$\textit{ - Construção da Ferramenta
  (Prova de Conceito)}; $IV$\textit{ - Avaliação (Estudo de Caso)}. Cada uma das etapas é detalhada nas próximas seções.

\section{Revisão Sistemática da Literatura}
\label{sec:revisao_sistematica}

Uma \textit{Revisão Sistemática da Literatura} - SLR (do inglês Systematic Literature Review) é uma
metodologia científica cujo objetivo é identificar, avaliar e interpretar
\textit{toda} pesquisa \textit{relevante} sobre uma questão de pesquisa, área
ou fenômeno de
interesse\cite{keele2007guidelines,wohlin2012experimentation}. Neste trabalho
será utilizada as diretrizes proposta \cite{keele2007guidelines} no qual uma
Revisão Sistemática deve seguir os seguintes passos:

\begin{enumerate}
  \item \textbf{Planejamento}
  \begin{enumerate}
    \item \textit{Identificar a necessidade da Revisão}
    \item \textit{Especificar questões de pesquisa}
    \item \textit{Desenvolver o Protocolo da Revisão}
  \end{enumerate}
  \item \textbf{Condução/Execução}
  \begin{enumerate}
    \item \textit{Seleção dos Estudos Primários}
    \item \textit{Análise da qualidade dos Estudos Primários}
     \item \textit{Extração dos Dados}
     \item \textit{Sintetização dos Dados}
   \end{enumerate}
  \item \textbf{Escrita/Publicação}
  \begin{enumerate}
    \item \textit{Redigir documento com os resultados da Revisão}
    \item \textit{Redigir documento com lições aprendidas}
  \end{enumerate}
\end{enumerate}

Será realizada uma Revisão Sistemática da Literatura seguindo as diretrizes por
\cite{keele2007guidelines} a fim de responder as seguintes questões de
pesquisa:


\begin{itemize}
  \item \textbf{$Q1$}: Quais são as metodologias/técnicas utilizar para
    recomendação de novas demandas?
  \item \textbf{$Q2$}: Qual procedimento utilizado para avaliar a
    técnica/metodologia proposta?
  \item \textbf{$Q3$}: Em qual tipo de projeto (comercial ou código-aberto) a
    metodologia/técnica proposto foi aplicada?
\end{itemize}


\section{Prova de Conceito}
\label{sec:prova-conceito}

Como prova de conceito da técnica proposta será construída uma ferramenta que
possibilite a recomendação de novas demandas a serem tratadas baseada em uma ``demanda-semente''.
Com intuito de reduzir o problema da troca de contexto e ao mesmo aumentar a
produtividade de equipes de manutenção de software é proposto uma ferramenta
para recomendação de demandas similares utilizando Redes Neurais.


\section{Avaliação}
\label{sec:avaliacao}

Será realizada um \textit{Quasi-Experimento} \cite{wohlin2012experimentation}
com o desenvolvedores da Empresa de Informática de Belo Horizonte - PRODABEL
visando avaliar as sugestões proposta neste trabalho comparada com as demandas
sugeridas pela gerência imediata e por um outro programador do mesmo setor. O
participante deverá informar qual ``lista de sugestão'' aumentaria sua
produtividade e reduziria a troca de contexto.

Com o objetivo de avaliar a ferramentas prposto nes trebablho será realizado um
experimento utilizando a base de dados de demandas de manutenção de uma empresa
de software real. Dado que uma demanda i foi atribuida a um desenvolvedor d,
serão geradas 03 listas de suguestões: uma lista proposta pelo gerente imediato
de d; outra proposta por um desenvolver do mesmo setor de d; e a terceira
gerada por nossa ferramenta. Naturalmente o desenvolvedor não saberá a origem
de nenhum das listas.  De posse das três listas pediremos ao
desenvolvedor d que informe qual delas pode reduzir a troca de contexto e
aumentar sua produtividade.

\chapter{Conclusão e Trabalhos Futuros}
\label{ch:conclusao_trab_futuros}

Conforme pôde ser observado, a XBRL vêm se tornando de fato um padrão para
intercâmbio de dados contáveis e financeiros. Contudo, apesar de sua crescente
adoção, exitem poucos trabalhos e ferramentas que suportem o processo de geração
de Relatórios de Negócio naquela linguagem.  Tentando preencher esta lacuna, o
trabalho ora proposto pretende definir um conjunto de diretrizes para criação
de ferramentas para a extração de Relatórios de Negócio que suportem a XBRL. Para tanto, a tabela \ref{tab:cronograma} descreve as atividades que serão realizadas para atingir este objetivo.

\begin{table}[!h]
\resizebox{\textwidth}{!}{%
\begin{tabular}{|c|l|c|c|}
\hline
\rowcolor[HTML]{C0C0C0}
{\textbf{\#}} & \multicolumn{1}{c|}{\cellcolor[HTML]{C0C0C0}{\textbf{Atividade}}} & {\textbf{Início(MM/AAAA)} } & {\textbf{Término(MM/AAAA)}}\\ \hline
1 & Revisão da literatura & 07/2015 & 08/2015 \\ \hline
2 & Ponto de Controle 01: reunião com o orientador & 08/2015 & 08/2015 \\ \hline
3 & Revisão da especificação XBRL & 09/2015 & 09/2015 \\ \hline
4 & Ponto de Controle 02: reunião com o orientador & 09/2015 & 09/2015 \\ \hline
5 & Revisão Sistemática da Literatura & 10/2015 & 12/2015 \\ \hline
6 & Ponto de Controle 03: reunião com o orientador & 10/2015 & 10/2015 \\ \hline
7 & Ponto de Controle 04: reunião com o orientador & 11/2015 & 11/2015 \\ \hline
8 & Ponto de Controle 05: reunião com o orientador & 12/2015 & 12/2015 \\ \hline
9 & Pesquisa com Usuário & 01/2016 & 02/2016 \\ \hline
10 & Ponto de Controle 06: reunião com o orientador & 01/2016 & 01/2016 \\ \hline
11 & Ponto de Controle 07: reunião com o orientador & 02/2016 & 02/2016 \\ \hline
12 & Ponto de Controle 08: reunião com o orientador & 03/2016 & 03/2016 \\ \hline
13 & Prova de Conceito: Construção da Ferramenta & 03/2016 & 04/2016 \\ \hline
14 & Ponto de Controle 09: reunião com o orientador & 04/2015 & 04/2015 \\ \hline
15 & Redigir a dissertação & 04/2015 & 05/2016 \\ \hline
16 & Ponto de Controle 10: reunião com o orientador & 05/2016 & 05/2016 \\ \hline
17 & Defesa da dissertação & 07/2016 & 07/2016 \\ \hline
\end{tabular}
}
\caption{Cronograma do projeto}
\label{tab:cronograma}
\end{table}


% Incluindo bibliografia:
\ppgccbibliography{../bib/bibliografia}
\end{document}
