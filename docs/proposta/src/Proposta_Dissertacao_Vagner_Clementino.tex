\documentclass[msc,proposal,hidelot,hideabstract]{ppgccufmg} % ou [msc] para dissertações
                                        % de mestrado. Para propostas ou
                                        % projetos, usar [phd,project],
                                        % [msc,proposal], etc.
\usepackage[brazil]{babel} % ajusta vários detalhes para
                           % documentos escritos em português,
                           % principalmente hifenização
\usepackage[T1]{fontenc}   % permite a hifenização de
                           % palavras acentuadas
\usepackage[utf8x]{inputenc} % ou [utf8x] para quem prefere
                             % usar a codificação UTF-8
\usepackage{graphicx} % define o comando \includegraphics
                      % para a inclusão de Figuras
%\usepackage[square]{natbib} % permite citações naturalmente
                            % integradas ao texto
\usepackage[a4paper,
portuguese,
bookmarks=true,
bookmarksnumbered=true,
linktocpage,
colorlinks=true,
citecolor=black,
urlcolor=black,
linkcolor=black,
filecolor=black,
]{hyperref}
\usepackage[table,xcdraw]{xcolor}
\usepackage{amsmath}
\begin{document}
\ppgccufmg{
title={Uma Ferramenta para\\
Recomendação de \\
Demandas Similares},
author={Vagner Clementino dos Santos},
university={Universidade Federal de Minas Gerais},
course={Ciência da Computação},
address={Belo Horizonte},
date={2015-11},
advisor={Rodolfo F. Resende},
abstract={Resumo}{resumo},
}
\chapter{Introdução}
\label{ch:intro}
Dentro do ciclo de vida do produto de software o processo de manutenção tem
papel fundamental. Apesar de não ter merecido tanta atenção quanto a parte de
projeto e desenvolvimento de software, no últimos anos o processo de manter o
software vem ganhando relevância devido, primordialmente, à seu custo
associado.

No Capítulo \ref{ch:justificativa} . O Capítulo \ref{ch:revisao} . No
Capítulo \ref{ch:metodologia} é discutida a metologia a ser aplicada. No
Capítulo \ref{ch:conclusao_trab_futuros}, especialmente na Tabela \ref{tab:cronograma} é exibido o cronograma do trabalho.

\chapter{Justificativa}
\label{ch:justificativa}
Desde o final da década de 1970 percebe-se o aumento do custo referente as
atividades de  manutenção de software. Na
Tabela \ref{tab:software-cost}, adaptada de  \cite{koskinen2003software},  é possível verificar esta tendência. Em um trabalho mais
recente \cite{1423995}, Yong \& Mookerjee propõe um modelo que reduz o custos de
manutenção e reposição durante a vida útil de um sistema de software. O modelo
proposto demonstrou quem em algumas situações é \textit{melhor substituir um sistema do que mantê-lo}.

\begin{table}[ht]
\resizebox{\textwidth}{!}{%
\begin{tabular}{|c|c|c|c|}
\hline
\rowcolor[HTML]{FFFFFF}
\textbf{Ano} & \textbf{Proporção} & \multicolumn{1}{c|}{\cellcolor[HTML]{FFFFFF}\textbf{Metodologia}} & \multicolumn{1}{c|}{\cellcolor[HTML]{FFFFFF}\textbf{Referência}} \\ \hline
2000 & \textgreater90\% & $\frac{\text{(Custo com manutenção e evolução do software)}}{\text{(Custo total do software)}}$ &\cite{846201} \\ \hline
1993 & 75\% & $\frac{\text{(Manutenção de Software)}}{\text{(Orçamento com Sistemas de Informação)}}$ & \cite{swe.cost.legacy2}\\ \hline
1990 & \textgreater90\% & $\frac{\text{(Custo com manutenção de sistema)}}{\text{(Custo total com Software)}}$ & \cite{moad1990maintaining}\\ \hline
1988 & 60-70\% & $\frac{\text{(Manutenção de software)}}{\text{(Orçamento com a operação de Sistemas de Gerenciamento da Informação)}}$ & \cite{Port1988}\\ \hline
1984 & 65-75\% & $\frac{\text{(Esforço gasto em manutenção de software)}}{\text{(Tempo total disponível para esforço em Engenharia de Software)}}$ & \cite{McKee:1984:MFD:1499310.1499334}\\ \hline
1981 & 50\% & $\frac{\text{(Tempo gasto com manutenção de software)}}{\text{(Tempo Total)}}$ & \cite{Lientz:1981:PAS:358790.358796}\\ \hline
1979 & 67\% & $\frac{\text{(Custo com Manutenção)}}{\text{(Custo Total com Software)}}$ & \cite{Zelkowitz:1979:PSE:578504}\\ \hline
\end{tabular}
}
\caption{Proporção do Custo de Manutenção de Software. Adaptado de \cite{koskinen2003software}}
\label{tab:software-cost}
\end{table}


Com o objetivo de mensurar o custo relativo à manutenção de software
\cite{benaroch2013primary,5873461,ren2011research}, bem com melhoria as
atividades relacionadas ao processo de manutenção  mediante a proposição
de modelos \cite{April200973}, \cite{930170} e \cite{5741246} diversos
trabalhos vêm sendo propostos. Nesta mesma linha de pesquisa, alguns estudo
focam na melhoria da produtividade do desenvolvedor através de ações como a remoção de Requisições de Mudança -
Modification Request (MR) duplicadas
\cite{Alipour:2013:CAT:2487085.2487123,6671283,09639314} ou ainda recomendando
MR's similares que reduzem a mudança de contexto (context switch) \cite{5741246,101186}.

Em geral, afim de remover duplicadas ou sugerir MR's similares são utilizadas de
técnicas de Recuperação da Informação - Information Retrieve
\cite{baeza1999modern} tomando como base a descrição textual da Requisição \cite{101186,Runeson:2007:DDD:1248820.1248882}. Apesar dos resultados
relevantes obtidos por esta técnica verifica-se que ela é fortemente
dependente da forma que o solicitante da MR descreve a sua requisição. Além disso
há problema dos sinônimos, em que demandas similares mas, escritas com palavras
diferentes, podem não ser recuperadas.

Neste contexto é proposto uma ferramenta para recomendação de Requisições de
Mudanças similares utilizando Redes Neurais. Uma ferramenta deste tipo trará os
seguinte benefícios:

\begin{itemize}
\item \textit{Redução da troca de contexto (context switch)};
\item \textit{Aumento da produtividade};
\item \textit{Melhorar a qualidade das MR's sugeridas através de uma técnica mais
  potente do que aquelas da Recuperação da Informação}.
\end{itemize}

%Dois problemas podem afetar a produtividade do desenvolvedor, e por conseguinte
%toda a equipe de manutenção, são as demandas duplicadas e a troca de
%contexto. No primeiro caso, temos um mesmo problema sendo reportado por
%diversos usuários em demandas distintas. No segundo, a dificuldade analisar
%duas demandas totalmente distintas uma da outra. Diversas trabalhos vêm sendo
%realizados no intuito de remover bug duplicados da base de dados dos Sistemas
%de Controle de Demandas (Issue Tracking System - ITS).
%Com relação ao problema da troca de contexto, existem na literatura trabalhos
%com objetivo de recomendar demandas similares, evitando a repetição de algumas
%tarefas e por consequência a produtividade do desenvolvedor.
O problema de recomendar Requisições de Mudanças similares pode ser definido
formalmente conforme segue:

Seja $I$ o conjunto de Requisições de Mudanças (MR) em aberto para um
sistema $S$ qualquer. A cardinalidade de $I$ é dada por $n$. Seja $i$ um MR,
tal que $i \in I$, que foi atribuída para um desenvolver
$d$. Pede-se que seja encontrado um subconjunto $J \subset I$, de tamanho $k \ll n$, tal
que para todo $j$, tal que $j \in J$, seja similar a $i$ em um grau maior ou igual a
$s$. Onde $s$ é limiar inferior de similaridade. Neste caso, o problema se resume em
encontrar uma função de similaridade a ser aplicada a cada elemento $m \in I$,
onde $m \neq i$.

\chapter{Revisão da Literatura}
\label{ch:revisao}

No trabalho de Junio et al. \cite{5741246} é proposto um processo denominado PASM (Process for Arranging
Software Maintenance Requests) que propõe lidar com tarefas de manutenção como
projetos de software. Para tanto, utilizou-se técnicas de análise de
agrupamento (clustering) a fim de melhor compreender e comparar as demandas de
manutenção. Os resultados demostraram que depois de adotar PASM os
desenvolvedores têm dedicado mais tempo para análise e validação e menos tempo
para as tarefas de execução e de codificação.

\textit{NextBug} \cite{101186} é uma extensão (plugin) para a ferramenta de Controle de Demanda -
Issue Tracking System (ITS) Bugzilla\footnote{\url{https://www.bugzilla.org/}}
que recomenda novos bugs para um desenvolvedor baseado no bug que ele esteja
tratando atualmente. O objetivo da extensão é sugerir bugs com base em técnicas de
Recuperação de Informação \cite{baeza1999modern}.


\chapter{Metodologia}
\label{ch:metodologia}

O processo de desenvolvimento deste trabalho pode ser dividido nas seguintes
etapas $I$\textit{ - Revisão Sistemática da Literatura}; $II$\textit{ - Construção da Ferramenta
  (Prova de Conceito)}; $IV$\textit{ - Avaliação}. Cada uma das etapas é detalhada nas próximas seções.

\section{Revisão Sistemática da Literatura}
\label{sec:revisao_sistematica}

Uma \textit{Revisão Sistemática da Literatura} - SLR (do inglês Systematic Literature Review) é uma
metodologia científica cujo objetivo é identificar, avaliar e interpretar
\textit{toda} pesquisa \textit{relevante} sobre uma questão de pesquisa, área
ou fenômeno de interesse \cite{keele2007guidelines,wohlin2012experimentation}. Neste trabalho
será utilizada as diretrizes proposta \cite{keele2007guidelines} no qual uma
Revisão Sistemática deve seguir os seguintes passos:

\begin{enumerate}
  \item \textbf{Planejamento}
  \begin{enumerate}
    \item \textit{Identificar a necessidade da Revisão}
    \item \textit{Especificar questões de pesquisa}
    \item \textit{Desenvolver o Protocolo da Revisão}
  \end{enumerate}
  \item \textbf{Condução/Execução}
  \begin{enumerate}
    \item \textit{Seleção dos Estudos Primários}
    \item \textit{Análise da qualidade dos Estudos Primários}
     \item \textit{Extração dos Dados}
     \item \textit{Sintetização dos Dados}
   \end{enumerate}
  \item \textbf{Escrita/Publicação}
  \begin{enumerate}
    \item \textit{Redigir documento com os resultados da Revisão}
    \item \textit{Redigir documento com lições aprendidas}
  \end{enumerate}
\end{enumerate}

A SRL que será realizada visa responder as seguintes questões de
pesquisa:

\begin{itemize}
  \item \textbf{$Q1$}: Quais são as metodologias/técnicas utilizadas para
    recomendação de novas demandas?
  \item \textbf{$Q2$}: Quais são as metodologias/técnicas utilizadas para
    remoção de bugs duplicados?
    \item \textbf{$Q3$}: Qual procedimento utilizado para avaliar a
    técnica/metodologia proposta?
  \item \textbf{$Q4$}: Em qual tipo de projeto (comercial ou código-aberto) a
    metodologia/técnica proposto foi aplicada?
\end{itemize}

\section{Construção da Ferramenta}
\label{sec:prova-conceito}

Como prova de conceito da técnica proposta será construída uma ferramenta que
possibilite a recomendação Requisições de Mudança  a serem tratadas a partir de
uma demanda base, denominada ``demanda-semente''. Esta ferramenta utilizar da
técnica de Redes Neurais para encontrar as demais similares. O plano é que a
base de dados a ser utilizada seja de uma empresa de desenvolvimento a fim de validarmos
a ferramenta em uma situação real. Maiores detalhes sobre o processo de
avaliação estão descritos na Seção \ref{sec:avaliacao}.

\section{Avaliação}
\label{sec:avaliacao}

Com o objetivo de avaliar a ferramentas proposto neste trabalho será realizado um
\textit{Quasi-Experimento} \cite{wohlin2012experimentation} utilizando a base de dados de demandas de manutenção de uma empresa
de software real. O experimento consistirá de dado que uma demanda $i$ foi atribuída a um desenvolvedor $d$,
serão geradas 03 listas de sugestões: $(i)$ lista produzida pelo gerente imediato
de $d$;$(ii)$ lista feita por um desenvolver do mesmo setor de $d$; $(iii)$
gerada pela ferramenta proposta. Naturalmente o desenvolvedor não saberá a origem
de nenhum das listas.  De posse das três listas pediremos ao
desenvolvedor $d$ que informe qual delas pode reduzir a troca de contexto e
aumentar sua produtividade. Espera-se a lista proposta pela nossa ferramenta
possua o melhor desempenho na maioria dos casos.

\chapter{Conclusão e Trabalhos Futuros}
\label{ch:conclusao_trab_futuros}

Para tanto, a tabela \ref{tab:cronograma} descreve as atividades que serão realizadas para atingir este objetivo.

\begin{table}[ht]
\centering
\resizebox{\textwidth}{!}{%
\begin{tabular}{|c|l|c|c|}
\hline
\rowcolor[HTML]{EFEFEF}
\# & \multicolumn{1}{c|}{\cellcolor[HTML]{EFEFEF}\textbf{Atividade}} & \textbf{Início (MM/AAAA)} & \textbf{Término (MM/AAAA)} \\ \hline
01 & Revisão da Literatura & \textit{10/2015} & \textit{11/2015} \\ \hline
02 & Ponto de Controle 01 – Reunião com orientador sobre Revisão da Literatura & \textit{12/2016} & \textit{12/2016} \\ \hline
03 & Avaliação da Técnica de Rede Neural & \textit{01/2016} & \textit{01/2016} \\ \hline
04 & Ponto de Controle 02 – Reunião com orientador sobre a Técnica de Rede Neural & \textit{02/2016} & \textit{02/2016} \\ \hline
05 & Implementação da Ferramenta & \textit{02/2016} & \textit{04/2016} \\ \hline
06 & Ponto de Controle 03 – Avaliação da Ferramenta Avaliada & \textit{05/2016} & \textit{05/2016} \\ \hline
07 & Experimento de Avaliação da Ferramenta & \textit{05/2016} & \textit{05/2016} \\ \hline
08 & Ponto de Controle 04 – Avaliação do Experimento junto com o orientador & \textit{05/2016} & \textit{05/2016} \\ \hline
09 & Finalização do texto da dissertação & \textit{06/2016} & \textit{07/2016} \\ \hline
10 & Ponto de Controle 05 – Avaliação do texto da dissertação com o orientador & \textit{07/2016} & \textit{07/2016} \\ \hline
11 & Defesa da dissertação & \textit{07/2016} & \textit{07/2016} \\ \hline
\end{tabular}
}
\caption{Cronograma de execução do trabalho}
\label{tab:cronograma}
\end{table}

% Incluindo bibliografia:
\ppgccbibliography{../bib/bibliografia}
\end{document}
