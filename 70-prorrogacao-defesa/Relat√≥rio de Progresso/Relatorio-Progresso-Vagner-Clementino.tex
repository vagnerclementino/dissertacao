\documentclass[10pt,a4paper]{report}
\usepackage[utf8x]{inputenc}
\usepackage{ucs}
\usepackage{amsmath}
\usepackage{amsfonts}
\usepackage{amssymb}
\usepackage{makeidx}
\usepackage{graphicx}
\usepackage[brazil]{babel}
\usepackage{hyperref}
\author{Vagner Clementino}
\title{Relatório de Progresso do Mestrado\\
	Vagner Clementino}
	\makeindex

\begin{document}
\maketitle

	
\section{Objetivo do Documento}
\label{sec:objetivo}

O presente documento tem por objetivo subsidiar o pedido de prorrogação de defesa do aluno Vagner Clementino dos Santos. Ele detalha o histórico da vida acadêmica do aluno, descreve a situação atual do trabalho desenvolvido e apresenta um plano de trabalho visando a conclusão da dissertação. São apresentadas ainda a justificativa do pedido de prorrogação.

\section{Histórico do Aluno}
\label{sec:historico}

O aluno Vagner Clementino dos Santos ingressou no Programa de Pós Graduação em Ciência da Computação (PPGCC) no segundo semestre de 2014 sob a orientação do professor Rodolfo F. Resende. Trata-se de um aluno de tempo parcial que divide o seu tempo como aluno de mestrado e desenvolvedor na Empresa de Informática de Belo Horizonte (PRODABEL), onde cumpri a carga horária de 40 horas semanais. 

O período 2014/2, primeiro semestre como aluno do PPGCC, foi devotado às disciplinas do programa, sendo que a proposta de dissertação estava sendo planejada com base na literatura da área. No período 2014/1 a proposta de dissertação foi materializada e apresentada ao colegiado em maio de 2015. Sob o título de \textit{UMA LINGUAGEM PARA MODELAGEM CONCEITUAL EM XBRL} foi proposto o desenvolvimento de uma linguagem conceitual para a XBRL (\textit{eXtensible Business Reporting Language})\footnote{\url{www.xbrl.org}} que é uma linguagem para divulgação e intercâmbio de informações financeiras baseada em XML. Durante o desenvolvimento da referida proposta verificou-se a inviabilidade da mesma. Apesar da XBRL ter relevância do ponto de vista de profissionais envolvidos no processo de prestação de contas\footnote{\url{https://siconfi.tesouro.gov.br/}} verificou-se pouca literatura acadêmica sobre assunto. Neste sentido optou-se por alterar os caminhos da dissertação.

Em dezembro de 2015 foi apresentada uma nova proposta denominada  \textit{UM ESTUDO DE FERRAMENTAS DE SUPORTE DE PROBLEMAS DE SOFTWARE (FSPS)} que, em síntese, visa entender como este tipo de ferramenta está sendo melhorada ou estendida no contexto da transformação do processo de desenvolvimento de software, bem como da manutenção, de um modelo tradicional para um outro que incorpora práticas ágeis. As Ferramentas de Suporte de Problemas de Software são aquelas utilizadas pelas organizações para \textit{gerenciar as Requisições de Mudança}\cite{1703974} em Software. 

A proposta sobre as FSPS's encontra-se em desenvolvimento e foi revista após análise do colegiado ocorrida em junho/2016. O prazo final para submissão da Proposta de Dissertação alterada é o dia 03/07/2016. As atividades para a conclusão do trabalho de dissertação estão detalhadas na Seção \ref{situacao-atual}. 

No tocante aos créditos necessário a integralização do curso, o aluno  encontra-se em situação normal, faltando apenas os créditos relativos à disciplina Estágio em Docência, conforme a Tabela \ref{tab:historico}.

\begin{table}[htb]
	\centering
	\resizebox{\textwidth}{!}{%
		\begin{tabular}{|c|l|c|c|c|}
			\hline
			\textbf{Período} & \multicolumn{1}{c|}{\textbf{Nome Atividade}} & \textbf{Nota} & \textbf{Conc} & \textbf{Integ?} \\ \hline
			2014/2 & TOPICOS ESPECIAIS EM CIENC DA COMPUTACAO & 80 & B & Sim \\ \hline
			2014/2 & TOPICOS EM ENGENHARIA DE SOFTWARE & 91 & A & Sim \\ \hline
			2015/1 & PROJETO E ANALISE DE ALGORITMOS & 70 & C & Sim \\ \hline
			2015/1 & TOPICOS EM ENGENHARIA DE SOFTWARE & 73 & C & Sim \\ \hline
			2015/2 & TOPICOS EM ENGENHARIA DE SOFTWARE & 81 & B & Sim \\ \hline
			2016/1 & TOPICOS EM ENGENHARIA DE SOFTWARE & - & - & Sim \\ \hline
			2016/1 & ESTAGIO EM DOCENCIA I & - & - & Sim \\ \hline
			2016/1 & ELABORACAO DE TRABALHO FINAL & - & - & Sim \\ \hline
		\end{tabular}%
	}
	\caption{Créditos Integralizados Aluno}
	\label{tab:historico}
\end{table}

\section{Situação Atual do Trabalho}
\label{situacao-atual}

O trabalho de dissertação em andamento propõe as atividades listadas a seguir visando alcançar os objetivos propostos:

\begin{itemize}
	\item Mapeamento Sistemático da Literatura \cite{keele2007guidelines}
	\item Caracterização das Ferramentas de Suporte de Problemas de Software (FSPS)
	\item Pesquisa (Survey) com os desenvolvedores \cite{wohlin2012experimentation}
	\item Desenvolvimento de extensões para as FGRM's
\end{itemize}


Nas próximas subseções iremos detalhar o andamento de cada etapa deste trabalho de dissertação. Um resumo de cada atividade desenvolvida pode ser visualizado na Tabela \ref{tab:situacao}

\begin{table}[ht]
	\centering
	\resizebox{\textwidth}{!}{%
		\begin{tabular}{|c|l|c|}
			\hline
			\textbf{\#} & \multicolumn{1}{c|}{\textbf{Descrição}}                                                                                                      & \textbf{Situação}  \\ \hline
			01          & Caracterização dos Sistemas de Controle de Demandas com base nos requisitos comum a todos eles                                               & Em Desenvolvimento \\ \hline
			02          & Mapeamento das propostas de extensões existente na literatura                                                                                & Feito  \\ \hline
			03          & Coleta da opinião (survey)dos profissionais & Em Desenvolvimento \\ \hline
			04          & Desenvolvimento de extensões para os Sistema de Controle de Demandas                                                                         & Para Fazer         \\ \hline
		\end{tabular}%
	}
	\caption{Situação das Atividades da Dissertação}
	\label{tab:situacao}
\end{table}


\subsection{Mapeamento Sistemático da Literatura}
\label{subsec:revisao_sistematica}

Um \textit{Mapeamento Sistemático da Literatura}, também conhecido como Estudos de Escopo (Scoping Studies), tem como objetivo fornecer uma visão geral de determinada área de pesquisa, estabelecer se existem evidências de estudos sobre determinado tema e fornecer uma indicação da quantidade de trabalho na linha de pesquisa sob análise \cite{keele2007guidelines,wohlin2012experimentation}. Esta etapa do trabalho foi finalizada e os  seus resultados estão sendo utilizados para o planejamento do survey com os profissionais (vide Subseção \ref{subsec:survey} ).

\subsection{Caracterização das funcionalidades das Ferramentas de Suporte de Problemas de Software }
\label{subsec:caracterizacao}

Esta etapa do trabalho consistirá de um estudo exploratório com o objetivo de determinar quais são as funcionalidades comuns às Ferramentas de Gerenciamento de Requisição de Mudança (FGRM). O estudo consistirá na leitura da documentação de alguns FGRM para que de forma sistemática seja levantado quais são as funcionalidades oferecidas por determinada ferramenta. O método de escolha das FGRM será avaliado posteriormente, todavia, um possível ponto de partida é a lista disponível na Wikipédia que compara diversas FGRM\footnote{\url{https://en.wikipedia.org/wiki/Comparison_of_issue-tracking_systems}}.

A previsão de término desta etapa é julho/2016. O resultado deste estudo permitirá compreender melhor este tipo de ferramenta tomando como base as suas funcionalidades em comum. Também será possível propor  para as FGRM tendo em vista a possibilidade de determinar o conjunto mínimo de funções deste tipo de sistema. Uma outra utilização dos resultados desta etapa é no planejamento e desenvolvimento de um survey com profissionais envolvidos em manutenção de software.

\subsection{Pesquisa com Profissionais}
\label{subsec:survey}
Com o objetivo de coletar os aspectos mais importantes das FGRM's do ponto de
vista dos profissionais ligados à manutenção de software será realizada uma
pesquisa (survey). O planejamento e o desenho da pesquisa seguirá as diretrizes propostas em \cite{wohlin2012experimentation}.

Esta etapa será concluída após as atividades descritas nas Subseções \ref{subsec:revisao_sistematica} e \ref{subsec:caracterizacao}. Todavia, esta sendo desenvolvido um sistema para coletar dados dos possíveis participantes. Neste sentido a previsão de término desta etapa é setembro/2016.

\subsection{Extensões para Ferramentas de Gerenciamento de Requisição de Mudança}
\label{subsec:novas-extensoes}

A partir dos resultados do Mapeamento Sistemática, do Estudo de Caracterização das ferramentas e da Pesquisa com os profissionais pretende-se desenvolver uma ou mais extensão (plugin) para determinada FGRM. Cabe ressaltar que esta parte do trabalho será realizada caso o esforço seja compatível com os prazos e recursos disponíveis. Esta etapa ainda não foi iniciada e sua previsão de término é novembro/2016.

\section{Justificativa do Pedido}
\label{justificativa}

A prorrogação de defesa deste trabalho de dissertação é realizada com seguintes argumentos:

\begin{itemize}
	\item Sou aluno de tempo parcial e apenas nos últimos seis meses foi possível	reduzir a minha carga horária de trabalho para 06 horas. Antes, todavia, cumpria 40 semanais, desta forma, a primeira parte do 
	mestrado dediquei o meu tempo as disciplinas do curso. 
	\item  Na primeira proposta de dissertação, que tratava de uma linguagem
	\texttt{'XML-LIKE'} denominada XBRL, verifiquei que o tema não teria viabilidade de
	execução, especialmente por conta da falta de artigos 
	científicos sobre o assunto. 
	\item Efetuei uma nova proposta que foi submetida para análise em dezembro/2015 e foi parcialmente aceita em junho/2016, sendo necessário alguns ajustes. Estas mudanças irão gerar alterações
	no prazo de defesa.
\end{itemize}

Sendo assim solicito a este Colegiado a prorrogação da minha defesa. Estou
sempre à disposição para esclarecimentos.

\section{Plano de Trabalho}
\label{Plano de Trabalho}
 As atividades para atingir o objetivo da dissertação são exibidas de forma macro na Tabela \ref{tab:cronograma}. No desenvolvimento da dissertação será realizado um cronograma mais detalhado das atividades.

\begin{table}[ht]
	\label{tab:cronograma}
	\centering
	\resizebox{\textwidth}{!}{%
		\begin{tabular}{|l|l|c|c|}
			\hline
			\multicolumn{4}{|c|}{\textbf{CRONOGRAMA DISSERTAÇÃO}}                                                                                                                 \\ \hline
			\# & \multicolumn{1}{c|}{\textbf{Atividade}}                                               & \textit{\textbf{Início (MM/AAAA)}} & \textit{\textbf{Término (MM/AAAA)}} \\ \hline
			01 & Mapeamento  Sistemático da Literatura                                                 & \textit{03/2016}                   & \textit{06/2016}                    \\ \hline
			02 & Ponto de Controle 01 – Reunião com orientador sobre Revisão Sistemática da Literatura & \textit{06/2016}                   & \textit{06/2016}                    \\ \hline
			03 & Caracterização das Ferramentas de Gerenciamento de Requisição de Mudança              & \textit{06/2016}                   & \textit{07/2016}                    \\ \hline
			04 & Ponto de Controle 02 – Reunião com orientador sobre a Caracterização das FGRM           & \textit{07/2016}                   & \textit{07/2016}                    \\ \hline
			05 & Pesquisa com Profissionais                                                            & \textit{08/2016}                   & \textit{09/2016}                    \\ \hline
			06 & Ponto de Controle 03 – Reunião com orientador sobre a Pesquisa com o Profissionais    & \textit{09/2016}                   & \textit{09/2016}                    \\ \hline
			07 & Implementação da Ferramenta                                                           & \textit{09/2016}                   & \textit{10/2016}                    \\ \hline
			08 & Ponto de Controle 03 – Avaliação da Ferramenta Avaliada                               & \textit{10/2016}                   & \textit{10/2016}                    \\ \hline
			09 & Experimento de Avaliação da Ferramenta                                                & \textit{11/2016}                   & \textit{11/2016}                    \\ \hline
			10 & Ponto de Controle 04 – Avaliação do Experimento junto com o orientador                & \textit{11/2016}                   & \textit{11/2016}                    \\ \hline
			11 & Finalização do texto da dissertação                                                   & \textit{12/2017}                   & \textit{12/2016}                    \\ \hline
			12 & Ponto de Controle 05 – Avaliação do texto da dissertação com o orientador             & \textit{01/2017}                   & \textit{01/2017}                    \\ \hline
			13 & Defesa da dissertação                                                                 & \textit{01/2017}                   & \textit{01/2017}                    \\ \hline
		\end{tabular}%
	}
\end{table}

\bibliographystyle{acm}
\bibliography{relatorio}
\end{document}