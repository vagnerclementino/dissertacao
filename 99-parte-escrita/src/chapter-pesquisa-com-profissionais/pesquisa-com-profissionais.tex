%%%%%%%%%%%%%%%%%%%%%%%%%%%%%%%%%%%%%%%%%%%%%%%%%%%%%%%%%%%%%%%%%%%%%%%%%
%Objetivo: 
%Autor:
%Data Criação:
%Data Modificação:
%Data Revisão:
%%%%%%%%%%%%%%%%%%%%%%%%%%%%%%%%%%%%%%%%%%%%%%%%%%%%%%%%%%%%%%%%%%%%%%%%%%
\chapter{Pesquisa com Profissionais}
\label{ch:pesquisa-profissionais}

\section{Introdução}
\label{sec:pesquisa-profissionais-intro}

Questionário baseado em inquérito consiste em um método de pesquisa em que os
participantes responder a perguntas ou responder a declarações que foram
desenvolvidas com antecedência. Quando conduzido adequadamente, esse tipo de
pesquisa permite que os pesquisadores generalizem crenças e opiniões de uma
população relevante do público-alvo, estudando um subconjunto (amostra) deles.
Kasunic (2005) apresenta as seguintes etapas para a pesquisa baseada em
questionário
processo:
1. Identificar os objetivos da pesquisa
2. Identificar e caracterizar o público-alvo
3. Elaborar o plano de amostragem
4. Elaborar e escrever um questionário
5. Questionário de teste piloto
6. Distribua o questionário
7. Analise os resultados e escreva o relatório


Em uma série de artigos, Kitchenham e Pfleeger (2001) introduziram os princípios
de Pesquisa de pesquisa para pesquisadores de SE, cobrindo principalmente as
etapas 3, 4, 6 e 7 da pesquisa de Kasunic.  Proposta (2005). No contexto da
caracterização da população e da concepção do Descrevem-se alguns conceitos
básicos de estatística, com poucas discussões sobre as questões de população da
SE que são fornecidas (Kitchenham e Pfleeger Kitchenham e Pfleeger 2002).
Kitchenham e Pfleeger (2008) também investigaram o desenho de quatro pesquisas
SE realizadas entre 1998 e 2000, das quais os autores concluem que apenas uma
delas foi apoiada por uma amostra representativa de uma população claramente
estabelecida

Com o objetivo de coletar os aspectos mais importantes das FGRM's do ponto de
vista dos profissionais ligados à manutenção de software será realizada uma
pesquisa (survey). O planejamento e o desenho da pesquisa seguirá as diretrizes
propostas em\cite{wohlin2012experimentation}.

A população da pesquisa proposta é a comunidade envolvida com o processo de
manutenção de software e que faça uso de FGRM's. Neste contexto, seriam
possíveis amostras, os desenvolvedores envolvidos com tarefas de manutenção nos
projetos da Mozilla\footnote{\url{https://bugzilla.mozilla.org/}} ou da Eclipse
Foundation\footnote{\url{https://bugs.eclipse.org/bugs/}}.

A importância deste tipo de trabalho está na possibilidade de avaliar se as
pesquisas relativas a evolução das FGRM estão em consonância com as necessidades
dos profissionais envolvidos em manutenção de software, reduzindo, desta forma,
a distância entre o estado da arte e o estado da prática.  \todo[inline]{usar a
	expectativa de sermos aderentes às demandas do praticante}

\section{Objetivo da Pesquisa com Profissionais}
\label{sec:objetivo_da_pesquisa_com_profissionais}
Em linhas gerais, o objetivo desta etapa do estudo é analisar, através da
percepção e opinião dos profissionais envolvidos em manutenção de software, a
situação das funcionalidades atualmente oferecidas pelas FGRM, bem como a
relevância das extensões propostas na literatura. Estruturando melhor o
objetivo, conforme propõe a metodologia GQM (Goal, Question e Metric)\cite{gqm},
\textit{o propósito deste estudo avaliar as funcionalidade oferecidas e as
	extensões propostas nas literatura para as FGRM do ponto de vista dos
	profissionais envolvidos em manutenção de software no contexto de projetos
	de software de código aberto e uma empresa pública informática de médio
	porte.}  

Com intuito de atingir os objetivos propostos fora definidas as seguintes
questões de pesquisa:
\todo[inline]{Incluir uma descrição para cada questão de
	pesquisas}
\begin{description}
	\item[Questão 01] Qual o perfil dos profissionais envolvidos em Manutenção
		de Software?
	\item[Questão 02] Qual a opinião dos profissionais envolvidos em Manutenção
		de Software com relação as funcionalidades oferecidas pelas FGRM\@?
	\item [Questão 03] Na visão  dos profissionais envolvidos em Manutenção de Software quais das
		extensões propostas na literatura teriam maior relevância em suas atividades atuais?
\end{description}

As questões de pesquisas foram respondidas mediante a realização de uma pesquisa
baseada em questionário (survey). O desenho da pesquisa é detalhada na próxima
seção onde discutimos a estrutura do questionário bem como a amostra a população
e a sua respectiva amostra utilizada no estudo.

\section{Desenho e Metodologia da Pesquisa com Profissionais}
\label{sec:desenho_da_pesquisa_com_profissionais}

Esta Pesquisa com Profissionais (survey) consistiu de um estudo exploratório sem
uma hipótese prévia a ser avaliada. Realizamos um survey com um desenho
trans-seccional\cite{kitchenham2002principles} onde a nossa população de
interesse são os profissionais envolvidos em manutenção de software e estamos
especialmente interessado na experiência deles no uso das FGRM\@. 

A pesquisa foi realizada através um questionário auto-administrável disponível
pela Internet\footnote{\url{https://www.google.com/forms/about/}}. O formulário
foi enviado aos participantes mediante o e-mail previamente coletado.

Um Conceptual Framework proposto por de Mello et al. (2014d) que está
representado na Fig. 1. Para além dos conceitos estatísticos de público-alvo,
população, quadro de amostragem e unidade de observação (Thompson 2012), este
quadro introduz o seguinte conjunto de novos conceitos para melhor apoio à
amostragem nos inquéritos SE: fonte de amostragem, unidade de pesquisa, pesquisa
	Plano e estratégia de amostragem

Uma fonte de amostragem consiste em um banco de dados (automatizado ou não) no
qual subpopulações válidas do público-alvo podem ser sistematicamente
recuperadas e amostradas aleatoriamente. Assim, se uma fonte de amostragem pode
ser considerada válida para um contexto de pesquisa específico, pode-se concluir
que os quadros de amostragem podem ser estabelecidos a partir dele para o mesmo
contexto de pesquisa. Para ser considerada válida, uma fonte de amostragem deve
satisfazer, pelo menos, os seguintes requisitos essenciais (ER):

ER1. Uma fonte de amostragem não deve representar intencionalmente um
subconjunto segregado Do público-alvo, isto é, para um público-alvo "X", não é
adequado pesquisar Para unidades de uma fonte intencionalmente concebido para
compor um subconjunto específico de "X".
ER2. Uma fonte de amostragem não deve apresentar qualquer viés em incluir na sua
base de dados Preferencialmente apenas subconjuntos do público-alvo. Critérios
desiguais para Unidades de pesquisa significam oportunidades de amostragem
desiguais.
ER3. Todas as unidades de pesquisa das amostras e suas unidades de observação
devem ser Identificado por um id lógico ou numérico.
ER4. Todas as unidades de pesquisa de fontes de amostragem devem ser acessíveis.
Se houver Unidades de pesquisa, não é possível contextualizar a população.

Há também nove requisitos desejáveis (DR), três relacionados com as amostras '
(ADR), duas relacionadas com clareza (CDR) e quatro com relação à integridade da
amostra (CoDR). Estes critérios adicionais e exemplos de avaliar tais fontes
usando-os podem ser encontrados em (de Mello et al., 2014d)

A unidade de pesquisa caracteriza como uma ou mais unidades de observação podem
ser recuperadas de uma fonte específica de amostragem. Em um cenário ideal,
espera-se que tanto unidade de observação e unidade de pesquisa são tanto quanto
possível o mesmo. O plano de pesquisa descreve como as unidades de pesquisa
serão sistematicamente recuperadas de uma fonte de amostragem e avaliadas para
compor uma moldura de amostragem. Finalmente, a estratégia de amostragem
descreve as etapas que devem ser seguidas para amostragem e recrutamento de
indivíduos que participarão do estudo. Eventualmente, os dados usados para
apoiar o projeto de amostragem podem ser recuperados após coletar respostas com
um instrumento de medição, como um formulário de caracterização, antes de
executar
\subsection{Métodologia}
No caso desta pesquisa, o público-alvo é composto por profissionais de SE em
geral, uma vez que um conjunto de características e práticas identificadas como
"ágeis" na SE podem ser avaliados independentemente do processo de software
adotado. Assim, todos os profissionais que trabalham em projetos de software
podem potencialmente contribuir com esta investigação. É importante destacar que
o plano de levantamento pondera a relevância do participante e responde
correspondentemente pelo seu nível de experiência. As subseções a seguir
descrevem a estratégia de recrutamento projetada para este estudo de pesquisa

Fonte de amostragem, unidade de pesquisa e população

Uma rede social profissional (LinkedIn) foi estabelecida como fonte de
amostragem devido à sua cobertura, composta por mais de 10 milhões de
profissionais de TI espalhados pelo mundo (novembro de 2014). Para a realização
do plano de recrutamento e da análise de dados apresentados neste estudo, foi
necessária a utilização de uma conta "Premium". Esse tipo de conta permite que
os usuários do LinkedIn realizem análises mais precisas sobre a distribuição de
membros entre grupos de interesse. Desde LinkedIn permite realizar um grupo
abrangente de interesses procurando, "grupo de interesse" será a unidade de
pesquisa. De cada grupo identificado, serão extraídos os seguintes atributos:
Nome do Grupo, Descrição do Grupo, Tamanho do Grupo (quantidade de membros) e
Língua Oficial do Grupo. Esses atributos serão utilizados para verificar se cada
grupo de interesse pode ser incluído na moldura de amostragem, que deverá ser
composto por todos os grupos de interesse envolvidos com agilidade no processo
de software. Assim, a população deste estudo de pesquisa será composta por todos
os membros desses grupos selecionados.

Unidade de observação e unidade de análise

Nesta pesquisa, a unidade de observação ea unidade de análise são a mesma
entidade (indivíduo) e cada membro distinto de cada grupo é potencialmente
considerado uma unidade válida a ser amostrada. Os seguintes atributos devem ser
coletados de cada um:

Atributos coletados através da fonte de amostragem: ID do Membro, Nome, País e
Status de Associação em cada grupo de interesse na moldura de amostragem.
Pode-se ver que os perfis dos indivíduos no LinkedIn apresentam outros atributos
que podem ser utilizados em nossa investigação, tais como grau acadêmico,
experiência profissional e habilidades de topo. No entanto, esses dados
geralmente não são acessíveis quando o perfil individual não está diretamente
conectado com conta de usuário. Além disso, não há controle sobre se esses
atributos são atualizados.

Atributos coletados através do questionário de pesquisa (instrumento de
medição): País, Principais Habilidades em SE, SE Nível de Experiência, Nível de
Experiência de Agilidade e Grau Acadêmico.

Plano de pesquisa


A pergunta de busca para estabelecer a moldura de amostragem é: "Quais são os
grupos do LinkedIn relacionados à agilidade em processos de software?" Assim,
com base na seqüência de pesquisa da SLR e seus resultados (Abrantes e Travassos
2013), as seguintes expressões de busca foram Estabelecido:


Então, visando restringir a seleção de grupos de interesse para aqueles que
discutem ágil Práticas e características no contexto global, será excluído
qualquer grupo de Interesse que:
Proíbe expressamente a realização de estudos;
Restringe explicitamente a mensagem individual entre seus membros (um recurso padrão
Fornecido pelo LinkedIn);
Uma cidade, região ou país, uma vez que o nosso público-alvo não é
Geograficamente restrito;
Organizações específicas, ou fornecidas por eles, nem
Divulgar eventos específicos;
Tem a sua descrição fora do âmbito da Engenharia de Software;
Tem uma descrição vaga;
Tem um único membro;
É conduzido à headhunting e oferta de trabalho;
Representa os subgrupos do LinkedIn, uma vez que a moldura de amostragem deve
ser composta por Grupos de interesse, e; Tem uma língua não-Inglês como padrão,
uma vez que o idioma Inglês é padrão em fóruns internacionais

\subsection{Questionário}
\label{subsec:questionario}

O formulário enviado aos participantes foi estruturado em três parte, cada uma
com o objetivo de coletar um conjunto de informações. Na primeira parte estamos
interessados na formação de base (background) dos profissionais. O segundo
conjunto de perguntas visa obter a percepção dos participantes sobre as
funcionalidades atualmente oferecidas pelas FGRM\@. A terceira parte é do
formulário contêm as perguntas sobre a percepção dos participantes sobre as
extensões propostas na literatura. 

A fim de obtermos um formulário que conseguisse atingir os objetivos deste
estudo, realizamos um processo de avaliação em quatro etapas. O formulário
resultante de uma etapa anterior foi utilizado como entrada de uma etapa
posterior. Desta forma, utilizamos um processo iterativo para produzirmos o
formulário.
\begin{enumerate}[(i)]
	\item Avaliação por Pesquisadores: Nesta etapa o formulário inicialmente
		proposto foi enviado para dois pesquisadores experientes na área de
		manutenção de software.
	\item Avaliação por Profissionais O formulário resultante da análise
		anterior foi encaminhado a dois profissionais experientes envolvidos com
		manutenção de software. 
	\item Piloto da Pesquisa O formulário obtido após a fase anterior foi
		utilizado em um piloto com
		dez profissionais envolvidos da manutenção de software de uma empresa
		pública de
		informática~-~PRODABEL\footnote{\url{{http://www.prodabel.pbh.gov.br}}}
	\item Tradução do Formulário Em cada uma das etapas de anteriores o
		formulário foi aplicado em
		português, tendo em vista que alguns profissionais envolvidos no
		processo de avaliação não
		ter fluência em língua inglesa, em especial na fase ``Piloto da
		Pesquisa'. Neste sentido, a última etapa  consistiu na tradução do
		formulário para a língua inglesa.  Esta etapa foi conduzida com  o
		suporte de um pesquisador experiente na área de Engenharia de Software.	
\end{enumerate}
\todo[inline]{Acho que devemos tentar um grupo não PUBLICO PRODABEL e também um grupo PRODABEL (onde seja possível comentar sobre possíveis especificidades!!)}
\todo[inline]{Avaliar o impacto de ter um questionário em inglês e outro em português}

\subsection{População, Amostra e Respostas}
\label{subsec:populacao_amostra_respostas}

Estudos primários em Engenharia de Software (SE) são muitas vezes conduzidos em
amostras estabelecidas por conveniência (Pickard et al., 1998, Sjøberg et al.,
2005, Dybå et al., 2007). Este cenário é especialmente crítico para os
levantamentos em grande escala (Kasunic 2005), nos quais são aplicados esforços
consideráveis no recrutamento e recolha de dados dos participantes (Conradi et
al., 2005), mas a generalização dos resultados é limitada, mesmo quando as
características de outros inquéritos são claramente descritas e Repetidos em
seus ensaios.
Um desafio no estabelecimento de amostras representativas nos inquéritos SE
inclui a identificação de fontes relevantes e disponíveis a partir das quais
podem ser estabelecidas estruturas de amostragem. Portanto, os pesquisadores da
SE são tentados a explorar fontes alternativas tipicamente disponíveis na Web
para ampliar o tamanho das amostras, como as redes sociais (de Mello e
Travassos, 2013). No entanto, o uso ad hoc dessas tecnologias Web por si só não
é suficiente para evoluir o cenário de amostragem em relação aos levantamentos
SE, uma vez que o tamanho das amostras é apenas uma das questões que dificultam
a generalização dos resultados das pesquisas SE

(Kruskal e Mosteller, 1979) impulsionam nossa definição de "amostra
representativa": Método de amostragem específico (amostragem probabilística),
cobertura da heterogeneidade das populações e representativo como típico (com
relação a certas características conhecidas de a população)












\section{Resultados}
\label{sec:analise_dados}

Neste seção apresentamos os resultado obtidos da aplicação do questionário. Os
foram divididos pela questão de pesquisa ao qual visa responder. Por se tratar
de um estudo exploratório, no qual não foi proposta determinada tese a ser
provada, a análise dos resultados é feita mediante o uso de gráficos
representando a escala de Likert. Este tipo de grafo é recomendado para
visualizar dados na escala de Likert tendo em vista que possibilita o
entendimento da divergência entre as respostas dos
participantes~\cite{robbins2011plotting}

\section{Discussão}

\section{Ameças à Validade}


\subsubsection{Ameaças Internas}
\label{ssub:Ameacas_internas}

\subsubsection{Ameaças Externas}
\label{ssub:Ameacas_externas}


\section{Resumo do Capítulo}
\label{sec:resumo_do_capitulo}
