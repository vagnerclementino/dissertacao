
\chapter{Pesquisa com Profissionais}
\label{ch:pesquisa-profissionais}

\section{Introdução}
\label{sec:pesquisa-profissionais-intro}


Com o objetivo de coletar os aspectos mais importantes das FGRM's do ponto de
vista dos profissionais ligados à manutenção de software será realizada uma
 pesquisa (survey). O planejamento e o desenho da pesquisa seguirá as diretrizes propostas em \cite{wohlin2012experimentation}.

A população da pesquisa proposta é a comunidade envolvida com o processo de
manutenção de software e que faça uso de FGRM's. Neste contexto, seriam
possíveis amostras, os desenvolvedores envolvidos com tarefas de manutenção nos
projetos da Mozilla\footnote{\url{https://bugzilla.mozilla.org/}} ou da
Eclipse Foundation\footnote{\url{https://bugs.eclipse.org/bugs/}}. Durante a
execução da dissertação será avaliado qual amostra caracteriza melhor a
população do estudo.

A importância deste tipo de trabalho está na possibilidade de avaliar se as pesquisas relativas a
evolução das FGRM estão em consonância com as necessidades dos profissionais envolvidos em
manutenção de software, reduzindo, desta forma, a distância entre o estado da arte e o estado da
prática.
\section{Objetivo da Pesquisa com Profissionais}
\label{sec:objetivo_da_pesquisa_com_profissionais}

O objetivo é entender, através da  percepção e opinião dos profissionais envolvidos em manutenção de
software, 
\todo[inline]{Utilizar o template do GQM para definir o objetivo do survey}
\todo[inline]{Propor as questões de pesquisas que serão respondidas pelo survey}


\section{Desenho da Pesquisa com Profissionais}
\label{sec:desenho_da_pesquisa_com_profissionais}

Esta Pesquisa com Profissionais (survey) consistiu de um estudo exploratório sem uma hipótese prévia
a ser avaliada.


\subsection{Questionário}
\label{subsec:questionario}

\todo[inline]{Propor um processo de avaliação do questionário em três etapas: (i) avaliação por
	pesquisadores experientes - Rodolfo e mais um; (ii) avaliação por dois profissionais envolvidos
com manutenção de software; (iii) realização de um survey piloto com um pequeno grupo da PRODABEL}


\todo[inline]{Avaliar o impacto de ter um questionário em inglês e outro em português}

\subsection{População,Amostra e Respostas}
\label{subsec:populacao_amostra_respostas}


\todo[inline]{Avaliar a utilização de um ranqueamento para aplicar o questionário em desenvolvedores
de projetos de código aberto}

\section{Análise dos Dados}
\label{sec:analise_dados}

Neste seçaõ apresentamos os resultado obtidos da aplicação do questionário. Os foram dividos pela questão de pesquisa ao uqal visa responder. Por se trtara de um estudo explorário, no qual não foi proposta determinada tese a ser provada, a análise dos resultados é feita mediante o uso de gráficos representando a escala de Likert. Este tipo de grafo é recomendado para visualizar dados na escala de Likert tendo em vista que possibilita o entendimento da divergência entre as respostas dos participantes \cite{robbins2011plotting}

\section{Discussão}

\section{Ameças à Validade}


\subsubsection{Ameaças Internas}
\label{ssub:Ameaças Internas}

\subsubsection{Ameaças Externas}
\label{ssub:Ameaças Externas}


\section{Resumo do Capítulo}
