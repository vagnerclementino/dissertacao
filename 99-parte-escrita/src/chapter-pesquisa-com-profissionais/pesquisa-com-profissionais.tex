%%%%%%%%%%%%%%%%%%%%%%%%%%%%%%%%%%%%%%%%%%%%%%%%%%%%%%%%%%%%%%%%%%%%%%%%%%%%%%%%%%%%%%%%%%%%%%%%%%%%%%%%%%%%
%Objetivo:
%Autor:
%Data Criação:
%Data Modificação:
%Data Revisão:
%%%%%%%%%%%%%%%%%%%%%%%%%%%%%%%%%%%%%%%%%%%%%%%%%%%%%%%%%%%%%%%%%%%%%%%%%%%%%%%%%%%%%%%%%%%%%%%%%%%%%%%%%%%%
\chapter{Pesquisa com Profissionais}
\label{ch:pesquisa-profissionais}

\section{Introdução}
\label{sec:pesquisa-profissionais-intro}


Com o objetivo de coletar os aspectos mais importantes das FGRM's do ponto de
vista dos profissionais ligados à manutenção de software será realizada uma
 pesquisa (survey). O planejamento e o desenho da pesquisa seguirá as diretrizes propostas em\cite{wohlin2012experimentation}.

A população da pesquisa proposta é a comunidade envolvida com o processo de
manutenção de software e que faça uso de FGRM's. Neste contexto, seriam
possíveis amostras, os desenvolvedores envolvidos com tarefas de manutenção nos
projetos da Mozilla\footnote{\url{https://bugzilla.mozilla.org/}} ou da
Eclipse Foundation\footnote{\url{https://bugs.eclipse.org/bugs/}}.

A importância deste tipo de trabalho está na possibilidade de avaliar se as pesquisas relativas a
evolução das FGRM estão em consonância com as necessidades dos profissionais envolvidos em
manutenção de software, reduzindo, desta forma, a distância entre o estado da arte e o estado da
prática.  \todo[inline]{usar a expectativa de sermos aderentes às demandas do praticante}

\section{Objetivo da Pesquisa com Profissionais}
\label{sec:objetivo_da_pesquisa_com_profissionais}
Em linhas gerais, o objetivo desta etapa do estudo é analisar, através da percepção e opinião dos
profissionais envolvidos em manutenção de software, a situação das funcionalidades atualmente
oferecidas pelas FGRM, bem como a relevância das extensões propostas na literatura. Estruturando
melhor o objetivo, conforme propõe a metodologia GQM (Goal, Question e Metric)\cite{gqm}, \textit{o propósito deste estudo avaliar as funcionalidade oferecidas e as extensões propostas nas literatura para as FGRM do ponto de vista dos profissionais envolvidos em manutenção de software no contexto de projetos de software de código aberto e uma empresa pública informática de médio porte.}  

Com intuito de atingir os objetivos propostos fora definidas as seguintes questões de pesquisa:
\todo[inline]{Incluir uma descrição para cada questão de pesquisas}
\begin{description}
	\item[Questão 01] Qual o perfil dos profissionais envolvidos em Manutenção de Software?
	\item[Questão 02] Qual a opinião dos profissionais envolvidos em Manutenção de Software com
		relação as funcionalidades oferecidas pelas FGRM\@?
	\item [Questão 03] Na visão  dos profissionais envolvidos em Manutenção de Software quais das
		extensões propostas na literatura teriam maior relevância em suas atividades atuais?
\end{description}

As questões de pesquisas foram respondidas mediante a realização de uma pesquisa baseada em
questionário (survey). O desenho da pesquisa é detalhada na próxima seção onde discutimos a estrutura
do questionário bem como a amostra a população e a sua respectiva amostra utilizada no estudo.

\section{Desenho da Pesquisa com Profissionais}
\label{sec:desenho_da_pesquisa_com_profissionais}

Esta Pesquisa com Profissionais (survey) consistiu de um estudo exploratório sem uma hipótese prévia
a ser avaliada. Realizamos um survey com um desenho trans-seccional\cite{kitchenham2002principles} onde a nossa população de
interesse são os profissionais envolvidos em manutenção de software e estamos especialmente
interessado na experiência deles no uso das FGRM\@. 

A pesquisa foi realizada através um questionário auto-administrável disponível pela
Internet\footnote{\url{https://www.google.com/forms/about/}}. O formulário foi enviado aos
participantes mediante o e-mail previamente coletado.


\subsection{Questionário}
\label{subsec:questionario}

O formulário enviado aos participantes foi estruturado em três parte, cada uma com o objetivo de
coletar um conjunto de informações. Na primeira parte estamos interessados na formação de base
(background) dos profissionais. O segundo conjunto de perguntas visa obter a percepção dos
participantes sobre as funcionalidades atualmente oferecidas pelas FGRM\@. A terceira parte é do
formulário contêm as perguntas sobre a percepção dos participantes sobre as extensões propostas na
literatura. 

A fim de obtermos um formulário que conseguisse atingir os objetivos deste estudo, realizamos um
processo de avaliação em quatro etapas. O formulário resultante de uma etapa anterior foi utilizado como entrada de uma etapa posterior. Desta forma, utilizamos um processo iterativo para produzirmos o formulário.
\begin{enumerate}[(i)]
	\item Avaliação por Pesquisadores: Nesta etapa o formulário inicialmente proposto foi enviado para dois pesquisadores experientes na área de manutenção de software.
	\item Avaliação por Profissionais O formulário resultante da análise anterior foi encaminhado a dois profissionais experientes envolvidos com manutenção de software. 
	\item Piloto da Pesquisa O formulário obtido após a fase anterior foi utilizado em um piloto com
		dez profissionais envolvidos da manutenção de software de uma empresa pública de informática~-~PRODABEL\footnote{\url{{http://www.prodabel.pbh.gov.br}}}
	\item Tradução do Formulário Em cada uma das etapas de anteriores o formulário foi aplicado em
		português, tendo em vista que alguns profissionais envolvidos no processo de avaliação não
		ter fluência em língua inglesa, em especial na fase ``Piloto da Pesquisa'. Neste sentido, a última etapa  consistiu na tradução do formulário para a língua inglesa.  Esta etapa foi conduzida com  o suporte de um pesquisador experiente na área de Engenharia de Software.	
\end{enumerate}
\todo[inline]{Acho que devemos tentar um grupo não PUBLICO prodabel e também um grupo PRODABEL (onde seja possível comentar sobre possíveis especificidades!!)}
\todo[inline]{Avaliar o impacto de ter um questionário em inglês e outro em português}

\subsection{População, Amostra e Respostas}
\label{subsec:populacao_amostra_respostas}

\todo[inline]{Avaliar a utilização de um ranqueamento para aplicar o questionário em desenvolvedores
de projetos de código aberto}

\section{Análise dos Dados}
\label{sec:analise_dados}

Neste seção apresentamos os resultado obtidos da aplicação do questionário. Os foram divididos pela
questão de pesquisa ao qual visa responder. Por se tratar de um estudo exploratório, no qual não foi
proposta determinada tese a ser provada, a análise dos resultados é feita mediante o uso de gráficos
representando a escala de Likert. Este tipo de grafo é recomendado para visualizar dados na escala
de Likert tendo em vista que possibilita o entendimento da divergência entre as respostas dos
participantes~\cite{robbins2011plotting}

\section{Discussão}

\section{Ameças à Validade}


\subsubsection{Ameaças Internas}
\label{ssub:Ameacas_internas}

\subsubsection{Ameaças Externas}
\label{ssub:Ameacas_externas}


\section{Resumo do Capítulo}
\label{sec:resumo_do_capitulo}
