%%%%%%%%%%%%%%%%%%%%%%%%%%%%%%%%%%%%%%%%%%%%%%%%%%%%%%%%%%%%%%%%%%%%%%%%%%%%%%%%%%%%%%%%%%%%%%%%%%%%%%%%%%%%
%Objetivo: Introduzir os conceitos envolvidos na dissertação bem como do trabalho realizado.
%		   A ideia é que qualquer pessoa que leia a introdução consiga ter uma visão geral
%		   sobre a dissertação.
%Autor: Vagner Clementino <vagnercs@dcc.ufmg.br> e Rodolfo Resende <rodolfo@dcc.ufmg.br>
%Data Criação: Dom Set 18 22:55:43 BRT 2016
%Data Modificação: Dom Set 18 22:55:57 BRT 2016
%Data Revisão: Dom Set 18 22:56:08 BRT 2016
%%%%%%%%%%%%%%%%%%%%%%%%%%%%%%%%%%%%%%%%%%%%%%%%%%%%%%%%%%%%%%%%%%%%%%%%%%%%%%%%%%%%%%%%%%%%%%%%%%%%%%%%%%%%
\chapter{Introdução}
\label{ch:intro}
\todo[inline]{O objetivo de seção é introduzir ao leitor na disciplina de Engenharia de Software, em
especial quanto aos tipos de manutenção descritos na literatura e a utilização de uma ferramenta
para o seu gerenciamento}

Dentro do ciclo de vida de um produto de software o processo de manutenção tem
papel fundamental. Devido ao seu alto custo, em alguns casos chegando a 60\%
do preço final~\cite{kaur2015review}, as atividades relacionadas a manter e evoluir software tem sua
importância considerada tanto pela comunidade científica quanto pela indústria.

Desde o final da década de 1970~\cite{Zelkowitz:1979:PSE:578504} percebe-se o aumento do custo
referente as atividades de  manutenção de software. Nas décadas de 1980 e 1990 alguns trabalhos
tiveram seu foco no desenvolvimento de modelos de mensuração do custo para manter o
software~\cite{Herrin:1985:SMC:323287.323383,hirota1994approach}. Apesar da evolução das metologias
de manutenção a estimativa é que nas últimas duas décadas o custo de manutenção tenha aumentado em
50\%~\cite{koskinen2010software}. Esta tendência pode ser observada na
Figura~\ref{fig:software-maintence-costs} no qual é possível verificar a evolução do custo da
manutenção de software como fração do custo total do produto.

\begin{figure}
\centering
\includegraphics[width=0.7\linewidth]{./chapter-intro/img/software-maintence-costs}
\caption{Evolução da manutenção de software como percentual do custo total.	Extraído de~\cite{engelbertink2010save}}
\label{fig:software-maintence-costs}
\end{figure}

Uma vez que o software entra em operação, anomalias são descobertas, mudanças ocorrem no ambiente
de operação e novos requisitos são solicitados pelo usuário. Todas estas demandas devem ser
solucionadas na fase de Manutenção que inicia efetivamente com entrega do sistema, contudo, certas
atividades ocorrem antes.

A \textit{Manutenção}, dentre outros aspectos, corresponde ao processo de modificar um componente ou
sistema de software após a sua entrega com o objetivo de \textit{corrigir falhas, melhorar o
	desempenho ou adaptá-lo devido à mudanças ambientais}~\cite{{159342}}. 
De maneira relacionada, \textit{Manutenibilidade} é a propriedade de um sistema ou componente de software em relação ao grau
de \textit{facilidade} que ele pode ser corrigido, melhorado ou adaptado~\cite{{159342}}.

Verificamos na literatura certa discussão sobre a diferença entre manutenção e evolução de software.
Percebe-se ainda que pesquisadores e profissionais utilizam evolução como substituto preferido para
manutenção~\cite{Bennett:2000:SME:336512.336534}. Todavia, não está no escopo desta dissertação
discutir e apresentar as diferenças entre os conceitos. Neste sentido, utilizamos os termos \textit{manter} e
\textit{evoluir} software de forma intercambiáveis.

As manutenções em software podem ser divididas em \textit{Corretiva, Adaptativa, Perfectiva e
	Preventiva}~\cite{Lientz:1980:SMM:601062,159342}. A ISO 14764 discute
os quatro tipos de manutenções, conforme já descrito, e além disso propõe que exista um elemento comum
denominado \textit{Requisição de Mudança} que representa as características comuns a todas aqueles
tipos de manutenção.

Por conta do volume das Requisições de Mudança se faz necessária a utilização de
ferramentas com o objetivo de gerenciá-las. Esse controle é geralmente
realizado por \textit{Ferramentas de Gerenciamento de Requisição de Mudança -~FGRM}, que auxiliam os
desenvolvedores na correção de forma individual ou colaborativa de defeitos (bugs), no
desenvolvimento de novas funcionalidades, dentre outras tarefas relativas à manutenção de software.
A literatura não define uma nomenclatura comum para este tipo de ferramenta. Em alguns estudos é
possível verificar nomes tais como Sistema de Controle de Defeito -~Bug Tracking Systems, Sistema de
Gerenciamento da Requisição -~Request Management System, Sistemas de Controle de Demandas (SCD)-
Issue Tracking Systems. Todavia, de modo geral, o termo se refere as
ferramentas utilizadas pelas organizações para \textit{gerir as Requisições de Mudança}. Estas
ferramentas podem ainda ser utilizadas por gestores, analistas de qualidade e usuários finais para
atividades tais como gerenciamento de projetos, comunicação, discussão e revisões de código. Neste
trabalho utilizaremos o termo \texttt{Ferramentas de Gerenciamento de Requisições de Mudança} (FGRM)
ao referimos a este tipo de ferramenta.  A Tabela~\ref{tab:exemplo} apresenta alguns exemplos de
software que podem ser classificadas como FGRM's. Também são listados serviços da Internet que
oferecem funcionalidades presentes nas FGRM na forma de Software como
Serviço~\cite{fox2013engineering}.

\begin{table}[ht]
	\centering
	\resizebox{\textwidth}{!}{%
		\begin{tabular}{llll}
			\hline
			\multicolumn{2}{c}{\textbf{Ferramentas}}           & \multicolumn{2}{c}{\textbf{Serviços da Internet}} \\ \hline
			Bugzilla & https://www.bugzilla.org/               & SourceForge    & https://sourceforge.net/    \\ \hline
			MantisBT & https://www.mantisbt.org/               & Lauchpad       & https://launchpad.net/      \\ \hline
			Trac     & https://trac.edgewall.org/              & Code Plex      & https://www.codeplex.com/   \\ \hline
			Redmine  & www.redmine.org/                        & Google Code    & https://code.google.com/    \\ \hline
			Jira     & https://www.atlassian.com/software/jira & GitHub         & https://github.com/         \\ \hline
		\end{tabular}%
	}
	\caption{Exemplos de ferramentas e serviços da Internet. Adaptado de~\cite{cavalcanti2014challenges}}\label{tab:exemplo}
\end{table}

\section{Motivação}
\label{sec:intro-motivacao}


\todo[inline]{O objetivo desta seção é apresentar a motivação do estudo. Em síntese, tenta responder
a seguinte pergunta: Por que dentro do contexto da manutenção de software estudar as Ferramentas de
Gerenciamento de Requisição de Mudança é IMPORTANTE?}

Diante da maior presença de software em todos os setores da sociedade
existe um interesse por parte da academia e da industria no desenvolvimento de
processos, técnicas e \textit{ferramentas} que reduzam o esforço e o custo das tarefas
de desenvolvimento e manutenção de software. Neste linha, o trabalho de Yong \&
Mookerjee~\cite{1423995}  propõe um modelo que reduz os custos de manutenção e reposição durante a
vida útil de um sistema de software. O modelo demonstrou que em algumas situações é \textit{melhor
	substituir um sistema do que mantê-lo}. Este problema é agravado tendo em vista que o custo de
manutenção que alguns necessita que 60\% dos desenvolvedores dedicados à tarefas de manutenção de sistemas~\cite{Zhang_2003}.

Diversos projetos de software, especialmente durante as etapas de desenvolvimento e teste do
software, necessitam de uma ferramenta para gerenciar as suas Requisições de Mudança tendo em vista
o seu volume bem como pela grande quantidade de pessoas inserir dados sobre os erros
encontrados~\cite{1407819}, por exemplo. Este tipo de ferramenta vêm sendo utilizados em projetos de
código aberto (Apache, Linux, Open Office) bem como em organizações públicas e privadas
(NASA,IBM).

Não obstante, alguns estudos demonstram que as FGRM's desempenham um papel além daquele de gerenciar
os pedidos de manutenção e evolução do software. Avaliando o controle de demandas como um processo
social, Bertram et al.~\cite{Bertram:2010:CCB:1718918.1718972} realizaram um estudo qualitativo em
FGRM's quando utilizados por pequenas equipes de desenvolvimento de software. Os resultados
mostraram que este tipo ferramenta não é apenas um banco de dados de rastreamento de defeitos,
recursos ou pedidos de informação, mas também atua como um ponto focal para a comunicação e
coordenação para diversas partes interessadas (stakeholders) dentro e fora da equipe de software. Os
clientes, gerentes de projeto, o pessoal envolvido com a garantia da qualidade e programadores,
contribuem em conjunto para o conhecimento compartilhado dentro do contexto das FGRM's.
 
No trabalho de Breu et al.\cite{Breu:2010:INB:1718918.1718973} o foco é analisar o papel dos FGRM's
no suporte à colaboração entre desenvolvedores e usuários de um software. A partir da análise
quantitativa e qualitativa de defeitos registrados em uma FGRM de dois projetos de software livre
foi possível verificar que o uso da ferramenta propiciou que os usuários desempenhassem um papel
além de simplesmente reportar uma falha: a participação ativa e permanente dos usuários finais foi
importante no progresso da resolução das falhas que eles descreveram.


Um outro importante benefício da utilização das FGRM é que as mudanças no software podem ser
rapidamente identificada e reportada para os desenvolvedores~\cite{anvik2005coping}. Além disso, eles podem ajudar a estimar
o custo do software, na análise de impacto, planejamento, rastreabilidade, descoberta do
conhecimento~\cite{cavalcanti2013bug}.

Não obstante, no contexto de utilização desta ferramentas diversos desafios se apresentam:
duplicação RM's, pedidos de modificação que são abertos inadvertidamente, grande quantidade de RM's
que devem ser atribuídas as desenvolvedores, bugs descrito de forma incompleta, análise de
impacto das RM's e RM's atribuídas de maneira incorreta~\cite{cavalcanti2014challenges}.  Diante de
tantos problemas e desafios é importante entender como aquele tipo de ferramente vêm sendo utilizada
bem como analisar o que está sendo proposta na literatura com objetivo de melhor as funcionalidades
oferecidas pelas FGRM.
%
%No trabalho de Junio et al.~\cite{5741246} é proposto um processo denominado
%PASM (Process for Arranging Software Maintenance Requests) que propõe lidar com
%tarefas de manutenção como projetos de software. Para tanto, utilizou-se
%técnicas de análise de agrupamento (clustering) a fim de melhor compreender e
%comparar as demandas de manutenção. Os resultados demostraram que depois de
%adotar o PASM os desenvolvedores tem dedicado um tempo maior para análise e
%validação. De outra forma, relacionada um menor tempo foi dedicado às tarefas
%de execução e codificação.
%
%No estudo realizado por Bettenburg et al.~\cite{bettenburg2008makes} foi
%desenvolvida uma pesquisa (\textit{survey}) entre desenvolvedores e usuários
%dos projetos Apache\footnote{\url{http://www.apache.org/}},
%Eclipse\footnote{\url{https://www.eclipse.org}} e
%Mozilla\footnote{\url{https://www.mozilla.org}} a fim de verificar o que
%produziria uma boa FGRM\@. Os resultados demonstraram que do ponto de vista dos
%desenvolvedores eram consideradas úteis funcionalidades tais como reprodução do
%erro, rastros de pilhas (stack traces) e casos de testes. A partir deste
%resultado foi construído um protótipo capaz de conduzir os usuários na coleta e
%fornecimento de um maior número de informações úteis para a resolução do
%defeito reportado.
%
%
%Em Zimmermann et al.~\cite{5070993} é discutido a importância de que a
%informação descrita em uma Requisição de Mudança seja relevante e completa a
%fim de que o defeito reportado seja resolvido rapidamente. Contudo, na prática,
%a informação apenas chega ao desenvolvedor com a qualidade requerida após
%diversas interações com o usuário afetado. Com o objetivo de minimizar este
%problema os autores propõe um conjunto de diretrizes para a construção de um
%ferramenta capaz de reunir informações relevantes a partir do usuário e
%identificar arquivos que precisam ser corrigidos para resolver o defeito.
%

%No trabalho de Kononenko et al.~\cite{Kononenko:2014:DED:2591062.2591075} é
%apresentada uma ferramenta denominada \textit{DASH} cujo objetivo é agrupar as
%demandas que são relevantes para as atividades de um desenvolvedor.
%Naturalmente todas as demandas ditas relevantes deveriam estar sob a
%responsabilidade de um mesmo programador. O principal objetivo desta ferramenta
%é aumentar a Consciência Situacional (Situational Awareness) dos
%desenvolvedores. Segundo os autores, o principal ganho do uso da ferramenta é
%que os programadores podem gerenciar melhor o excesso de informação e ficar
%mais ciente da evolução das demais demandas do sistema.
%
%Na ferramenta proposta por Thung et al.~\cite{Thung:2014:DIT:2642937.2648627} o
%foco é na determinação de defeitos duplicados. A contribuição deste trabalho é
%a integração do estado da arte de técnicas não supervisionadas para detecção de
%falhas duplicadas conforme proposto por Runeson et
%al.~\cite{Runeson:2007:DDD:1248820.1248882}. A ferramenta utiliza o Modelo de
%Vetor Espacial (Vetor Space Model) como métrica de similaridade entre os
%defeitos e fornece aos desenvolvedores uma lista de possíveis duplicatas.


%A manutenção não necessariamente exige que o processo de software envolvido
%seja o tradicional. Percebe-se alguns exemplos de adoção das práticas ágeis
%para fins de manutenção e evolução do software~\cite{kajko2009model, Heeager2015, Devulapally2015,Naz2016}. Tal
%tendência não é surpreendente tendo em vista que os métodos ``ágeis'' enfatizam
%características úteis à eficiência da implementação de software, tais como desenvolvimento incremental e teste contínuo que agregam valor para a evolução e manutenção eficaz de um sistema
%\cite{thomas2006agile}. Dentro desta tendência verifica-se a necessidade de que as ferramentas envolvidas no suporte à manutenção de software se adéquem à este nova forma de manter software. 
%


\section{Análise do Problema}
\label{sec:intro-problema}

\todo[inline]{OBJETIVO: Apresentar o problema que esta dissertação pretende resolver. O problema
	deverá ser definido claramente ou deverão ser apresentadas provas da importância do mesmo dentro
do escopo da Engenharia de Software}

O desenvolvimento e a manutenção de software envolvem diversos tipos de métodos, técnicas e
ferramentas. Em especial no processo de manutenção, um importante aspecto são as diversas
Requisições de Mudanças que devem ser gerenciadas. Este controle é realizado pelas FGRM's cujo o uso
vem crescendo em importância, sobretudo, por sua utilização por gestores, analistas da qualidade e
usuários finais para atividades como tomada de decisão e comunicação. How-
ever, most bug tracking systems are far from perfect. Many
of them are merely better interfaces to a database that stores
all reported bugs

Many
of them are merely better interfaces to a database that stores
all reported bugs. As a result, they often ask too much from
end-users who are not familiar with development practices.
At the same time they cause frustration for developers who
are disappointed about the quality of bug reports submitted
by users.

Apesar da inegável importância das FGRM's, percebe-se um aparente desacoplamento deste tipo de
ferramenta com as necessidades das diversas partes interessadas (stakeholders) na manutenção e
evolução de software. A utilização de  \textit{``demanda''} como conceito central para Ferramentas de Gerenciamento de
Requisição de Mudanças (FGRM) parece ser distante das necessidades práticas dos projetos de
software, especialmente no ponto de vista dos desenvolvedores
\cite{Baysal:2013:SAP:2486788.2486957}.

Um exemplo deste desacoplamento deste tipo de ferramenta com a necessidade de seus usuários pode ser visto no
trabalho proposto por Baysal \& Holme \cite{baysal2012qualitative} no qual desenvolvedores que
utilizam o Bugzilla\footnote{\url{https://www.bugzilla.org}} relatam a dificuldade em manter uma
compreensão global das RM's em que eles estão envolvidos. Segundo os desenvolvedores seria
interessante que a ferramenta tivesse um suporte melhorado para a Consciência Situacional
-~Situational Awareness. Em síntese, eles gostariam de estar cientes da situação global do projeto
bem como das atividades que outras pessoas estão realizando.

Com o objetivo de melhorar as FGRM (issue tracking system), mediante a melhoria daquilo que é
reportado aos dessenvolvedores, Zimmermann e outros discute quatro dimensões de melhorias deste tipo
de ferramenta, conforme esquematizado na Figura~\ref{fig:dimensoes_melhorias_fgrm}:

\begin{description}
	\item[Informação] Estas melhorias foram diretamente a informação sendo fornecida pelo reportador
					  da RM. Com ajuda da FGRM o responsável por relatar o bug, por exemplo, poderia ser motivado a
					  coletar mais informação sobre o problema. O sistema poderia verificar a validade e consistência da
					  informação repassada pelo usuário.  
	\item[Processo]	  Melhorias com foco no processo visam dar suporte à adminstração de
					  atividades relacionadas à solução de RM. Por exemplo, a triagem de RM, poderia ser automatizada
					  visando acelar o processo. Como outros exemplos seriam uma melhor consciência do progresso realizado
					  em cada RM ou mesmo fornecer ao usuário afetado uma estimativa de em quanto tempo a sua requisição
					  será solucionada.
	\item[Usuário]    Nesta dimensão estão incluídos tanto os usuário que relatam as RM's e os
				      desenvolvedores. Os reportadores podem ser educados de qual informação fornecer e como
				      coletá-la. Os desenvolvedores também podem beneficiar de um treinamento similar em qual informação
					  esperar e como esta informação pode ser utilizada para solucionar a RM.
	\item[Ferramenta] As melhorias centradas na ferramenta são realizadas nas funcionalidades
					  fornecidas pelas FGRM. Elas podem reduzir a complexidade da coleta e fornecimento das
					  informações necessárias para solucionar o RM. Por exemplo, as FGRM poderiam ser configuradas
					  para automatica localizar a pilha de erro (stack trace) e adicioná-la ao erro reportado. A
					  ferramenta poderia simplificar o processo de reprodução do erro mediante a simplicação do
					  processo de capturas de telas. Estes exemploes visam ajudar com a coleta das informações
					  necessárias pelos desenvolvedores para corrigir o bug, por exemplo. 
\end{description}

\begin{figure}[htpb]
	\centering
	\includegraphics[width=0.666666\linewidth]{chapter-intro/img/dimensoes_melhorias_fgrm.pdf}
	\caption{Dimensões de melhoria das FGRM's. Adaptado de~\cite{zimmermann2005mining}}
	\label{fig:dimensoes_melhorias_fgrm}
\end{figure}

%We believe that one reason for this
%problem is that current bug tracking systems are merely in-
%terfaces to relational databases that store the reported bugs.
%They provide little or no support to reporters to help them
%provide the information that developers need.


Neste estudo estamos especialmente interessandos em analisar e propor as melhorias relátivas ao
domínio da Ferramenta. Ao bem do nosso conhecimento é relativamento pequeno o número de trabalhos
que avaliem de forma sistemática as funcionalidades oferecidas pelas FGRM e que faça relação com os
estudos propostos na literatura. Além disso, os estudores anteriormente propostos não discute que da
mesma forma que ocorre no desenvolvimento de software, é possível verificar uma crescente adoção de
técnicas da metodologia ágil na manutenção de software~\cite{Soltan2016,Devulapally2015,
	Heeager2015}. Neste contexto, é natural que ferramentas que dão suporte à manutenção, tal como
as FGRM's, tenham que evoluir para se adaptar a esta nova forma de trabalhar. Mesmo em um ambiente
tradicional de  desenvolvimento e manutenção de software, verifica-se a necessidade de adequação das
FGRM's. Um outro sinal da necessidade de evolução deste tipo de ferramenta pode ser observado
considerando as diversas extensões (plugins) propostas na literatura
\cite{101186,Thung:2014:BIT:2635868.2661678,Kononenko:2014:DED:2591062.2591075}.

\section{Visão Geral do Estudo}
\label{sec:intro-visao-geral}

\todo[inline]{OBJETIVO: Apresentar de forma sucinta o objetivo da dissertação}

In order to achieve the goal of this work, stated in the previous section, a context-aware
architecture towards automating CR assignment is proposed. The construction of the architecture is based on two empirical studies: the systematic mapping study presented in Chapter 3, and the surveys with practitioners, presented in Chapter 4. Through the survey, we identified a set of functional and non-functional requirements
that should be satisfied when assigning CRs to developers, and investigated if these requirements are implemented in current approaches identified in the mapping study. We observed only a small subset of the requirements that is implemented in these approaches. However, in order to achieve a better performance and leverage the approaches’ applicability, it is desirable to consider all these requirements together. It, in turn, requires the integration and coordination of information from different data sources. Such required information varies from project to project, as well as from one organization
to another, characterizing the need for a context-aware approach. In this sense, our context-aware architecture is implemented by integrating current approaches found in the literature, which rely on IR models, and rule-based expert systems. The IR models deal with recommendation based on textual similarities and the expert systems reason on context information. Nevertheless, it is important to note that, although we are using the word automated to
characterized our approach, it is still needed to perform manual configurations before and during the approach’s execution. Thus, when we say automated we are referring to the assignments that the approach is able to perform autonomously after it have been properly configured

Neste sentido, este trabalho de dissertação investiga e contribui no entendimento de
como as Ferramentas de Gerenciamento de Requisição de Mudança estão sendo melhoradas ou estendidas
no contexto da transformação do processo de desenvolvimento e manutenção de software de um modelo
tradicional para outro que incorpora cada vez mais as práticas propostas pelos agilistas. O intuito
foi analisar como as FGRM estão sendo modificadas com base na literatura da área em contraste com o
ponto de vista dos profissionais envolvidos em manutenção de software.

Neste contexto, elaboramos um estudo das Ferramentas de Gerenciamento de Requisição de Mudança (FGRM) com o objetivo:
\begin{enumerate}[(i)]
	\item entender os requisitos comuns deste tipo de ferramenta;
	\item mapear as extensões para as FGRM que estão sendo propostas na literatura;
	\item avaliar sobre o ponto de vista dos profissionais a situação atual dos FGRM\@;
	\item propor novas extensões para as FGRM\@.  
\end{enumerate}

Vamos discutir os aspectos que são considerados mais importantes a partir da literatura da área, bem
como do ponto de vista de profissionais envolvidos em manutenção de software. De forma particular,
iremos estudar os mecanismos de personalização que algumas destas ferramentas permitem e tentaremos
ainda criar exemplos de personalização para alguma possível extensão a ser identificada ao longo do
trabalho.

\section{Metodologia de Pesquisa}
\label{sec:intro-metodologia}
\todo[inline]{OBJETIVO: Em linhas gerais apresenta como o problema descrito na seção anterior foi
	resolvido mediante a apresentação da metodologia científica utilizada}


This research design of this thesis is based on a multimethod approach (HESSE-BIBER,
2010). Such approach combines two or more quantitative (or qualitative) methods in a single study,
such as a survey and an experiment (HESSE-BIBER, 2010). Multimethod must not be confused with mixed
method. In this last, methods for both qualitative and quantitative types of research are applied in
a single study. On the other hand, multimethod studies combine different methods for a single
research type

When applying a multimethod approach, the triangulation is used to consolidate the
results from the different methods, considering, however, that the same research question(s)
was/were investigated in these methods. As a consequence, the triangulation of methods en- hances
the conclusions and completeness of the study, bringing more credibility to the research findings
(HESSE-BIBER, 2010). Figure 1.2 shows the multimethod research design applied in this thesis. The

The design started by stating the research objective, which we defined in Section 1.2, and
performing the initial literature review. This last provided the basic concepts and understanding of
the area. Then, a systematic mapping study and a questionnaire-based survey were conducted. These
two gathered detailed information on our research topic. Indeed, both of them were used to
understand the key aspects of CR assignment and identify the set of requirements to automate the
assignments. In the evidence synthesis step, these results were detailed and organized in order to
formulate the approach to automate CR assignments, which was constructed in the next step. Finally,
the research design states the validation of the proposed approach.

O trabalho de dissertação pode ser dividido nas etapas listadas a seguir:

\begin{itemize}[(i)]
	\item Mapeamento Sistemático da Literatura~\cite{keele2007guidelines}
	\item Caracterização das Ferramentas de Gerenciamento de Requisição de Mudança (FGRM)
	\item Pesquisa (Survey) com os desenvolvedores~\cite{wohlin2012experimentation}
	\item Desenvolvimento de extensões para as FGRM's
\end{itemize}

\section{Contribuições da Dissertação}
\label{sec:intro-contribuicao}

\section{Organização da Dissertação}
\label{sec:intro-organizacao-dissertacao}
