%%%%%%%%%%%%%%%%%%%%%%%%%%%%%%%%%%%%%%%%%%%%%%%%%%%%%%%%%%%%%%%%%%%%%%%%%%%%%%%%%%%%%%%%%%%%%%%%%%%%%%%%%%%%
%Objetivo:
%Autor: Vagner Clementino <vagnercs@dcc.ufmg.br> e Rodolfo Resende <rodolfo@dcc.ufmg.br>
%Data Criação: Dom Set 18 22:55:43 BRT 2016
%Data Modificação: Dom Set 18 22:55:57 BRT 2016
%Data Revisão: Dom Set 18 22:56:08 BRT 2016
%%%%%%%%%%%%%%%%%%%%%%%%%%%%%%%%%%%%%%%%%%%%%%%%%%%%%%%%%%%%%%%%%%%%%%%%%%%%%%%%%%%%%%%%%%%%%%%%%%%%%%%%%%%%
\chapter{Introdução}
\label{ch:intro}

Dentro do ciclo de vida de um produto de software o processo de manutenção tem
papel fundamental. Devido ao seu alto custo, em alguns casos chegando a 60\%
do preço final~\cite{kaur2015review}, as atividades relacionadas a manter e evoluir software tem sua
importância considerada tanto pela comunidade científica quanto pela indústria.

Uma vez que o software entra em operação, anomalias são descobertas, mudanças ocorrem no ambiente
de operação e novos requisitos são solicitados pelo usuário. Todas estas demandas devem ser
solucionadas na fase de Manutenção que inicia efetivamente com entrega do sistema, contudo, certas
atividades ocorrem antes.

A \textit{Manutenção}, dentre outros aspectos, corresponde ao processo de modificar um componente ou
sistema de software após a sua entrega com o objetivo de \textit{corrigir falhas, melhorar o
	desempenho ou adaptá-lo devido à mudanças ambientais}~\cite{{159342}}.  
De maneira relacionada, \textit{Manutenibilidade} é a propriedade de um sistema ou componente de software em relação ao grau
de \textit{facilidade} que ele pode ser corrigido, melhorado ou adaptado~\cite{{159342}}.

As manutenções em software podem ser divididas em \textit{Corretiva, Adaptativa, Perfectiva e
	Preventiva}~\cite{Lientz:1980:SMM:601062,159342}. A ISO 14764 discute
os quatro tipos de manutenções, conforme já descrito, e além disso propõe que exista um elemento comum
denominado \textit{Requisição de Mudança} que representa as características comuns a todas aqueles
tipos de manutenção.

Por conta do volume das Requisições de Mudança se faz necessária a utilização de
ferramentas com o objetivo de gerenciá-las. Esse controle é geralmente
realizado por Ferramentas de Gerenciamento de Requisição de Mudança -~FGRM, que auxiliam os
desenvolvedores na correção de forma individual ou colaborativa de defeitos (bugs), no
desenvolvimento de novas funcionalidades, dentre outras tarefas relativas à manutenção de software.
A literatura não define uma nomenclatura comum para este tipo de ferramenta. Em alguns estudos é
possível verificar nomes tais como Sistema de Controle de Defeito -~Bug Tracking Systems, Sistema de
Gerenciamento da Requisição -~Request Management System, Sistemas de Controle de Demandas (SCD)-
Issue Tracking Systems. Todavia, de modo geral, o termo se refere as
ferramentas utilizadas pelas organizações para \textit{gerir as Requisições de Mudança}. Estas
ferramentas podem ainda ser utilizadas por gestores, analistas de qualidade e usuários finais para
atividades tais como gerenciamento de projetos, comunicação, discussão e revisões de código. Neste
trabalho utilizaremos o termo \texttt{Ferramentas de Gerenciamento de Requisições de Mudança} (FGRM)
ao referimos a este tipo de ferramenta.  A Tabela~\ref{tab:exemplo} apresenta alguns exemplos de
ferramentas que podem ser classificadas como FGRM's. Também são listados serviços da Internet que
oferecem funcionalidades presentes nas FGRM na forma de Software como
Serviço~\cite{fox2013engineering}.

\begin{table}[ht]
	\centering
	\resizebox{\textwidth}{!}{%
		\begin{tabular}{llll}
			\hline
			\multicolumn{2}{c}{\textbf{Ferramentas}}           & \multicolumn{2}{c}{\textbf{Serviços da Internet}} \\ \hline
			Bugzilla & https://www.bugzilla.org/               & SourceForge    & https://sourceforge.net/    \\ \hline
			MantisBT & https://www.mantisbt.org/               & Lauchpad       & https://launchpad.net/      \\ \hline
			Trac     & https://trac.edgewall.org/              & Code Plex      & https://www.codeplex.com/   \\ \hline
			Redmine  & www.redmine.org/                        & Google Code    & https://code.google.com/    \\ \hline
			Jira     & https://www.atlassian.com/software/jira & GitHub         & https://github.com/         \\ \hline
		\end{tabular}%
	}
	\caption{Exemplos de ferramentas e serviços da Internet. Adaptado de~\cite{cavalcanti2014challenges}}\label{tab:exemplo}
\end{table}


\section{Justificativa}
\label{sec:intro-justificativa}
Desde o final da década de 1970~\cite{Zelkowitz:1979:PSE:578504} percebe-se o aumento do custo
referente as atividades de  manutenção de software. Nas décadas de 1980 e 1990 alguns trabalhos
tiveram seu foco no desenvolvimento de modelos de mensuração do custo para manter o
software~\cite{Herrin:1985:SMC:323287.323383,hirota1994approach}. Apesar da evolução das metologias
de manutenção a estimativa é que nas últimas duas décadas o custo de manutenção tenha aumentado em
50\%~\cite{koskinen2010software}. Esta tendência pode ser observada na
Figura~\ref{fig:software-maintence-costs} no qual é possível verificar a evolução do custo da
manutenção de software como fração do custo total do produto.

\begin{figure}
\centering
\includegraphics[width=0.7\linewidth]{./chapter-intro/img/software-maintence-costs}
\caption{Evolução da manutenção de software como percentual do custo total.	Extraído de~\cite{engelbertink2010save}}
\label{fig:software-maintence-costs}
\end{figure}

Diante da maior presença de software em todos os setores da sociedade
existe um interesse por parte da academia e da industria no desenvolvimento de
processos, técnicas e \textit{ferramentas} que reduzam o esforço e o custo das tarefas
de desenvolvimento e manutenção de software. Neste linha, o trabalho de Yong \& Mookerjee~\cite{1423995}  propõe um modelo que reduz os custos de manutenção e reposição durante a vida útil de um sistema de software. O modelo demonstrou que em algumas situações é \textit{melhor substituir um sistema do que mantê-lo}. Este problema é agravado tendo em vista que o custo de manutenção pode chegar a 60\% do custo total do software~\cite{kaur2015review}. Este percentual reflete a fração de desenvolvedores dedicados à tarefas de manutenção de sistemas~\cite{Zhang_2003}.

A manutenção não necessariamente exige que o processo de software envolvido
seja o tradicional. Percebe-se alguns exemplos de adoção das práticas ágeis
para fins de manutenção e evolução do software~\cite{kajko2009model, Heeager2015, Devulapally2015,Naz2016}. Tal
tendência não é surpreendente tendo em vista que os métodos ``ágeis'' enfatizam
características úteis à eficiência da implementação de software, tais como desenvolvimento incremental e teste contínuo que agregam valor para a evolução e manutenção eficaz de um sistema
\cite{thomas2006agile}. Dentro desta tendência verifica-se a necessidade de que as ferramentas envolvidas no suporte à manutenção de software se adéquem à este nova forma de manter software. 

\section{Motivação}
\label{sec:intro-motivacao}

\todo[inline]{Entender a diferença entre justificativa e motivação}


Da mesma forma que ocorre no desenvolvimento de software, é possível verificar uma crescente adoção
de técnicas da metodologia ágil na manutenção de software~\cite{Soltan2016,Devulapally2015,
	Heeager2015}. Neste contexto, é natural que ferramentas que dão suporte à manutenção, tal como
as FGRM's, tenham que evoluir para se adaptar a esta nova forma de trabalhar. Mesmo em um ambiente
tradicional de  desenvolvimento e manutenção de software, verifica-se a necessidade de adequação das
FGRM's. Uma das justificativas desta exigência se deve ao fato que a maioria desses sistemas são
projetados em torno do termo ``demanda' (bug, defeito, bilhete, recurso, etc.), contudo, cada vez
mais este modelo parece estar distante das necessidades práticas dos projetos de software,
resultando, por exemplo, que os desenvolvedores tenham um
baixo entendimento da situação geral bem como das atividades das outras pessoas envolvidas no
projeto~\cite{Baysal:2013:SAP:2486788.2486957}


\section{Problema}
\label{sec:intro-problema}

Apesar da inegável importância das FGRM's, percebe-se um aparente desacoplamento deste tipo de
ferramenta com as necessidades das diversas partes interessadas (stakeholders) na manutenção e
evolução de software. Um sinal deste distanciamento pode ser observado pelas diversas extensões
(plugins) propostas na literatura
\cite{101186,Thung:2014:BIT:2635868.2661678,Kononenko:2014:DED:2591062.2591075}.
 
O desenvolvimento e a manutenção de software envolvem diversos tipos de métodos, técnicas e
ferramentas. Em especial no processo de manutenção, um importante aspecto são as diversas
Requisições de Mudanças que devem ser gerenciadas. Este controle é realizado pelas Ferramentas de
Gerenciamento de Requisição de Mudanças (FGRM) cujo o uso vem crescendo em importância, sobretudo,
por sua utilização por gestores, analistas da qualidade e usuários finais para atividades como
tomada de decisão e comunicação.

A utilização de  \textit{``demanda''} como conceito central para Ferramentas de Gerenciamento de
Requisição de Mudanças (FGRM) parece ser distante das necessidades práticas dos projetos de
software, especialmente no ponto de vista dos desenvolvedores
\cite{Baysal:2013:SAP:2486788.2486957}. Um exemplo deste desacoplamento das FGRM com a necessidade
de seus usuários pode ser visto no trabalho proposto por Baysal \& Holme
\cite{baysal2012qualitative} no qual desenvolvedores que utilizam o
Bugzilla\footnote{\url{https://www.bugzilla.org}} relatam a dificuldade em manter uma compreensão
global das RM's em que eles estão envolvidos. Segundo os desenvolvedores seria interessante que a
ferramenta tivesse um suporte melhorado para a Consciência Situacional -~Situational Awareness. Em
síntese, eles gostariam de estar cientes da situação global do projeto bem como das atividades que
outras pessoas estão realizando. Um outro sinal da necessidade de evolução deste tipo de ferramenta
pode ser observado considerando as diversas extensões (plugins) propostas na literatura
\cite{101186,Thung:2014:BIT:2635868.2661678,Kononenko:2014:DED:2591062.2591075}.

\section{Visão Geral da Trabalho}
\label{sec:intro-visao-geral}

Neste sentido, este trabalho de dissertação investiga e contribui no entendimento de
como as Ferramentas de Gerenciamento de Requisição de Mudança estão sendo melhoradas ou estendidas
no contexto da transformação do processo de desenvolvimento e manutenção de software de um modelo
tradicional para outro que incorpora cada vez mais as práticas propostas pelos agilistas. O intuito
foi analisar como as FGRM estão sendo modificadas com base na literatura da área em contraste com o
ponto de vista dos profissionais envolvidos em manutenção de software.

Neste contexto, elaboramos um estudo das Ferramentas de Gerenciamento de Requisição de Mudança (FGRM) com o objetivo:
\begin{enumerate}[(i)]
	\item entender os requisitos comuns deste tipo de ferramenta;
	\item mapear as extensões para as FGRM que estão sendo propostas na literatura;
	\item avaliar sobre o ponto de vista dos profissionais a situação atual dos FGRM\@;
	\item propor novas extensões para as FGRM\@.  
\end{enumerate}


Vamos discutir os aspectos que são considerados mais importantes a partir da literatura da área, bem
como do ponto de vista de profissionais envolvidos em manutenção de software. De forma particular,
iremos estudar os mecanismos de personalização que algumas destas ferramentas permitem e tentaremos
ainda criar exemplos de personalização para alguma possível extensão a ser identificada ao longo do
trabalho.

\section{Metodologia de Pesquisa}
\label{sec:intro-metodologia}

O trabalho de dissertação pode ser dividido nas etapas listadas a seguir:

\begin{itemize}[(i)]
	\item Mapeamento Sistemático da Literatura~\cite{keele2007guidelines}
	\item Caracterização das Ferramentas de Gerenciamento de Requisição de Mudança (FGRM)
	\item Pesquisa (Survey) com os desenvolvedores~\cite{wohlin2012experimentation}
	\item Desenvolvimento de extensões para as FGRM's
\end{itemize}

\section{Contribuições da Dissertação}
\label{sec:intro-contribuicao}

\section{Organização da Dissertação}
\label{sec:intro-organizacao-dissertacao}
