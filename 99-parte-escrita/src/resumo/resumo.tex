Dentro do ciclo de vida de um produto de software o processo de manutenção tem
papel fundamental. Devido ao seu custo, em alguns casos chegando a 60\% do
montante investido~\cite{kaur2015review}, as atividades relacionadas a manter e
evoluir software têm sua importância considerada tanto pela comunidade
científica quanto pela indústria.

As manutenções em software podem ser divididas em \textit{Corretiva, Adaptativa,
Perfectiva e Preventiva}~\cite{Lientz:1980:SMM:601062,159342}. A Manutenção
Corretiva lida com a reparação de falhas encontradas. A Adaptativa tem o seu
foco na adequação do software por conta de mudanças ocorridas no ambiente em que
ele está inserido. A Perfectiva trabalha para detectar e corrigir falhas
latentes antes que elas se manifestem como tal. A Preventiva se preocupa com
atividades que possibilitem aumento da manutenibilidade do sistema. A ISO
14764~\cite{1703974} propõe que exista um elemento denominado Requisição de
Mudança (RM) que corresponde a uma agregação de características que representam
uma solicitação de manutenção de qualquer das quatro categorias.

Por conta do volume das Requisições de Mudança é necessária a utilização de
ferramentas com o objetivo de gerenciá-las. Esse controle é geralmente realizado
por Sistemas de Controle de Demandas (SCD)- Issue Tracking Systems, que auxiliam
os desenvolvedores na correção, de forma individual ou colaborativa, de defeitos
(bugs), no desenvolvimento de melhorias ou de novas funcionalidades. Não existe
na literatura uma nomenclatura comum para este tipo de ferramenta. Nesta
dissertação utilizamos o termo \texttt{Ferramentas de Gerenciamento de
Requisições de Mudança} (FGRM) ao referimos a este tipo de software.

Apesar da inegável importância das FGRMs, percebe-se um aparente desacoplamento
deste tipo de ferramenta com as necessidades das diversas partes interessadas
(stakeholders) na manutenção e evolução de um software. Um sinal deste
distanciamento pode ser observado pelas diversas extensões (plugins) propostas
na literatura
\cite{101186,Thung:2014:BIT:2635868.2661678,Kononenko:2014:DED:2591062.2591075}
e por estudos que estão propondo melhorias para este tipo de
software~\cite{zimmermann2010makes, cavalcanti2014challenges,
zimmermann2009improving}. Neste sentido, este trabalho de dissertação se propõe
a investigar e contribuir no entendimento de como as FGRMs estão sendo
melhoradas ou estendidas no contexto da transformação do processo de
desenvolvimento e manutenção de software de um modelo tradicional para outro que
incorpora cada vez mais as práticas propostas pelos agilistas. O intuito é
analisar como as FGRM estão sendo modificadas com base na literatura da área ao
mesmo tempo que consideramos o ponto de vista dos profissionais envolvidos com
Manutenção de Software.

Neste trabalho de dissertação realizamos um estudo exploratório com o objetivo
de entender as funcionalidade propostas na literatura e aquelas já existentes de
modo a melhorá-las. Foi realizado um Mapeamento Sistemático da Literatura a fim
de avaliar os trabalhos já existentes nesta área; também foi conduzido um estudo
exploratório na documentação de algumas ferramentas deste tipo de modo a
caracterizá-las. Para coletarmos o ponto de vista dos profissionais envolvidos
em desenvolvimento e manutenção de software foi conduzido um Levantamento com
questionário (survey) com o objetivo de apurar como os respondentes  avaliam as
funcionalidades existentes e as melhorias que possam ser realizadas neste tipo
de software. Com base no conhecimento adquirido foi proposto um conjunto de
melhorias para este tipo de ferramenta que tiveram uma boa aceitação quando
foram validades com profissionais que desenvolvem FGRMs. Uma das recomendações
propostas foi implementada como Prova de Conceito e apresentou resultados
satisfatórios.

\keywords{Engenharia de Software, Manutenção de Software, Ferramentas de
	Gerenciamento de Requisições de Mudança, Melhorias}
