%%%%%%%%%%%%%%%%%%%%%%%%%%%%%%%%%%%%%%%%%%%%%%%%%%%%%%%%%%%%%%%%%%%%%%%%%%%%%%%%
%Objetivo: Propor um conjunto de recomendações de melhorias para as FGRM's
%Autores: Vagner Clementino <vagnercs@dcc.ufmg.br>
%		  Rodolfo Resende <rodolfo@dcc.ufmg.br>
%Criação: dom fev 26 12:49:27 BRT 2017
%Modificação:
%Revisão:
%%%%%%%%%%%%%%%%%%%%%%%%%%%%%%%%%%%%%%%%%%%%%%%%%%%%%%%%%%%%%%%%%%%%%%%%%%%%%%%%
\chapter{Sugestões de Melhorias para as FGRM's}
\label{ch:sug_melhoria}

\section{Introdução}
\label{sec:sug_melhoria_intro}

Conforme foi discutido neste estudo é inegável a importância das Ferramentas de
Gerenciamento de Requisições de Mudança \@-\@ FGRM no contexto manutenção de
Software. Conforme foi discutido no
Capítulo~\ref{ch:caracterizacao_ferramentas} este tipo de sistema oferece
suporte ao relato das Requisições de Mudança \@-\@ RM, ao processo de
atribuição de determinada RM ao desenvolvedor mais apto, integração com
Sistemas de Controle de Versão \@-\@ SCV, dentre outras. Os usuários deste tipo
de software, em especial os ligados à manutenção de software, se mostraram, em
geral, satisfeitos com as funcionalidades oferecidas para o desempenho do seu
trabalho. O percentual de quase 90\% dos participantes da pesquisa descrita no
Capítulo~\ref{ch:pesquisa-profissionais} fizeram uma avaliação positiva, ao
mesmo tempo que a mesma quantidade de respondentes afirmam que recomendariam a
utilização da FGRM utilizada em um novo projeto.

Não obstante, naquele mesmo levantamento, ao questionarmos se o profissional
sentiria falta de determinada funcionalidade, do qual listamos algumas, cerca de
15\% afirmaram não necessitar dos itens apontados em sua rotina de trabalho. A
partir desta última informação podem inferir que os desenvolvedores estão
satisfeitos com a ferramenta utilizada, contudo, não conhecem ou têm acesso ao
potencial de funções que este tipo software pode oferecer.

Diante do exposto, entendemos que podemos contribuir com o estado atual das
funcionalidades das FGRM's apresentando um conjunto de sugestões de melhorias.
As sugestões foram compiladas utilizando os resultados obtido nesta dissertação,
especialmente com base nos
Capítulo~\ref{ch:mapeamento-sistematico},~\ref{ch:caracterizacao_ferramentas}
e~\ref{ch:pesquisa-profissionais}, nos estudo que propõe melhorias para as
FGRM~\cite{zimmermann2009improving, bettenburg2008makes, singh2011bug} e na
experiência dos autores. Estas diretrizes podem ser utilizadas por pesquisadores
interessados no tema com o objetivo de modo a conduzir estudos estudos sobre
melhoria da produtividade dos desenvolvedores mediante o uso das FGRM's. Além
disso, os responsáveis pelo desenvolvimento deste tipo de software podem
utilizar este conjunto a fim de implementar futuras versões do software. Na
mesma linha, os profissionais envolvidos em manutenção de software podem propor
extensões (plugins) para as FGRM de modo a utilizar as melhorias propostas neste
estudo em sua rotina de trabalho.

Este capítulo está organizado da seguinte forma: a
Seção~\ref{sec:sug_melhoria_melhorando_as_ferraementas} apresenta as sugestões
de melhoria do qual acreditamos poderia ser implantadas nas FGRM's, cada
sugestão apresenta foi seguida de uma breve justificativa de como foi obtida e
dos motivos de sua implementação; na
Seção~\ref{sec:sug_melhoria_avaliacao_das_melhorias} realizamos a avaliação das
sugestões que foram propostas, onde solicitamos a opinião de profissionais que
participam de projetos de código aberto que desenvolvem FGRM's; a
Seção~\ref{sec:sug_melhoria_discussao} discutimos o resultado obtidos do
processo de avaliação; na Seção~\ref{sec:sug_melhoria_ameacas} apresentamos as
ameaças à validade do trabalho realizado neste capítulo; encerramos esta parte
do estudo com um breve resumo na Seção~\ref{sec:sug_melhoria_resumo}.

\section{Sugestões de Melhorias para as FGRM's}
\label{sec:sug_melhoria_melhorando_as_ferraementas}

Nesta seção apresentamos um conjunto de recomendação de melhorias das
funcionalidades das FGRM's. As sugestões propostas não estão vinculadas
exclusivamente à funcionalidades já existentes neste tipo de ferramenta. O que
está sendo proposto pode representar o desenvolvimento de um novo tipo de
comportamento para as FGRM's. Cabe-nos ressaltar que o conjunto proposto não é
exaustivo e é baseado nos resultados desta dissertação. Além disso não houve
compromisso com as dificuldades operacionais que implementação das
funcionalidades pode estar relacionada, mesmo porquê avaliar esta complexidade
está fora do escopo deste estudo.


Conforme discutido na Subseção~\ref{} o responsável por reportar uma RM pode ser
tanto um usuário do sistema quando um membro da equipe de desenvolvimento e
manutenção. Este diferente nível de conhecimento do software acarretar em
diferente níveis de qualidade do que é relato. Esta situação pode causar atraso
na análise da RM por falta da informação necessária a sua resolução. Alguns
estudos se dedicam a minimizar este problema através da análise da qualidade do
que é relatado em uma RM. A premissa é que o responsável pelo relato tenha
ciência das informações que são necessárias à resolução do que foi solicitado.
Para este fim apresentamos a sugestão descrita a seguir.

\sugestao{1}{As FGRM's devem fornecer um retorno(\textit{feedback}) da qualidade
do relato realizado em uma RM.}

As RM permitem a inclusão de código fonte em diversas etapas do seu ciclo de
vida. O código pode ser incluído durante a sua criação, nas discussões
realizadas para a sua resolução ou mesmo quando é concluída, onde recebe o nome
de \textit{patch.} Esta informação é bastante relevante para o projeto do qual a
RM faz parte, contudo, as FGRM's não permitem a sua recuperação. Neste sentido,
sugerimos uma funcionalidade para atender a recomendação à seguir.

\sugestao{2}{As FGRM's devem possibilitar a busca por código fonte contido em
	seu relato, comentários ou anexos.}

Caso seja possível identificar que um \textit{Reportador} tem por hábito relatar
uma RM com qualidade e de forma relevante.

\sugestao{3}{As FGRM's devem diferenciar os reportadores que relatam
	RM's com maior qualidade e relevância.}


\section{Avaliação das Melhorias}
\label{sec:sug_melhoria_avaliacao_das_melhorias}

Avaliar as sugestões enviando \textit{pull request} no \textit{Github} para os
desenvolvedores de FGRM's de código aberto. Eles poderiam dizer ``concordo' ou
``não concordo' com as sugestões. O percentual de

\section{Discussão}
\label{sec:sug_melhoria_discussao}

\section{Ameaças à Validade}
\label{sec:sug_melhoria_ameacas}

\section{Resumo do Capítulo}
\label{sec:sug_melhoria_resumo}
