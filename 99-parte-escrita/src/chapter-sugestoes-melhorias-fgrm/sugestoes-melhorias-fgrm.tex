\chapter{Sugestões de Melhorias para Ferramentas de Gerenciamento de Requisição de Mudança}
\label{ch:extensoes}

\section{Introdução}
\label{sec:sug_melhoria_intro}

Conforme foi discutido durante este estudo é inegável a importância das
Ferramentas de Gerenciamento de Requisições de Mudança \@-\@ FGRM no contexto
manutenção de Software. Conforme foi discutido no
Capítulo~\ref{ch:caracterizacao_ferramentas} este tipo de sistema oferece
suporte ao relato das Requisições de Mudança \@-\@ RM, ao processo de atribuição
de determinada RM ao desenvolvedor mais apto, integração com Sistemas de
Controle de Versão \@-\@ SCV, dentre outras. Os usuários deste tipo de software,
em especial os ligados à manutenção de software, se mostraram, em geral,
satisfeitos com as funcionalidades oferecidas para o desempenho do seu
trabalho. O percentual de quase 90\% dos participantes da pesquisa descrita no
Capítulo~\ref{ch:pesquisa-profissionais} fizeram uma avaliação positiva, ao
mesmo tempo que a mesma quantidade de respondentes afirmam que recomendariam a
utilização da FGRM utilizada em um novo projeto.

Não obstante, naquele mesmo levantamento, ao questionarmos se o profissional
sentiria falta de determinada funcionalidade, do qual listamos algumas, cerca de
15\% afirmaram não necessitar dos itens apontados em sua rotina de trabalho. A
partir desta última informação podem inferir que os desenvolvedores estão
satisfeitos com a ferramenta utilizada, contudo, não conhecem ou têm acesso ao
potencial de funções que este tipo software pode oferecer.

Diante do exposto, entendemos que podemos contribuir com o estado atual das
funcionalidades das FGRM's apresentando um conjunto de sugestões de melhorias.
As sugestões foram compiladas utilizando os resultados obtido nesta dissertação,
especialmente com base nos
Capítulo~\ref{ch:mapeamento-sistematico},~\ref{ch:caracterizacao_ferramentas}
e~\ref{ch:pesquisa-profissionais}, nos estudo que propõe melhorias para as
FGRM~\cite{zimmermann2009improving, bettenburg2008makes, singh2011bug} e na
experiência dos autores. Estas diretrizes podem ser utilizadas por pesquisadores
interessados no tema com o objetivo de modo a conduzir estudos estudos sobre
melhoria da produtividade dos desenvolvedores mediante o uso das FGRM's. Além
disso, os responsáveis pelo desenvolvimento deste tipo de software podem
utilizar este conjunto a fim de implementar futuras versões do software. Na
mesma linha, os profissionais envolvidos em manutenção de software podem propor
extensões (plugins) para as FGRM de modo a utilizar as melhorias propostas neste
estudo em sua rotina de trabalho.

Este capítulo está organizado da seguinte forma: a
Seção~\ref{sec:sug_melhoria_melhorando_as_ferraementas} apresenta as sugestões
de melhoria do qual acreditamos poderia ser implantadas nas FGRM's, cada
sugestão apresenta foi seguida de uma breve justificativa de como foi obtida e
dos motivos de sua implementação; na
Seção~\ref{sec:sug_melhoria_avaliacao_das_melhorias} realizamos a avaliação das
sugestões que foram propostas, onde solicitamos a opinião de profissionais que
participam de projetos de código aberto que desenvolvem FGRM's; a
Seção~\ref{sec:sug_melhoria_discussao} discutimos o resultado obtidos do
processo de avaliação; na Seção~\ref{sec:sug_melhoria_ameacas} apresentamos as
ameaças à validade do trabalho realizado neste capítulo; encerramos esta parte
do estudo com um breve resumo na Seção~\ref{sec:sug_melhoria_resumo}.

\section{Melhorando as FGRM's}
\label{sec:sug_melhoria_melhorando_as_ferraementas}

Nesta seção apresentamos o estudo realizado que levam as recomendações sobre
como as FGRM's poderiam ser melhorados.

\section{Avaliação das Melhorias}
\label{sec:sug_melhoria_avaliacao_das_melhorias}

Avaliar as sugestões enviando \textit{pull request} no \textit{Github} para os
desenvolvedores de FGRM's de código aberto. Eles poderiam dizer ``concordo' ou
``não concordo' com as sugestões. O percentual de

\section{Discussão}
\label{sec:sug_melhoria_discussao}

\section{Ameaças à Validade}
\label{sec:sug_melhoria_ameacas}

\section{Resumo do Capítulo}
\label{sec:sug_melhoria_resumo}
