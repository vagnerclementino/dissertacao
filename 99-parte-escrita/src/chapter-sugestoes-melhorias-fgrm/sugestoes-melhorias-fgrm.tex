%%%%%%%%%%%%%%%%%%%%%%%%%%%%%%%%%%%%%%%%%%%%%%%%%%%%%%%%%%%%%%%%%%%%%%%%%%%%%%%%
%Objetivo: Propor um conjunto de recomendações de melhorias para as FGRM's
%Autores: Vagner Clementino <vagnercs@dcc.ufmg.br>
%		  Rodolfo Resende <rodolfo@dcc.ufmg.br>
%Criação: dom fev 26 12:49:27 BRT 2017
%Modificação: qui mar  9 05:19:57 BRT 2017
%Revisão:
%%%%%%%%%%%%%%%%%%%%%%%%%%%%%%%%%%%%%%%%%%%%%%%%%%%%%%%%%%%%%%%%%%%%%%%%%%%%%%%%
\chapter{Sugestões de Melhorias para as FGRM's}
\label{ch:sug_melhoria}

\section{Introdução}
\label{sec:sug_melhoria_intro}

Conforme já foi discutido nesta dissertação é inegável a importância das
Ferramentas de Gerenciamento de Requisições de Mudança \@-\@ FGRM no contexto da
manutenção de software. Conforme apresentado na
Seção~\ref{sec:caracterizacao_ferramentas} este tipo de sistema oferece suporte
ao relato das Requisições de Mudança \@-\@ RM, ao processo de atribuição das
RM's ao desenvolvedor mais apto, integração com Sistemas de Controle de Versão
\@-\@ SCV, dentre outras funcionalidades. Os usuários deste tipo de software, em
especial os ligados à manutenção de software, se mostraram, em geral,
satisfeitos com as funcionalidades oferecidas para o desempenho do seu trabalho.
O percentual de cerca de 90\% dos par\-ti\-ci\-pan\-tes da pesquisa descrita no
Capítulo~\ref{ch:pesquisa-profissionais} fizeram uma avaliação positiva, ao
mesmo tempo que a mesma quantidade de respondentes afirmam que recomendariam a a
FGRM que utilizam para um novo projeto.

Não obstante, naquele mesmo levantamento, ao questionarmos se o profissional
sentiria falta de determinada funcionalidade, do qual listamos algumas, cerca de
15\% afirmaram não necessitar dos itens apontados em sua rotina de trabalho. A
partir desta última informação podemos inferir que os desenvolvedores estão
satisfeitos com a ferramenta utilizada, contudo, \textit{não conhecem ou não têm
	acesso ao potencial de funções que este tipo software pode oferecer}.

Diante do exposto, entendemos que podemos contribuir com o estado a\-tu\-al das
funcionalidades das FGRM's apresentando um conjunto de sugestões de melhorias.
As sugestões foram compiladas utilizando os resultados obtidos nesta
dissertação, especialmente com base nos
Capítulos~\ref{ch:mapeamento-sistematico} e~\ref{ch:pesquisa-profissionais}, na
Seção~\ref{sec:caracterizacao_ferramentas} e nos estudos que propõem melhorias
para as FGRM~\cite{zimmermann2009improving, bettenburg2008makes, singh2011bug}.
Estas recomendações podem ser utilizadas por pesquisadores interessados em
conduzir estudos sobre melhoria da produtividade dos desenvolvedores mediante o
uso das FGRM's. Além disso, os responsáveis pelo desenvolvimento deste tipo de
software podem utilizar este conjunto a fim de implementar futuras versões do
projeto. Na mesma linha, os profissionais envolvidos em manutenção de software
podem desenvolver extensões (plugins) para as FGRM com base no que foi proposto
de modo a utilizar as melhorias propostas neste estudo em sua rotina de
trabalho.

Este capítulo está organizado da seguinte forma: a
Seção~\ref{sec:sug_melhoria_melhorando_as_ferraementas} apresenta as sugestões
de melhoria do qual acreditamos poderia ser implantadas nas FGRM's, cada
sugestão apresenta foi seguida de uma breve justificativa de como foi obtida e
dos motivos de sua implementação; na
Seção~\ref{sec:sug_melhoria_avaliacao_das_melhorias} realizamos a avaliação das
sugestões que foram propostas, onde solicitamos a opinião de profissionais que
participam de projetos de código aberto que desenvolvem FGRM's; na
Seção~\ref{sec:sug_melhoria_discussao} discutimos os resultados obtidos do
processo de avaliação; na Seção~\ref{sec:sug_melhoria_ameacas} apresentamos as
ameaças à validade deste capítulo; encerramos esta parte do estudo com um breve
resumo na Seção~\ref{sec:sug_melhoria_resumo}.

\section{Sugestões de Melhorias para as FGRM's}
\label{sec:sug_melhoria_melhorando_as_ferraementas}

Nesta seção apresentamos um conjunto de recomendações de melhorias das
funcionalidades das FGRM's. As sugestões propostas não estão vinculadas
exclusivamente à melhorias de funcionalidades já existentes neste tipo de
ferramenta. O que está sendo proposto pode representar o desenvolvimento de um
novo tipo de comportamento para as FGRM's. Cabe-nos ressaltar que o conjunto
proposto não é exaustivo e é baseado nos resultados desta dissertação. Além
disso não houve compromisso com as dificuldades operacionais que implementação
das funcionalidades podem estar relacionadas, mesmo porquê avaliar esta
complexidade está fora do escopo deste estudo. É possível que algumas das
su\-ges\-tões propostas já estejam implementadas de maneira parcial ou integral
em alguma FGRM\@.  Contudo, não é possível validar esta premissa por conta de
volume de ferramentas disponíveis quando esta dissertação foi escrita.

Conforme discutido na
Subseção~\ref{subsec:man_visao_geral_papeis_na_manutencao_de_software} o
responsável por reportar uma RM pode ser tanto um usuário do sistema quando um
membro da equipe de desenvolvimento/manutenção. Neste caso existem diferentes
níveis de conhecimento sobre o sistema. Este diferente nivelamento pode
acarretar em diferentes níveis de qualidade do que é relato. Esta situação pode
causar atraso na análise da RM por falta da informação necessária a sua
resolução. Alguns estudos demonstram que, do ponto de vista dos desenvolvedores,
a falta de informação tais como etapas para reproduzir e registro de pilha de
ativação (stack tracke) dificultam mais o trabalho do que bugs
duplicados~\cite{bettenburg2008makes, bettenburg2007quality}. Neste linha, estes
estudos se dedicam a minimizar o problema através da análise da qualidade do que
é relatado em uma RM\@. A premissa é que o responsável pelo relato tenha ciência
das informações que são necessárias à resolução do que foi solicitado. Com este
objetivo apresentamos a sugestão de desenvolvimento de funcionalidade conforme
descrito a seguir.

\sugestao{01}{As FGRM's devem fornecer um retorno (\textit{feedback}) da qualidade
do relato realizado em uma RM.}

As RM permitem a inclusão de código fonte em diversas etapas do seu ciclo de
vida. O código pode ser incluído durante a sua criação, nas discussões
realizadas para a sua resolução ou mesmo quando é concluída, onde recebe o nome
de \textit{patch.} Esta informação é bastante relevante para o projeto do qual a
RM faz parte, contudo, as FGRM's não permitem a sua recuperação. Neste sentido,
é descrito a \textit{Sugestão 02} que representa uma funcionalidade com este
objetivo.

\sugestao{02}{As FGRM's devem possibilitar a busca por código fonte contido em
	seu relato, comentários ou anexos.}

Caso seja possível identificar que um \textit{Reportador} tem por hábito relatar
RM's e que sejam relevantes ao projeto de software, tais requisições deveriam
receber algum tipo de etiqueta de modo a diferenciá-las dentro da FGRM\@.
Segundo o nosso entendimento uma RM pode ser classificada como relevante se
descreve um problema que afeta um grande número de usuários do sistema ou
represente uma falha de segurança do software. O grau de relevância de
determinada RM pode variar em diferentes projetos e pode depender de critérios
subjetivos de quem analisa. Com objetivo de diferenciar os reportadores
apresentamos a \textit{Sugestão 03}.

\sugestao{03}{As FGRM's devem diferenciar os reportadores que relatam
	RM's com maior qualidade e relevância.}

No ciclo de vida de uma RM, conforme discutido na Subseção~\ref{}, após a
verificação de que a RM foi incorporada com sucesso ao software, ela é movida
para o estado \textit{Fechado (Closed)} e deixar de estar atribuída a
determinado desenvolvedor ou analista de qualidade. Caso um desenvolvedor queria
acessá-la novamente deverá utilizar o identificador da RM a fim de recuperá-la
na FGRM\@. Este histórico de trabalho do desenvolvedor pode ser útil na resolução
de eventuais RM que surjam no projeto. No levantamento mediante questionário
apresentado no Capítulo~\ref{ch:pesquisa-profissionais} alguns participantes
relataram o desejo de uma funcionalidade  que gerencie este histórico conforme
apresentado a seguir. Por esta razão apresentamos a \textit{Sugestão 04}.

\begin{itemize}
	\item Eu gostaria de:
	\begin{itemize}
		\item \textit{``The ability to clearly visualize how many tickets are at
				the to do, in progress, to validate or done steps.''}.
		\item \textit{``History tracking, commenting, attachments, priority
				setting, task assignment, tie in with deployment systems.''}
	\end{itemize}
\end{itemize}

\sugestao{04}{As FGRM's devem permitir acesso facilitado para as $n$ últimas
	RM's que foram analisadas por um desenvolvedor.}

\section{Avaliação das Melhorias Propostas}
\label{sec:sug_melhoria_avaliacao_das_melhorias}

Este Capítulo se propôs em apresentar um conjunto de sugestões que foram
construídas tomando como base a literatura da área e os resultados e
contribuições desta dissertação. Com o objetivo de avaliar a relevância e o grau
de facilidade de implementação das recomendações produzidas neste estudo
conduzimos um levantamento mediante questionário com profissionais que
contribuem em projeto de código aberto hospedados no Github. A metodologia
utilização na condução do levantamento é descrita na próxima subseção.

\subsection{Metodologia Levantamento com Questionário}
\label{sub:sug_melhoria_metodologia_levantamento}

Conforme discutido com maior detalhes no Capítulo~\ref{ch:} um levantamento com
questionário, conhecido na literatura como \textit{Survey}, é uma abordagem de
coleta e análise de dados na qual os participantes respondem a perguntas ou
declarações que foram desenvolvidas antecipadamente~\cite{kasunic2005designing}.
Para realizarmos a coleta dos dados foi utilizado um questionário eletrônico
produzido previamente através da ferramenta \textit{Survey
	Gizmo}~\footnote{\url{https://surveygizmo.com}}. O processo de seleção dos
participantes, o desenho do questionário e como foi realização a sua aplicação
estão descritos na próximas subseções.

\subsubsection{Seleção dos Participantes}
\label{ssub:sug_melhoria_selecao_participantes}

Ficou definido que o público-alvo deste questionário seria profissionais que
estejam ligados ao processo de desenvolvimento e manutenção de FGRM's. Este
perfil foi selecionado porque permite que seja avaliado a relevância das
sugestões propostas ao mesmo tempo que possibilita verificar a viabilidade de
implementação do que foi recomendado em funcionalidades para as FGRM's. Por esta
razão, selecionamos profissionais que atuam como \textit{contribuidores} em três
projetos de código aberto hospedados no Github.


Com cerca de 38 milhões de
repositórios\footnote{\url{https://github.com/features}. Acesso em junho/2016.},
GitHub é atualmente o maior repositório de código na Internet. Sua popularidade
e a disponibilidade de metadados, acessíveis através de uma API, tem tornando
GitHub bastante atrativo para a realização de pesquisas na área de Engenharia de
Software.

Para escolha dos projetos foi definido inicialmente um conjunto de critérios
baseados em boas práticas recomendadas na literatura~\cite{Bird2009}. Em
síntese, um projeto para ser escolhido deve atender aos simultaneamente
seguintes requisitos:

\begin{itemize}
	\item Os projetos devem representar o desenvolvimento de uma FGRM\@.
	\item Os projetos devem ter no mínimo seis meses de desenvolvimento, para
		evitar projetos que não tenham passado por um tempo de manutenção
		relevante.
	\item Os  projetos devem  ter  no  mínimo  200  revisões  pelos  mesmos
		motivos  da restrição anterior.
	\item Os projetos escolhidos não devem ser ramificações (\textsl{forks}) um
		do outro projeto, para evitar dados duplicados.
	\item Os projetos obtidos devem ser os 10 mais populares que atendem aos
		demais critérios, utilizando como métrica o campo \texttt{most stars}
\end{itemize}

Após aplicação dos critérios descritos obtivemos os projetos descritos na
Tabela~\ref{tab:projetos_utilizados_para_avaliacao}. Conforme pode ser observado
os projetos selecionados referem-se a ferramentas bem estabelecidas e largamente
utilizadas por organizações e projetos de código aberto.

\begin{table}[htpb]
\centering
\resizebox{\textwidth}{!}{%
\begin{tabular}{|c|c|c|c|c|c|}
\hline
\textbf{Projeto} & \textbf{Commits} & \textbf{Branches} & \textbf{Releases} &
\textbf{Contribuidores} & \textbf{Contrib.\ com E-mail} \\ \hline
bugzilla & 9784 & 30 & 460 & 100 & 30\\
mantisbt & 10181 & 8 & 65 & 80 & 32 \\
redmine & 13115 & 27 & 131 & 6 & 2 \\ \hline
\end{tabular}%
}
\caption{Projetos utilizados no levantamento com profissionais. Os dados
	apresentados tem como referência 07/03/2017.}
\label{tab:projetos_utilizados_para_avaliacao}
\end{table}


Com base nos projetos selecionados ficou definido que a amostra a ser utilizada
no levantamento seria os respectivos contribuidores. Um contribuidor é alguém
que participa efetivamente do desenvolvimento de um projeto, tendo o privilégio
de acesso para alterar o código fonte. O contato com os contribuidores foi
realizado por correio eletrônico, todavia, conforme pode ser verificado na
coluna \textit{Contrib.\ com E-mail} nem todos permitem acesso público ao seu
endereço de e-mail.

\subsubsection{Desenho do Questionário}
\label{ssub:sug_melhoria_desenho_questionario}


\subsubsection{Processo de Aplicação}
\label{ssub:processo_de_aplicação}







Avaliar as sugestões enviando \textit{pull request} no \textit{Github} para os
desenvolvedores de FGRM's de código aberto. Eles poderiam dizer ``concordo' ou
``não concordo' com as sugestões. O percentual de

\section{Discussão}
\label{sec:sug_melhoria_discussao}

\section{Ameaças à Validade}
\label{sec:sug_melhoria_ameacas}

\section{Resumo do Capítulo}
\label{sec:sug_melhoria_resumo}
