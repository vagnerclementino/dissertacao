%%%%%%%%%%%%%%%%%%%%%%%%%%%%%%%%%%%%%%%%%%%%%%%%%%%%%%%%%%%%%%%%%%%%%%%%%%%%%%%
%Objetivo: Caracterizas as Ferramentas de Gerenciamento de Requisição de Mudança
%		   com base nas suas funcionalidades
%Autor: Vagner Clementino <vagnercs@dcc.ufmg.br>
% 		Rodolfo Resende	<rodolfo@dcc.ufmg.br>
%Criação: qua set 28 11:25:03 BRT 2016
%Modificação: qua set 28 11:25:35 BRT 2016
%Revisão: Ter Out 11 19:17:27 BRT 2016
%%%%%%%%%%%%%%%%%%%%%%%%%%%%%%%%%%%%%%%%%%%%%%%%%%%%%%%%%%%%%%%%%%%%%%%%%%%%%%%
\chapter{Caracterização das Ferramentas de Gerenciamento de Requisição de
	Mudança}
\label{ch:caracterizacao}


\section{Introdução}
\todo[inline]{OBJETIVO:\@ Apresentar uma visão geral desta parte do estudo que
consistirá de uma análise da funcionalidades oferecidas pelas FGRM}

Quando uma organização ou projeto software de código aberto decide adotar uma
Ferramenta de Gerenciamento de Requisições de Mudança um desafio é encontrar
aquela que melhor atenda suas necessidades. Um dos possíveis critérios de
seleção esta nas funcionalidades oferecidas pelo software. Outros critérios
podem envolver o custo bem como o suporte a falhas da ferramenta.  De maneira
relacionada, o pesquisador que tenha o desejo de propor melhorias para as FGRM's
possivelmente está interessado em analisar o conjunto de funções comuns que
caracterizam as FGRM's.

O numero de FGRM disponíveis atualmente é bastante elevado. Em uma inspeção
inicial, verificamos a existência de mais de 50 ferramentas disponíveis
comercialmente ou de código
aberto\footnote{\url{https://en.wikipedia.org/wiki/Comparison_of_issue-tracking_systems}}.
Apesar das diferentes opções disponíveis, ao bem do nosso conhecimento,
desconhecemos estudos que avaliem de forma sistemática as funcionalidades
oferecidas por este tipo de software. Entendemos que a partir de um conjunto
compartilhado de funcionalidades seja possível caracterizar as FGRM's ao mesmo
tempo que possibilita avaliar a contribuição das extensões que estão sendo
propostas na literatura, conforme discutido no
Capítulo\ref{ch:mapeamento-sistematico}.  Para alcançarmos este objetivo
realizamos um estudo exploratório visando determinar quais são as
funcionalidades comuns às Ferramentas de Gerenciamento de Requisição de Mudança
(FGRM). Um estudo exploratório está preocupado na análise de um objeto em sua
configuração natural e deixa as descobertas surgirem da própria
observação~\cite{wohlin2012experimentation}. Neste tipo de estudo nenhuma
hipótese é definida.

O trabalho descrito neste capítulo consistiu na leitura da documentação de
algumas FGRM de modo a obter as funcionalidades oferecidas por cada ferramenta.
As funções foram coletadas e organizadas utilizando a técnica de Cartões
Ordenados (Sorting Cards)~\cite{5070993}. As FGRM' objeto deste estudo foram
escolhidas com ajuda de profissionais envolvidos em manutenção de software.
Através de uma pesquisa (survey) os profissionais foram questionados a definir
quais das ferramentas apresentadas eram as mais representativas dentro do
domínio de aplicação das FGRM's. A representativa neste contexto não está no
número de projetos que utiliza determinada ferramenta, mas pelas caraterísticas
que determinado software possui que o torna diferenciável dentro do seu domínio
de aplicação.

\section{Objetivo do Capítulo}
\label{sec:objetivo_do_capítulo}

O primeiro objetivo deste capítulo é determinar as funcionalidades comuns às
Ferramentas de Gerenciamento de Requisição de Mudanças tomando como ponto de
partida um conjunto mínimo de sistemas que foram definidas como as mais
relevantes por profissionais envolvidos em manutenção e desenvolvimento de
software. Em um segundo momento, o foco é caracterizar este tipo de ferramenta
tomando como base as funções oferecidas. Conforme já exposto, a literatura em
Manutenção de Software apresenta diferentes nomenclaturas para este tipo de
ferramenta (Sistema de Controle de Defeito~-~Bug Tracking Systems, Sistema de
Gerenciamento da Requisição~-~Request Management System, Sistemas de Controle de
Demandas (SCD)~-~Issue Tracking Systems), sem, contudo, se preocupar em
diferenciá-las.

Acreditamos que o resultado deste estudo permitirá compreender melhor este tipo
de software tomando como base as suas funcionalidades. Também será possível
propor novas funções ou melhorias das já existentes tendo em vista a
possibilidade de determinar o conjunto mínimo de funcionalidades deste tipo de
sistema. Uma outra possível contribuição é a criação de um
esquema de caracterização com base nas funcionalidades oferecidas.

\section{Metodologia}
\label{sec:metodologia}

A fim de determinarmos o conjunto comum de funcionalidades das FGRM realizamos
um estudo compostos das etapas descritas a seguir. O resultado do trabalho de
uma etapa fui utilizado como entrada da próxima. Antes do início de um nova fase
de trabalho o resultado foi avaliado afim de verificar possíveis
inconsistências. Cada uma das partes deste estudo é explicada em detalhes nas
próximas seções.

\begin{enumerate}[(i)]
	\item Seleção das Ferramentas
	\item Inspeção da Documentação
	\item Agrupamento das Funcionalidades
\end{enumerate}

\subsection{Seleção das Ferramentas}
\label{subsec:selecao-ferramentas}

A primeira etapa consistiu na definição das ferramentas que seriam utilizadas no
estudo. A partir de uma pesquisa na Internet obtivemos um conjunto inicial de
mais de 50 ferramentas
utilizadas\footnote{\url{https://en.wikipedia.org/wiki/Comparison_of_issue-tracking_systems}}.
O Anexo exibe o conjunto de ferramentas levantadas para este estudo.

Não obstante, devido ao pouco tempo disponível e devido a dificuldade de
realizar a análise em cada uma daquelas ferramentas, optamos por escolher um
subconjunto de sistemas que fossem mais representativos, tomando como base a
opinião de profissionais envolvidos em manutenção de software. A
representatividade neste caso corresponde a opinião do profissional sobre
notoriedade que a ferramenta possui dentro do seu domínio de aplicação em
comparação com as demais que lhe foram apresentadas.

A opinião dos profissionais foi obtida mediante a realização de uma pesquisa
(survey~\cite{wohlin2012experimentation}) através de um formulário eletrônico.
O formulário foi estruturado em duas partes principais: a formação de
base do participante (background) e a avaliação das ferramentas. Na primeira
parte da pesquisa com profissionais estávamos interessados em conhecer as
características do respondente, especialmente, como descreveremos a seguir, o
questionário foi replicado em três grupos distintos de profissionais. Na segunda
parte do estudo apresentamos as ferramentas e foi pedido que o profissional
avaliasse a relevância de cada uma através de questões de múltipla escolha
utiliza uma escala do tipo Likert~\cite{robbins2011plotting}.

Antes da efetiva aplicação do questionário nos profissionais envolvidos em
manutenção de software, o documento foi validade em um processo de três etapas.
Em um primeiro momento foi solicitado a dois pesquisadores experientes da área
de Engenharia de Software que avaliassem o formulário inicialmente desenhado. A
partir das sugestões obtidas dos pesquisadores efetuamos adequações no
formulário e enviamos para dois profissionais envolvidos diretamente em
manutenção de software. O critério utilizado para seleção dos profissionais foi
o tempo dedicado à tarefa de manter software, que no caso específico eram de
mais de dez anos. O formulário foi modificado com as sugestões dos profissionais
finalizando desta forma a segunda etapa de validação. A última etapa consistiu
na realização de um piloto com dez profissionais envolvidos em manutenção de uma
empresa pública de informática. Os profissionais tiveram que efetivamente
preencher o questionário, contudo, foram adicionadas questões as quais era
possível inserir sugestões de melhoria. O resultado deste processo de validação
é o questionário presente no Anexo
\todo[inline]{Incluir anexo com o formulário final}

A população de interesse deste survey é o conjunto de profissionais envolvidos
em manutenção de software. Naturalmente é difícil definir o tamanho e
características desta população de modo a definir uma amostra significativa.
Neste sentido, visando reduzir possíveis víeis deste estudo, o questionário foi
replicado em três grupos distintos:

\begin{description}
	\item[Grupo 01:] Profissionais de uma empresa pública de informática
	\item[Grupo 02] Profissionais que participam em projetos de código aberto
	\item[Grupo 03:] Profissionais que participam de grupos de interesse em
		aplicativos de comunicação em celular.	
\end{description}

:%We chose these projects because (1)
%they are active, real-world projects with a sizable code base, and
%large numbers of users and developers,  and (2) bug reports from
%these projects have been used previously in several studies involv-
%ing bug report analyses [3, 13, 2, 4]. 

As FGRM selecionadas foram inicialmente classificadas como
\textit{''ferramentas''} e \textit{''serviços da internet''}. O primeiro grupo
representa os softwares capazes de serem implantados na infraestrutura do seu
cliente e do qual permite algum grau de personalização de alguns componentes,
como por exemplo o bando de dados utilizado. No segundo grupo estão os software
que ofertam a gerência das RM's mediante uma arquitetura do tipo Software como
Serviço (Software as Service), onde certos tipos de alterações no comportamento
do software por parte do usuário são mais restritas. As ferramentas selecionadas
são apresentadas por tipo de categoria na
Tabela~\ref{tab:ferramentas_selecionadas}

\todo[inline]{Incluir referência sobre
	Software como Serviço}

\subsection{Inspeção da Documentação}
\label{subsec:inspecao_doumentacao}

Nesta etapa do trabalho realizamos a leitura do material disponível na internet
para cada uma das ferramentas selecionadas. Entre estes materiais utilizamos
manual do usuário, manual do desenvolvedor, notas de lançamento e etc. Para cada
uma das FGRM optamos por analisar a última versão estável do software a fim de
analisarmos o que há de mais novo disponível aos usuários. A Tabela apresenta a
versão analisada e o elo de ligação para cada uma da documentação utilizada
neste estudo. Para aquelas ferramentas que apresentam documentação em mais de um
idioma optamos sempre por utilizar aquela escrita em inglês por entendermos ser
a que possivelmente mais atualizada.

\subsection{Agrupamento das Funcionalidades}
\label{subsec:agrupamento_fucionalidades}

Esta etapa tem por objetivo agrupar as funcionalidades que aparecem com
nomenclatura distintas em diferentes ferramentas, mas que apresentam o mesmo
significado. Cabe ressaltar que o agrupamento de algumas funcionalidades pode
depender de uma análise subjetiva de quem está realizando a atividade.  Neste
sentido, a fim de evitar algum tipo de viés o agrupamento foi realizado em duas
etapas:

\begin{description}
	\item[Análise Individual] Neste etapa o autor e um outro especialista
		realizam de forma separada os agrupamentos que acharem necessários
	\item[Anaĺise Compartilhada] Em um segundo momento tanto o auto quanto o
		especialista discutem as possíveis divergências até que um consenso seja
		obtido.
\end{description}

Após as duas etapas descritas anteriormente chegamos ao conjunto de
funcionalidades descritos na Tabela. A coluna agrupamento exibe o nome comum
para representas as funções descritas na coluna ``nomenclatura original''.



\section{Resultados}
\label{sec:resultados}



\section{Discussão}
\label{sec:discussao}

\section{Ameças à Validade}
\label{sec:ameacas_a_validade}


\section{Resumo do Capítulo}
\label{sec:resumo_do_capitulo}