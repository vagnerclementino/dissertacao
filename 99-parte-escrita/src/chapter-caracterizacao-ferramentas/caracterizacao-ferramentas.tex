%%%%%%%%%%%%%%%%%%%%%%%%%%%%%%%%%%%%%%%%%%%%%%%%%%%%%%%%%%%%%%%%%%%%%%%%%%%%%%%
%Objetivo: Caracterizas as Ferramentas de Gerenciamento de Requisição de Mudança
%		   com base nas suas funcionalidades
%Autor: Vagner Clementino <vagnercs@dcc.ufmg.br>
% 		Rodolfo Resende	<rodolfo@dcc.ufmg.br>
%Criação: qua set 28 11:25:03 BRT 2016
%Modificação: qua set 28 11:25:35 BRT 2016
%Revisão: qua set 28 11:25:27 BRT 2016
%%%%%%%%%%%%%%%%%%%%%%%%%%%%%%%%%%%%%%%%%%%%%%%%%%%%%%%%%%%%%%%%%%%%%%%%%%%%%%%
\chapter{Caracterização das Ferramentas de Gerenciamento de Requisição de
	Mudança}
\label{ch:caracterizacao}


\section{Introdução}
\todo[inline]{OBJETIVO: Apresentar uma visão geral desta parate do estudo que
consitirá de uma análise da funcionalidades oferecidas pelas FGRM}
 \todo[inline]{Definir conforme livro do wohlin(é assim que escreve??) o
	 conceito de estudo exploratório}


Esta etapa do trabalho consiste de um estudo
exploratório~\cite{wohlin2012experimentation} com o objetivo de determinar quais
são as funcionalidades comuns às Ferramentas de Gerenciamento de Requisição de
Mudança (FGRM). O estudo consistirá na leitura da documentação de alguns FGRM
para que de forma sistemática seja levantado quais são as funcionalidades
oferecidas pela ferramenta em análise. O método de escolha das FGRM será
avaliado posteriormente, todavia, um possível ponto de partida é a lista
disponível na Wikipédia que compara diversas
FGRM\footnote{
	\url{https://en.wikipedia.org/wiki/Comparison_of_issue-tracking_systems}}.
\section{Objetivo do Capítulo}
\label{sec:objetivo_do_capítulo}

Neste capítulo realizamos um estudo exploratório com o objetivo de determinar as
funcionalidades comuns as Ferramentas de Gerenciamento de Requisição de Mudanças
existentes no mercado. A partir deste conjunto de compartilhado de
funcionalidades é possível avaliar a contribuição das extensões que estão sendo
propostas na literatura, conforme discutido no
Capítulo\ref{ch:mapeamento-sistematico}.

Acreditamos que o resultado deste estudo permitirá compreender melhor este tipo
de ferramenta tomando como base as suas funcionalidades em comum. Também será
possível propor extensões para as FGRM tendo em vista a possibilidade de
determinar o conjunto mínimo de funções deste tipo de sistema. Uma outra
possível contribuição é a possibilidade de criação de um esquema de
caracterização com base nas funcionalidades oferecidas.

\section{Metodologia}
\label{sec:metodologia}

O processo de descoberta das funcionalidades comuns às FGRM é composto das
etapas descritas a seguir:

\begin{itemize}
	\item Seleção das Ferramentas
	\item Inspeção da Documentação
	\item Agrupamento das Funcionalidades
\end{itemize}

Cada uma etapas é explicada em detalhes nas próximas seções. 

\subsection{Seleção das Ferramentas}
\label{ssub:Seleção das Ferramentas}

A primeira etapa consistiu na definição das ferramentas que seriam utilizadas
neste estudo. Em uma inspeção inicial, verificamos a existência de mais de 50
FGRM disponíveis comercialmente ou de código
aberto\footnote{\url{https://en.wikipedia.org/wiki/Comparison_of_issue-tracking_systems}}.
Contudo, devido a dificuldade de realizar a análise de cada uma daquelas
ferramentas, optamos por escolher um subconjunto das ferramentas que fosse mais
representativas, segundo o nosso conhecimento na área. A representatividade
neste caso corresponde a notoriedade que a ferramenta possui dentro da
literatura (Ex. Bugzilla) ou mesmo na industria (Ex. Github).
\todo[inline]{Conforme sugestão do professor Rodolfo seria interessante avaliar
	quais das ferramentas têm real notoriedade com base na opinião de
	profissionais envolvidos em manutenção de software}

As FGRM selecionadas foram inicialmente classificadas como
\textit{"ferramentas"} e \textit{"serviços da internet"}. O primeiro grupo
representa os softwares capazes de serem implantados na infraestrutura do seu
cliente e do qual permite algum grau de personalização de alguns componentes,
como por exemplo o bando de dados utilizado. No segundo grupo estão os software
que ofertam a gerência das RM's mediante uma arquitetura do tipo Software como
Serviço (Software as Service), onde certos tipos de alterações no comportamento
do software por parte do usuário são mais restritas. As ferramentas selecionadas
são apresentadas por tipo de categoria na Tabela
\ref{tab:ferramentas_selecionadas} \todo[inline]{Incluir referência sobre
	Software como Serviço}

\begin{table}[ht]
	\centering
	\caption{Ferramentas e serviços da Internet selecionados.}
	\resizebox{\textwidth}{!}{%
		\begin{tabular}{llll}
			\hline
			\multicolumn{2}{c}{\textbf{Ferramentas}}           & \multicolumn{2}{c}{\textbf{Serviços da Internet}} \\ \hline
			Bugzilla & \url{https://www.bugzilla.org/}              & SourceForge    &
			\url{https://sourceforge.net/}    \\ \hline
			MantisBT & \url{https://www.mantisbt.org/}               & Lauchpad       &
			\url{https://launchpad.net/}      \\ \hline
			Trac     & \url{https://trac.edgewall.org/}              & Code Plex      &
			\url{https://www.codeplex.com/}   \\ \hline
			Redmine  & \url{www.redmine.org/}                        & GitLab			&
				\url{https://gitlab.com}          \\ \hline
			Jira     & \url{https://www.atlassian.com/software/jira} & GitHub         &
				\url{https://github.com/}         \\ \hline
		\end{tabular}%
	}
	\label{tab:ferramentas_selecionadas}
\end{table}

\subsection{Inspeção da Documentação}
\label{ssub:Inspeção da Documentação}

Neste etapa do trabalho realizamos a leitura do material disponível na internet
para cada uma das ferramentas. Entre estes materiais inclui manual do usuário,
manual do desenvolvedor, notas de lançamento e etc. Para cada uma das FGRM
optamos por analisar a última versão estável do software a fim de analisarmos o
que há de mais novo disponível aos usuários. A Tabela apresenta a versão
analisada e o elo de ligação para cada uma da documentação utilizada neste
estudo. Para aquelas ferramentas que apresentam documentação em mais de um
idioma optamos sempre por utilizar aquela escrita em inglês por entendermos ser
a que possivelmente mais atualizada.

\subsection{Agrupamento das Funcionalidades}
\label{ssub:Agrupamento das Funcionalidades}

Esta etapa tem por objetivo agrupar as funcionalidades que aparecem com
nomenclatura distintas em diferentes ferramentas, mas que apresentam o mesmo
significado. Cabe ressaltar que o agrupamento de algumas funcionalidades pode
depender de uma análise subjetiva de quem está realizando a atividade.  Neste
sentido, a fim de evitar algum tipo de viés o agrupamento foi realizado em duas
etapas:

\begin{description}
	\item[Análise Individual] Neste etapa o autor e um outro especialista
		realizam de forma separada os agrupamentos que acharem necessários
	\item[Anaĺise Compartilhada] Em um segundo momento tanto o auto quanto o
		especialista discutem as possíveis divergências até que um consenso seja
		obtido.	\end{description}

Após as duas etapas descritas anteriormente chegamos ao conjunto de
funcionalidades descritos na Tabela. A coluna agrupamento exibe o nome comum
para representas as funções descritas na coluna "nomenclatura original"


\section{Resultados}
\label{sec:resultados}



\section{Discussão}
\label{sec:discussao}

\section{Ameças à Validade}
\label{sec:ameacas_a_validade}


\section{Resumo do Capítulo}
\label{sec:resumo_do_capitulo}