%%%%%%%%%%%%%%%%%%%%%%%%%%%%%%%%%%%%%%%%%%%%%%%%%%%%%%%%%%%%%%%%%%%%%%%%%%%%%%%
%Objetivo: Caracterizas as Ferramentas de Gerenciamento de Requisição de Mudança
%		   com base nas suas funcionalidades
%Autor: Vagner Clementino <vagnercs@dcc.ufmg.br>
% 		Rodolfo Resende	<rodolfo@dcc.ufmg.br>
%Criação: qua set 28 11:25:03 BRT 2016
%Modificação: qua set 28 11:25:35 BRT 2016
%Revisão: Ter Out 11 19:17:27 BRT 2016
%%%%%%%%%%%%%%%%%%%%%%%%%%%%%%%%%%%%%%%%%%%%%%%%%%%%%%%%%%%%%%%%%%%%%%%%%%%%%%%
\chapter{Caracterização das Ferramentas de Gerenciamento de Requisição de
	Mudança}
\label{ch:caracterizacao}


\section{Introdução}
\todo[inline]{OBJETIVO:\@ Apresentar uma visão geral desta parte da dissertação
que consistirá de uma análise da funcionalidades oferecidas pelas FGRM}

Quando uma organização ou projeto software de código aberto decide adotar uma
Ferramenta de Gerenciamento de Requisições de Mudança um desafio é encontrar
aquela que melhor atenda suas necessidades. Um possível critério de seleção pode
ser as funcionalidades oferecidas pelo software. Outros critérios podem envolver
o custo bem como o suporte a falhas da ferramenta. De maneira relacionada, o
pesquisador que tenha o interesse em propor melhorias para as FGRM's também está
interessado em analisar o conjunto de funções comuns que caracterizam as FGRM's.

O numero de FGRM's disponíveis atualmente é bastante elevado. Em uma inspeção
inicial, verificamos a existência de mais de 50 ferramentas fornecidas
comercialmente ou de código
aberto\footnote{\url{https://en.wikipedia.org/wiki/Comparison_of_issue-tracking_systems}}.
Apesar das diferentes opções disponíveis, ao bem do nosso conhecimento,
desconhecemos estudos que avaliem de forma sistemática as funcionalidades
oferecidas por este tipo de software a fim de compará-las. Entendemos que a
partir de um conjunto compartilhado de funcionalidades seja possível
caracterizar as FGRM's ao mesmo tempo que possibilita avaliar a contribuição das
extensões que estão sendo propostas na literatura, conforme discutido no
Capítulo\ref{ch:mapeamento-sistematico}.  Para alcançarmos este objetivo
realizamos um estudo exploratório visando determinar quais são as
funcionalidades comuns às Ferramentas de Gerenciamento de Requisição de Mudança
(FGRM). Um estudo exploratório está preocupado na análise de um objeto em sua
configuração natural e deixa as descobertas surgirem da própria
observação~\cite{wohlin2012experimentation}. Neste tipo de estudo nenhuma
hipótese é previamente definida.

O trabalho descrito neste capítulo consistiu na leitura da documentação de
algumas FGRM de modo a sistematizar as funcionalidades oferecidas por cada
ferramenta.  As funções foram coletadas e organizadas utilizando a técnica de
Cartões Ordenados (Sorting Cards)~\cite{5070993, rugg2005sorting}. Devido ao
alto volume de ferramentas disponíveis e ao grande esforço de analisar a
documentação de todas elas, optamos por realizar este estudo com um conjunto
mínimo que foi escolhido com a ajuda de profissionais envolvidos em manutenção
de software.  Através de uma pesquisa (survey) os profissionais foram
questionados a definir dentre as ferramentas apresentadas eram as mais
representativas dentro do domínio de aplicação das FGRM's. A representativa
neste contexto não está no número de projetos que utiliza determinada
ferramenta, mas pelas caraterísticas que determinado software possui que o torna
diferenciável dentro do seu domínio.

\section{Objetivo do Capítulo}
\label{sec:objetivo_do_capítulo}

O objetivo inicial deste capítulo é determinar as funcionalidades comuns às
Ferramentas de Gerenciamento de Requisição de Mudanças tomando como ponto de
partida um conjunto mínimo de sistemas que foram definidas como as mais
relevantes por profissionais envolvidos em manutenção e desenvolvimento de
software. Em um segundo momento, o foco é caracterizar este tipo de ferramenta
tomando como base as funcionalidades oferecidas pelos softwares. Conforme já
exposto, a literatura em Manutenção de Software apresenta diferentes
nomenclaturas para este tipo de ferramenta (Sistema de Controle de Defeito~-~Bug
Tracking Systems, Sistema de Gerenciamento da Requisição~-~Request Management
System, Sistemas de Controle de Demandas (SCD)~-~Issue Tracking Systems), sem,
contudo, se preocupar em diferenciá-las.

Acreditamos que o resultado deste estudo permitirá compreender melhor este tipo
de software tomando como base o conjunto de funções que eles oferecem aos seus
usuários. Também será possível propor novas funções ou melhorias das já
existentes tendo em vista a possibilidade de determinar o conjunto mínimo de
funcionalidades deste tipo de sistema. Uma outra possível contribuição é a
criação de um esquema de caracterização com base nas funcionalidades oferecidas.

\section{Metodologia}
\label{sec:metodologia}

A fim de determinarmos o conjunto comum de funcionalidades das FGRM's realizamos
um estudo exploratório dividido em três etapas listadas a seguir. O resultado
obtido em determinada etapa fui utilizado para subsidiar as atividades de uma
etapa posterior. Antes do início de um nova fase de trabalho o resultado foi
avaliado afim de verificar possíveis inconsistências. Cada uma das partes deste
estudo é explicada em detalhes nas próximas seções.

\begin{enumerate}[(i)]
	\item Seleção das Ferramentas
	\item Inspeção da Documentação
	\item Agrupamento das Funcionalidades
\end{enumerate}

\subsection{Seleção das Ferramentas}
\label{subsec:selecao-ferramentas}

A primeira etapa consistiu na definição das ferramentas que seriam utilizadas no
estudo. A partir de uma pesquisa na Internet obtivemos um conjunto inicial de 50
ferramentas
utilizadas\footnote{\url{https://en.wikipedia.org/wiki/Comparison_of_issue-tracking_systems}}.
O Anexo exibe o conjunto de ferramentas levantadas para este estudo.
\todo[inline]{Incluir anexo com as ferramentas utilizadas inicialmente}

Devido ao pouco tempo disponível e a dificuldade de realizar a análise em cada
uma daquelas ferramentas, optamos por escolher um subconjunto de sistemas que
fossem mais representativos, tomando como base a opinião de profissionais
envolvidos em manutenção e desenvolvimento de software. A representatividade
neste caso corresponde a opinião do profissional sobre notoriedade que a
ferramenta possui dentro do seu domínio de aplicação em comparação com as demais
que lhe foram apresentadas.

A opinião dos profissionais foi obtida mediante a realização de uma pesquisa
(survey~\cite{wohlin2012experimentation}) através de um formulário eletrônico.
O formulário foi estruturado em duas partes principais: a formação de base do
participante (background) e a avaliação das ferramentas. Na primeira parte
estávamos interessados em conhecer as características do respondente. Esta
informação é relevante tendo em vista, como descreveremos a seguir, o
questionário foi replicado em três grupos distintos de profissionais. Neste
sentido, se torna possível realizar análises sobre como é formado cada um dos
grupos de profissionais utilizados neste estudo. Na segunda parte da pesquisa
apresentamos as ferramentas e foi pedido que o profissional avaliasse a
relevância de cada uma através de questões de múltipla escolha utiliza uma
escala do tipo Likert~\cite{robbins2011plotting}.

Antes da efetiva aplicação do questionário no público-alvo do estudo, o
documento foi validade em um processo de três etapas.Em um primeiro momento foi
solicitado a dois pesquisadores experientes da área de Engenharia de Software
que avaliassem o formulário inicialmente desenhado. A partir das sugestões
obtidas dos pesquisadores foram realizadas adequações no formulário. Após as
alteração a nova versão do documento foi enviado para dois profissionais
envolvidos diretamente em manutenção de software. O critério utilizado para
seleção dos profissionais foi o tempo dedicado à tarefa de manter software, que
no caso dos profissionais escolhidos era em média de 10 anos. O formulário foi
modificado com as sugestões dos profissionais finalizando desta forma a segunda
etapa de validação. A última etapa consistiu na realização de um piloto com dez
profissionais envolvidos em manutenção de uma empresa pública de informática. Os
profissionais tiveram que efetivamente preencher o questionário, contudo, foram
adicionadas questões as quais era possível inserir sugestões de melhoria. O
resultado deste processo de validação é o questionário presente no Anexo. Como o
público-alvo do questionário poderia incluir desenvolvedores de diferentes
nacionalidades foi construído em uma versão em língua inglesa do formulário que
consta no Anexo deste documento.

\todo[inline]{Incluir anexo com o formulário final em português}
\todo[inline]{Incluir anexo com o formulário final em inglês}

A população de interesse deste survey é o conjunto de profissionais envolvidos
em manutenção de software. Naturalmente é difícil definir o tamanho e
características desta população de modo a mensurar uma amostra significativa.
Neste sentido, visando minimizar possíveis víeis deste estudo, o questionário
foi replicado em três grupos distintos:

\begin{description}
	\item[Grupo 01:] Profissionais de empresa pública e privada de
			desenvolvimento e manutenção de software.
	\item[Grupo 02:] Profissionais que contribuem em projetos de
		código aberto
   	\item[Grupo 03:] Profissionais que participam de grupos de
		interesse em aplicativos de comunicação em celular.
\end{description}

%We chose these projects because (1)
%they are active, real-world projects with a sizable code base, and
%large numbers of users and developers,  and (2) bug reports from
%these projects have been used previously in several studies involv-
%ing bug report analyses [3, 13, 2, 4]. 
Os participantes que compõe cada grupo foram escolhidos conforme critérios que
são detalhados no Capítulo~\ref{ch:pesquisa-profissionais}. A reutilização desta
base de desenvolvedores se deu por conta de ambas as pesquisas terem a mesma
população como público-alvo, podendo neste caso compartilhar a mesma amostra.
Ademais, como o survey descrito nesta seção foi realizada antes daquele descrito
no Capítulo~ref{ch:pesquisa-profissionais}, as lições aprendidas no primeiro
serviram para melhorar o processo de desenho e aplicação do segundo.

Com base nos dados obtidos da pesquisa com os profissionais, as FGRM's foram
classificadas como \textit{''ferramentas''} e \textit{''serviços da internet''}.
O primeiro grupo representa os softwares que são capazes de serem implantados na
infraestrutura do seu cliente e permite algum grau de personalização de pelo
menos um dos componentes, como por exemplo, o banco de dados utilizado. No
segundo grupo estão os software que ofertam a gerência das RM's mediante uma
arquitetura do tipo Software como Serviço (Software as
Service)~\cite{fox2013engineering}, onde certos tipos de alterações no
comportamento do software são mais restritas. Acreditamos que ao escolher
ferramentas dos dois tipos tipos descritos anteriormente iremos cobrir uma
grande parte do domínio de aplicação das FGRM's. Tendo em vista o tempo
disponível e esforço necessário para análise da documentação de cada uma das
ferramentas ficou definido que seriam escolhidas 06 ferramentas para o estudo,
sendo três de cada um dos grupos. Neste sentido, serão escolhidas as três
ferramentas mais relevantes com base nos dados da pesquisa com os profissionais.

\subsection{Inspeção da Documentação}
\label{subsec:inspecao_doumentacao}

Nesta etapa do trabalho realizamos a leitura do material disponível na internet
para cada uma das ferramentas selecionadas. Entre estes materiais utilizamos
manual do usuário, manual do desenvolvedor, notas de lançamento e etc. Para cada
uma das FGRM's optamos por analisar a última versão estável do software a fim de
analisarmos o que há de mais novo disponível aos usuários. A Tabela apresenta a
versão analisada e o elo de ligação para cada uma da documentação utilizada
neste estudo. Para aquelas ferramentas que apresentam documentação em mais de um
idioma optamos sempre por utilizar aquela escrita em inglês por entendermos ser
a que possivelmente mais atualizada.

\todo[inline]{Incluir tabela com os link de acesso às ferramentas}

Os dados obtidos da leitura do material dísponíveis para cada ferramenta foram
inseridos em cartões
\subsection{Agrupamento das Funcionalidades}
\label{subsec:agrupamento_fucionalidades}

Esta etapa tem por objetivo agrupar as funcionalidades que aparecem com
nomenclatura distintas em diferentes ferramentas, mas que apresentam o mesmo
significado. Cabe ressaltar que o agrupamento de algumas funcionalidades pode
depender de uma análise subjetiva de quem está realizando a atividade.  Neste
sentido, a fim de evitar algum tipo de viés o agrupamento foi realizado em duas
etapas:

\begin{description}
	\item[Análise Individual] Neste etapa o autor e um outro especialista
		realizam de forma separada os agrupamentos que acharem necessários
	\item[Anaĺise Compartilhada] Em um segundo momento tanto o auto quanto o
		especialista discutem as possíveis divergências até que um consenso seja
		obtido.
\end{description}

Após as duas etapas descritas anteriormente chegamos ao conjunto de
funcionalidades descritos na Tabela. A coluna agrupamento exibe o nome comum
para representas as funções descritas na coluna ``nomenclatura original''.



\section{Resultados}
\label{sec:resultados}



\section{Discussão}
\label{sec:discussao}

\section{Ameças à Validade}
\label{sec:ameacas_a_validade}


\section{Resumo do Capítulo}
\label{sec:resumo_do_capitulo}