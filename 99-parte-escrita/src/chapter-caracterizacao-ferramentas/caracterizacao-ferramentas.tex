%%%%%%%%%%%%%%%%%%%%%%%%%%%%%%%%%%%%%%%%%%%%%%%%%%%%%%%%%%%%%%%%%%%%%%%%%%%%%%%
%Objetivo: Caracterizas as Ferramentas de Gerenciamento de Requisição de Mudança
%		   com base nas suas funcionalidades
%Autor: Vagner Clementino <vagnercs@dcc.ufmg.br>
% 		Rodolfo Resende	<rodolfo@dcc.ufmg.br>
%Criação: qua set 28 11:25:03 BRT 2016
%Modificação: qua set 28 11:25:35 BRT 2016
%Revisão: qua set 28 11:25:27 BRT 2016
%%%%%%%%%%%%%%%%%%%%%%%%%%%%%%%%%%%%%%%%%%%%%%%%%%%%%%%%%%%%%%%%%%%%%%%%%%%%%%%
\chapter{Caracterização das Ferramentas de Gerenciamento de Requisição de
	Mudança}
\label{ch:caracterizacao}


\section{Introdução}
\todo[inline]{OBJETIVO: Apresentar uma visão geral desta parte do estudo que
consistirá de uma análise da funcionalidades oferecidas pelas FGRM}
Quando um organização ou mesmo um projeto de código aberto decide adotar uma
Ferramenta de Gerenciamento de Requisições um desafio é encontrar aquela que
melhor atende as suas necessidades. Um bom ponto de partida é analisar as
funcionalidades oferecidas pelas ferramentas de interesse. De maneira
relacionada, o pesquisador que tenha interesse em propor melhorias para este
tipo de ferramenta precisa ter acesso ao conjunto básico de funções que compõe
as FGRM's.
Atualmente estão disponíveis um grande número de FGRM. Em uma inspeção inicial,
verificamos a existência de mais de 50 FGRM disponíveis comercialmente ou de
código
aberto\footnote{\url{https://en.wikipedia.org/wiki/Comparison_of_issue-tracking_systems}}.
Apesar das diversas opções disponíveis, ao bem do nosso conhecimento,
desconhecemos alguns estudo que avaliasse de forma sistemática as
funcionalidades oferecidas por este tipo de ferramenta. A partir deste conjunto
de compartilhado de funcionalidades é possível avaliar a contribuição das
extensões que estão sendo propostas na literatura, conforme discutido no
Capítulo\ref{ch:mapeamento-sistematico}.
Para alcançarmos este objetivo realizamos um estudo exploratório com o objetivo
de determinar quais são as funcionalidades comuns às Ferramentas de
Gerenciamento de Requisição de Mudança (FGRM). Um estudo exploratório está
preocupado em estudo um objeto em sua configuração natural e deixa as
descobertas surgirem da observação. Neste tipo de estudo nenhuma hipótese é
definida previamente~\cite{wohlin2012experimentation}.

O estudo consistiu na leitura da documentação de algumas FGRM de modo a obter as
funcionalidades oferecidas por cada ferramenta. As funções foram coletadas e
organizadas utilizando a técnica de Cartões Ordenados (Sorting
Cards)~\cite{5070993}. As FGRM utilizadas neste estudo foram escolhidas mediante
uma pesquisa com profissionais (survey) no qual os participantes foram
questionados a decidir quais ferramentas eram as mais representativas.


\section{Objetivo do Capítulo}
\label{sec:objetivo_do_capítulo}

O primeiro objetivo deste capítulo é determinar as funcionalidades comuns as
Ferramentas de Gerenciamento de Requisição de Mudanças dentre aqueles definidas
como mais relevantes por profissionais envolvidos em manutenção e
desenvolvimento de software. Em um segundo momento, o foco é caracterizar este
tipo de ferramenta tomando como base as funções oferecidas.  Conforme já exposto
a literatura apresenta diferentes nomenclaturas para este tipo de ferramenta
(Sistema de Controle de Defeito - Bug Tracking Systems, Sistema de Gerenciamento
da Requisição - Request Management System, Sistemas de Controle de Demandas
(SCD)- Issue Tracking Systems), sem, contudo, se preocupar em diferenciá-las.

Acreditamos que o resultado deste estudo permitirá compreender melhor este tipo
de ferramenta tomando como base as suas funcionalidades. Também será possível
propor extensões para as FGRM tendo em vista a possibilidade de determinar o
conjunto mínimo de funções deste tipo de sistema. Uma outra possível
contribuição é a possibilidade de criação de um esquema de caracterização com
base nas funcionalidades oferecidas.

\section{Metodologia}
\label{sec:metodologia}

O processo de descoberta das funcionalidades comuns às FGRM é composto das
etapas descritas a seguir:

\begin{itemize}
	\item Seleção das Ferramentas
	\item Inspeção da Documentação
	\item Agrupamento das Funcionalidades
\end{itemize}

Cada uma etapas é explicada em detalhes nas próximas seções. 

\subsection{Seleção das Ferramentas}
\label{ssub:Seleção das Ferramentas}

A primeira etapa consistiu na definição das ferramentas que seriam utilizadas
neste estudo. Contudo, devido a dificuldade de realizar a análise de cada uma daquelas
ferramentas, optamos por escolher um subconjunto das ferramentas que fosse mais
representativas, segundo o nosso conhecimento na área. A representatividade
neste caso corresponde a notoriedade que a ferramenta possui dentro da
literatura (Ex. Bugzilla) ou mesmo na industria (Ex. Github).
\todo[inline]{Conforme sugestão do professor Rodolfo seria interessante avaliar
	quais das ferramentas têm real notoriedade com base na opinião de
	profissionais envolvidos em manutenção de software}

As FGRM selecionadas foram inicialmente classificadas como
\textit{"ferramentas"} e \textit{"serviços da internet"}. O primeiro grupo
representa os softwares capazes de serem implantados na infraestrutura do seu
cliente e do qual permite algum grau de personalização de alguns componentes,
como por exemplo o bando de dados utilizado. No segundo grupo estão os software
que ofertam a gerência das RM's mediante uma arquitetura do tipo Software como
Serviço (Software as Service), onde certos tipos de alterações no comportamento
do software por parte do usuário são mais restritas. As ferramentas selecionadas
são apresentadas por tipo de categoria na Tabela
\ref{tab:ferramentas_selecionadas} \todo[inline]{Incluir referência sobre
	Software como Serviço}

\subsection{Inspeção da Documentação}
\label{ssub:Inspeção da Documentação}

Nesta etapa do trabalho realizamos a leitura do material disponível na internet
para cada uma das ferramentas selecionadas. Entre estes materiais utilizamos
manual do usuário, manual do desenvolvedor, notas de lançamento e etc. Para cada
uma das FGRM optamos por analisar a última versão estável do software a fim de
analisarmos o que há de mais novo disponível aos usuários. A Tabela apresenta a
versão analisada e o elo de ligação para cada uma da documentação utilizada
neste estudo. Para aquelas ferramentas que apresentam documentação em mais de um
idioma optamos sempre por utilizar aquela escrita em inglês por entendermos ser
a que possivelmente mais atualizada.

\subsection{Agrupamento das Funcionalidades}
\label{ssub:Agrupamento das Funcionalidades}

Esta etapa tem por objetivo agrupar as funcionalidades que aparecem com
nomenclatura distintas em diferentes ferramentas, mas que apresentam o mesmo
significado. Cabe ressaltar que o agrupamento de algumas funcionalidades pode
depender de uma análise subjetiva de quem está realizando a atividade.  Neste
sentido, a fim de evitar algum tipo de viés o agrupamento foi realizado em duas
etapas:

\begin{description}
	\item[Análise Individual] Neste etapa o autor e um outro especialista
		realizam de forma separada os agrupamentos que acharem necessários
	\item[Anaĺise Compartilhada] Em um segundo momento tanto o auto quanto o
		especialista discutem as possíveis divergências até que um consenso seja
		obtido.
\end{description}

Após as duas etapas descritas anteriormente chegamos ao conjunto de
funcionalidades descritos na Tabela. A coluna agrupamento exibe o nome comum
para representas as funções descritas na coluna "nomenclatura original"


\section{Resultados}
\label{sec:resultados}



\section{Discussão}
\label{sec:discussao}

\section{Ameças à Validade}
\label{sec:ameacas_a_validade}


\section{Resumo do Capítulo}
\label{sec:resumo_do_capitulo}