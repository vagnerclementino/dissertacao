\chapter{Caracterização das Ferramentas de Gerenciamento de Requisição de Mudança}
\label{ch:caracterizacao}


\section{Introdução}

Esta etapa do trabalho consistirá de um estudo exploratório com o objetivo de determinar quais são as funcionalidades comuns às Ferramentas de Gerenciamento de Requisição de Mudança (FGRM). O estudo consistirá na leitura da documentação de alguns FGRM para que de forma sistemática seja levantado quais são as funcionalidades oferecidas pela ferramenta em análise. O método de escolha das FGRM será avaliado posteriormente, todavia, um possível ponto de partida é a lista disponível na Wikipédia que compara diversas FGRM\footnote{\url{https://en.wikipedia.org/wiki/Comparison_of_issue-tracking_systems}}.

O resultado deste estudo permitirá compreender melhor este tipo de ferramenta tomando como base as suas funcionalidades em comum. Também será possível propor extensões para as FGRM (Seção ref{ch:extensoes}) tendo em vista a possibilidade de determinar o conjunto mínimo de funções deste tipo de sistema. Uma outra possível contribuição é desenvolver uma taxonomia deste tipo de ferramenta com base nas funcionalidades oferecidas.

\section{Discussão}

\section{Ameças à Validade}

\section{Replicação da Pesquisa com Profissionais}

\section{Resumo do Capítulo}