%%%%%%%%%%%%%%%%%%%%%%%%%%%%%%%%%%%%%%%%%%%%%%%%%%%%%%%%%%%%%%%%%%%%%%%%%%%%%%%%
%Objetivo: Descrever a implementação de uma extensão para uma FGRM de modo a
%avaliar o impacto este tipo de melhoria pode causar neste tipo de ferramenta.
%Autores: Vagner Clementino <vagnercs@dcc.ufmg.br>
%		  Rodolfo Resende <rodolfo@dcc.ufmg.br>
%Criação: dom fev 26 12:49:27 BRT 2017
%Modificação: qui mar  9 20:55:46 BRT 2017
%Revisão: dom mar 12 09:35:34 BRT 2017
%%%%%%%%%%%%%%%%%%%%%%%%%%%%%%%%%%%%%%%%%%%%%%%%%%%%%%%%%%%%%%%%%%%%%%%%%%%%%%%%
\chapter{Um Estudo sobre a Implementação de uma Extensão para uma FGRM}
\label{ch:implemtacao_extensao}

\section{Introdução}
\label{sec:implemtacao_extensao_intro}

Durante esta dissertação estamos discutindo que as funcionalidades oferecidas
pelas Ferramentas de Gerenciamento de Requisições de Mudança \@-\@ FGRM's
conseguem atender aos objetivos deste tipo de software. Todavia, verificamos que
exite espaço para melhorias das funções já existentes ou mesmo a proposição de
novas. O desenvolvimento de novas funcionalidades em FGRM's, mediante a
capacidade de extensão propiciada por algumas delas, vem sendo explorada na
literatura. A extensão
\textit{Buglocalizer}~\cite{Thung:2014:BIT:2635868.2661678}, criada para a
ferramenta Bugzilla, possibilita a localização dos arquivos do código fonte que
estão relacionados ao defeito relatado. A ferramenta extrai texto dos campos de
sumário e descrição da RM\@. Este texto é comparado com o código fonte por meio
de técnicas de Recuperação da Informação.

Na mesma linha, o \textit{NextBug}~\cite{101186} é uma extensão para o Bugzilla
que recomenda novas RMs para o desenvolvedor baseado naquela em que ele esteja
tratando atualmente. O objetivo da extensão é sugerir defeitos com base em
técnicas de Recuperação de Informação. Na ferramenta proposta por Thung e
outros~\cite{Thung:2014:DIT:2642937.2648627} o foco é na determinação de
defeitos duplicados. A contribuição deste trabalho é a integração do estado da
arte de técnicas não supervisionadas para detecção de falhas duplicadas conforme
proposto por Runeson e outros.

Esta dissertação também se propôs em contribuir com a melhoria das
funcionalidades das FGRM's mediante a apresentação e discussão de um conjunto de
recomendações conforme descrito no Capitulo~\ref{ch:sug_melhoria}. Apesar de ter
sido conduzido um processo de avaliação daquilo que foi proposto, cujo resultado
demonstrou uma boa aceitação dos participantes, optamos por analisar o impacto
da implementação do que foi proposto em determinada FGRM\@.

Conforme discutido, a Seção~\ref{sec:sug_melhoria_melhorando_as_ferraementas}
apresenta um conjunto de $N$ sugestões de melhorias das funcionalidades.
Idealmente gostaríamos de transformar todas as sugestões em extensões de
funcionalidades para as FGRM's. Não há razões que justifiquem a priorização de
implementação de uma recomendação sobre outra. Entretanto, após alguns ensaios e
combinando de maneira mais intuitiva do que seguindo um fluxo de critérios,  foi
investido mais esforço no desenvolvimento de uma extensão para o suporte à
qualidade de relato.

\section{Qualidade do Relato de uma RM}
\label{sec:avaliando_a_qualidade_do_relato_de_uma_rm}

No estudo realizado por Bettenburg e outros~\cite{bettenburg2008makes} foi
desenvolvida um levantamento com questionário (\textit{survey}) entre
desenvolvedores e usuários dos projetos
Apache\footnote{\url{http://www.apache.org/}},
Eclipse\footnote{\url{https://www.eclipse.org}} e
Mozilla\footnote{\url{https://www.mozilla.org}} a fim de verificar o que
produziria um bom relato de um RM\@. Os resultados demonstraram que do ponto de
vista dos desenvolvedores eram consideradas informações úteis, que idealmente
deveriam estar no relato de uma RM\@: \textit{(i)} a sequência de erros
executadas até o aparecimento do erro (se for o caso), também conhecida como
\textit{etapas para reproduzir}; \textit{(ii)} o registro de pilhas de ativação
(stack traces) que são arquivos com os histórico de chamada de métodos (logs)
que ocorreram antes da ocorrência do erro.

No estudo proposto por Zimmermann e outros~\cite{5070993} é discutido a
importância de que a informação descrita em uma RM seja relevante e completa a
fim de que esteja relatado seja resolvido rapidamente. Contudo, na prática, a
informação apenas chega ao desenvolvedor com a qualidade requerida após diversas
interações com o usuário afetado. Com o objetivo de minimizar este problema os
autores propõe um conjunto de diretrizes para a construção de uma extensão capaz
de reunir informações relevantes a partir do usuário além de identificar
arquivos que precisam ser corrigidos para resolver o defeito.

\section{Uma Extensão para Suporte da Qualidade do Relato}
\label{sec:uma_extensao_suporte_qualidade_relato}


\begin{description}
	\item[Questão 01:] Qual impacto da inclusão de uma extensão para o suporte à
		qualidade do relato pode ter no tempo necessário para análise de uma
		RM\@?
	\item[Questão 02:] Existe relação entre a frequência que um participante
		cria uma RM e a qualidade do relato?
	\item[Questão 03:] Do ponto de vista dos profissionais envolvidos em
		manutenção de software qual o impacto da inclusão de uma extensão para o
		suporte à qualidade do relato no processo de manutenção de software?
\end{description}

Na \textit{Questão 01} estamos interessados em verificar se a inclusão de uma
extensão deste tipo pode atrasar o processo de resolução de uma RM por conta da
eventual sobrecarga que a análise da qualidade do relato pode causar. A
\textit{Questão 02} possui o foco em avaliar se aquelas pessoas que criam RM com
maior frequência em determinado projeto possuem uma qualidade do relato superior
daqueles que fazem isso eventualmente. Por fim, na \textit{Questão 03} queremos
entender os prós e contras que a implantação deste tipo de extensão pode
produzir no processo de manutenção de software tomando com base a opinião de
profissionais da área.

\section{Avaliando a Extensão Proposta}
\label{sec:avaliando_a_extensao_proposta}

\subsection{Desenho da Avaliação}
\label{sub:implementacao_extenscao_desenho_da_avaliacao}

\subsection{Resultados}
\label{sub:implementacao_extensao_avaliacao_resultados}

\subsection{Discussão}
\label{sub:implemtacao_extensao_avaliacao_discussao}

\section{Limitações e Ameças à Validade}
\label{sec:limitações_e_ameças_à_validade}

\section{Conclusões}
\label{sec:conclusões}

\section{Resumo do Capítulo}
\label{sec:implemtacao_extensao_resumo}
