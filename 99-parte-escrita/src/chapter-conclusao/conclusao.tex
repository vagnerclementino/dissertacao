\chapter{Conclusão}
\label{ch:conclusao_trab_futuros}

A Manutenção de Software é um processo complexo e caro e, portanto, merece
atenção da comunidade científica e da indústria. Neste contexto, surge a
necessidade do desenvolvimento de técnicas, processos e ferramentas que reduzam
o custo e o esforço envolvidos nas atividades de manutenção e evolução de
software. Conforme discutido, as Ferramentas de Gerenciamento de Requisição de
Mudança desempenham um papel fundamental que ultrapassa a função de registrar as
falhas e pedidos de melhoria dos softwares. Este estudo se propôs em avaliar as
funcionalidades das FGRMs com o objetivo de propor melhorias.

Com base no estudo descrito na Seção~\ref{sec:caracterizacao_ferramentas}
verificamos que as FGRMs dispõem de funcionalidades para gerenciar a criação,
consulta, atualização e destruição de uma RM\@. Entretanto, em algumas
plataformas, tais como o Github e o Gitlab, foi possível perceber a tendência em
que não existe uma clara separação entre o gerenciamento das RMs e o controle de
versão do código. Um possível desdobramento desta dissertação é avaliar o
impacto deste tipo de abordagem no processo de manutenção de software. Este tipo
de análise poderia modificar o desenvolvimento de futuras versões das FGRMs,
especialmente para aqueles que estão acopladas à repositórios de software.

No Capítulo~\ref{ch:mapeamento-sistematico}, ao revisarmos a literatura sobre
melhorias nas FGRMs, recuperamos estudos que discutem diversos aspectos dos
problemas e desafios do gerenciamento das RMs. A maior frequência de trabalhos
está relacionado com temas como atribuição automática e RMs duplicadas. Conforme
discutimos, alguns autores afirmam que, do ponto de vista dos desenvolvedores, a
duplicação dos pedidos de manutenção não seria o principal problema a ser
tratado. Em contrapartida, os profissionais estariam mais interessados em
propostas que melhorem a qualidade do relato na RM\@. Este é um exemplo do
desacoplamento entre as necessidades dos desenvolvedores e o que está sendo
proposto na literatura. Apesar da qualidade dos estudos sobre melhorias das
funções das FGRMs, tais avanços aparentemente não estão disponíveis ao time de
manutenção. Com base nos resultados da amostra do levantamento realizado no
Capítulo~\ref{ch:pesquisa-profissionais} percebemos que os profissionais
consultados estão satisfeitos com as funcionalidades oferecidas. Todavia, a
nossa visão é que existem muitos outros comportamentos que poderiam ser
acoplados a este tipo de software de modo a melhorar as atividades de manter e
evoluir um software. Da mesma forma, as metodologias propostas pelos agilistas
vêm sendo adotadas por algumas equipes de manutenção de software. Neste
contexto, as FGRMs podem implantar funcionalidades de modo a suportar algumas
destas práticas.

De maneira relacionada, em uma das classificações feitas no Mapeamento
conduzido, os estudos foram agrupados pelo tipo de papel desempenhado na
Manutenção de Software o qual daria suporte. Utilizamos uma classificação
proposta por Polo e outros~\cite{Polo1999} construída em 1999. Em nossas
pesquisas não identificamos uma discussão mais recente sobre os papéis
desempenhados na Manutenção de Software. Entendemos que seria importante a
condução de um novo trabalho com o objetivo de descrever e avaliar os papéis
realizados no processo de manter um software. Este estudo poderia avaliar como
as práticas propostas pelos agilistas podem ter alterado a estrutura de trabalho
das equipes de manutenção.

O nível de satisfação dos profissionais com as funcionalidades das FGRMs pode
ser considerado alto. Esta percepção foi obtida mediante um levantamento por
questionário apresentado no Capítulo~\ref{ch:pesquisa-profissionais}.
Entretanto, o mesmo estudo demonstrou que os participantes desconhecem o
potencial deste tipo de software. No questionário, ao apresentarmos propostas de
melhorias que eram discutidas na literatura o grau de aceitação foi elevado.
Neste sentido, observamos um distanciamento entre o estado da arte e o estado da
prática das melhorias para as FGRMs. Apesar deste distanciamento ser algo
esperado, por diversas razões, estudos devem ser conduzidos afim de diminuir
estas diferenças. Entendemos que esta dissertação contribuiu neste sentido ao
apresentar para os profissionais algumas das melhorias discutidas na literatura.
Este conhecimento pode ser utilizado pelos desenvolvedores para exigir
ferramentas que atendam às suas demandas. Ao mesmo tempo, o nosso trabalho
conseguiu coletar, discutir e apresentar algumas da necessidades dos
desenvolvedores. Da mesma forma, pesquisadores podem usar esta informação para
propor estudos com o intuito de propor melhorias para as FGRMs.

Esta dissertação contribuiu para o estudo das FGRMs com a apresentação de um
conjunto de melhorias no Capítulo~\ref{ch:sug_melhoria}. Considerando a sua boa
aceitação, estas recomendações podem ser utilizadas por pesquisadores, pelos
responsáveis por desenvolver e manter projetos de FGRMs e por os profissionais
envolvidos com Manutenção de Software. Um possível desdobramento desta
dissertação é a avaliação do impacto da implantação destas sugestões. Esta
análise poderia ser realizada mediante um Estudo de Caso, por exemplo.

Na elaboração do modelo conceitual do contexto das FGRMs, correspondente à
Figura~\ref{fig:diagrama-classe-conceitual-fgrm}, verificamos que o tratamento
da ambiguidade e inconsistência, entre os diversos artigos que tratam do
assunto, exigia maior esforço do que foi possível investir. Neste sentido,
entendemos que seria importante a realização de um estudo com o objetivo de
melhorar a organização dos conceitos da área de Manutenção de Software, em
especial sobre as RMs e FGRMs.

No Capítulo~\ref{ch:implemtacao_extensao} implementamos uma das sugestões como
uma extensão para o módulo de \textit{issues} da plataforma Github. Por ser
tratar de uma Prova de Conceito, não foi realizada uma avaliação com membros dos
projetos selecionados. Por esta razão não é possível extrapolar os resultados.
Gostaríamos de futuramente conduzir um trabalho em uma configuração em que seja
possível avaliar a opinião de alguém envolvido no projeto estudado. Entendemos
que seria importante conduzir um estudo com os mesmo objetivos desta
dissertação, entretanto, coletando a opinião daqueles que reportam as RMs.
Gostaríamos de futuramente conduzir um trabalho em uma configuração em que seja
possível avaliar a opinião de alguém envolvido no projeto estudado.

Conforme apresentado na Seção~\ref{sec:caracterizacao_ferramentas} as FGRMs
disponibilizam diversos métodos para o registro de uma RM\@: envio de e-mail,
formulários que podem ser disponibilizados em sítios da Web e etc. Entretanto, a
maneira mais comum é o formulário disponibilizado pelas FGRMs, conforme exibido
na Figura~\ref{fig:rm-exemplo}. Segundo o nosso entendimento, o processo de
criação de RMs poderia ser melhorado com a utilização de uma interface que
utilize um \textit{chatbot}~\cite{mauldin1994chatterbots,huang2007extracting}.
Esta maneira interativa de criação de uma RM, dentre outros aspectos, pode
ajudar no fornecimento das informações necessárias para a correspondente
solução.
