\chapter{Conclusão}
\label{ch:conclusao_trab_futuros}

A Manutenção de Software é um processo complexo e caro e, portanto, merece
atenção da comunidade cientifica e da indústria. Neste contexto, surge a
necessidade do desenvolvimento de técnicas, processos e ferramentas que reduzam
o custo e o esforço envolvidos nas atividades de manutenção e evolução de
software. Neste contexto, as Ferramentas de Gerenciamento de Requisição de
Mudança desempenham um papel fundamental que ultrapassa a simples função de
registrar as falhas e os pedidos de melhoria dos softwares. Este estudo se
propôs a avaliar as funcionalidades das FGRMs de modo a melhorá-las.

Com base no estudo descrito na Seção~\ref{sec:caracterizacao_ferramentas}
verificamos que as FGRM dispõem de funcionalidades para gerenciar a criação,
consulta, atualização e destruição de uma RM\@. Entretanto, em algumas
plataformas, tais como o Github e o Gitlab, não há clara separação entre o
gerenciamento das RMs e o controle de versão de código. Uma possível
desdobramento desta dissertação é avaliar o impacto desta abordagem na
Manutenção de Software. Este tipo de análise poderia modificar o desenvolvimento
de futuras deste tipo de software.

No Capítulo~\ref{ch:mapeamento-sistematico}, ao revisarmos a literatura sobre
melhorias nas FGRMs, recuperamos estudos que discutem diversos aspectos dos
problemas e desafios do gerenciamento das RMs. Apesar do foco em temas como
atribuição automática e RMs duplicadas, outros estudos discutem que alguns
desenvolvedores estariam mais interessados em propostas que melhorem a qualidade
do relato da RM\@. Este é um exemplo do desacoplamento entre as necessidades dos
desenvolvedores e o que está sendo proposto na literatura. Verificamos que a
literatura da área tem se dedicado na melhoria das funções das FGRMs, contudo,
tais avanços ainda não chegaram aos desenvolvedores. Apesar deles se mostrarem
satisfeitos com as funcionalidades oferecidas, ainda existem muito outras que
poderiam se acopladas a este tipo de software de modo a melhorar as atividades
de manter e evoluir um software. Transversalmente as metodologias propostas
pelos agilistas vêm sendo adotadas por algumas equipes de manutenção de
software. Neste contexto, as FGRM podem implantar funcionalidade de modo a
suportar algumas destas práticas.

De maneira relacionada, em uma das classificações feitas no Mapeamento
conduzido, os estudos foram agrupados pelo tipo de papel desempenhado na
Manutenção de Software o qual daria suporte. Utilizamos uma classificação
proposta por Polo e outros~\cite{Polo1999} construída em 1999. Em nossas
pesquisas não identificamos uma discussão mais recente sobre os papéis
desempenhados na Manutenção de Software. Entendemos que seria importante a
realização de um novo trabalho com o objetivo de descreve e avaliar os papéis
desempenhados no processo de manter um software. Este estudo poderia avaliar
como as práticas propostas pelos agilistas podem ter alterado esta estrutura de
trabalho.

O nível de satisfação com as funcionalidades das FGRMs dos profissionais
envolvidos com Manutenção de Software pode ser considerado como alto. Esta
percepção foi obtida mediante um levantamento por questionário apresentado no
Capítulo~\ref{ch:pesquisa-profissionais}. Entretanto, o mesmo estudo demonstrou
que os participantes desconhecem o potencial deste tipo de software. No
questionário, ao apresentarmos propostas de melhorias que eram discutidas na
literatura o grau de aceitação foi bastante elevado. Neste sentido, observamos
um distanciamento entre o estado da arte e o estado da prática das melhorias das
FGRMs. Apesar deste distanciamento ser algo esperado, por diversas razões,
estudos devem ser conduzidos afim de diminuir estas diferenças. Entendemos que
esta dissertação contribuiu neste sentido ao apresentar para os profissionais
algumas das melhorias discutidas na literatura. Este conhecimento pode ser
utilizado pelos desenvolvedores para exigir ferramentas que atendam às suas
demandas. Ao mesmo tempo o nosso trabalho conseguiu coletar, discutir e
apresentar algumas da necessidades dos desenvolvedores. Da mesma forma,
pesquisadores podem usar esta informação para propor trabalhos com o intuito de
propor melhorias para as FGRMs.

Esta dissertação contribuiu para o estudo das FGRMs com a apresentação de um
conjunto de melhorias no Capítulo~\ref{ch:sug_melhoria}. Considerando a sua boa
aceitação, estas recomendações podem ser utilizadas por pesquisadores, pelos
responsáveis por desenvolver e manter projetos de FGRMs e por os profissionais
envolvidos com Manutenção de Software. Um possível desdobramento desta
dissertação é a avaliação do impacto da implantação destas sugestões. Esta
análise poderia ser realizada mediante um Estudo de Caso, por exemplo.

No Capítulo~\ref{ch:implemtacao_extensao} implementamos uma das sugestões como
uma extensão para o módulo de \textit{issues} da plataforma Github. Por ser
tratar de uma Prova de Conceito, como não foi realizada uma avaliação com
membros dos projetos selecionados. Por esta razão não é possível extrapolar o
resultados.  Gostaríamos de futuramente conduzir um trabalho em uma configuração
em que seja possível avaliar a opinião de alguém envolvido no projeto estudado.
A contribuição deste trabalho de dissertação está na proposição de melhorias
para as FGRMs tomando como base a literatura em Manutenção de Software e a
opinião de profissionais envolvidos em Manutenção de Software.

% O gerenciamento das RMs formam as funcionalidades centrais de uma FGRM\@.
% Estas funções podem ser agrupadas em um termo único denominado
% \textit{Operações de
% CRUD} (acrônimo de Create, Read, Update e Delete na língua Inglesa). As
% principais categorias de funcionalidades que foram encontradas para a dimensão
% de \textit{Gestão da RM} estão descritas a seguir.

% Realmente utilizar a justificativa esforço necessário é fazer como você mesmo
% disse aumenta superfície de ataque. Acho que o ideal é colocar que no momento de
% decisão qual ferramenta analisar optamos por um número menor, não pelo esforço,
% mas por entendermos que as escolhidas poderiam caracterizar o conjunto de
% ferramentas disponíveis. Posteriormente, verificamos que a escolha não se
% mostrou ideal já que no survey induziu a alguns erros na proposição das
% melhorias.
