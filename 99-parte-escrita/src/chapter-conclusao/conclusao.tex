\chapter{Conclusão}
\label{ch:conclusao_trab_futuros}

A Manutenção de Software é um processo complexo e caro e, portanto,  merece
atenção da comunidade acadêmica e da indústria. Desta forma, emerge a
necessidade do desenvolvimento de técnicas, processo e ferramentas que reduzam o
custo e o esforço envolvidos nas atividades de manutenção e evolução de
software. Neste contexto, as Ferramentas de Gerenciamento de Requisição de
Mudança desempenham um papel fundamental que ultrapassa a simples função de
registrar falhas em software. Este estudo se propôs a avaliar as funcionalidades
da FGRM de modo a melhorá-las. Verificamos que a literatura da área tem dedicado
nesta melhoria, contudo, tais avanças ainda não chegaram aos desenvolvedores.
Apesar deles se mostrarem satisfeitos coma s funcionalidades oferecidas, ainda
existem muito outras que poderiam se acopladas a este tipo de software de modo a
melhorar as atividades diárias de quem dedicar à manter software.

Transversalmente as metodologias propostas pelos agilistas vêm sendo adotadas
por algumas equipes de manutenção de software. Neste contexto, as FGRM podem
implantar funcionalidade de modo a suportar algumas destas práticas. A
contribuição deste trabalho está na proposição de melhorias para este tipo de
sistema tomando como base a literatura em Engenharia e o estado da prática, com
base na opinião dos profissionais.
