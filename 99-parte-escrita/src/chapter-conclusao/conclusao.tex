\chapter{Conclusão}
\label{ch:conclusao_trab_futuros}

A Manutenção de Software é um processo complexo e caro e, portanto,  merece atenção da
comunidade acadêmica e da indústria. Desta forma, emerge a necessidade do desenvolvimento de técnicas, processo e ferramentas que reduzam o custo e o esforço envolvidos nas atividades de manutenção e evolução de software. Neste contexto, as Ferramentas de Gerenciamento de Requisição de Mudança desempenham um papel fundamental que ultrapassa a simples função de registrar falhas em software. Neste sentido é proposto o estudo para entender o papel desta ferramenta, analisar a literatura sobre o assunto e discutir os aspectos que são considerados mais importantes do ponto de vista dos profissionais. Para alcançarmos este objetivo é proposto um Cronograma de Atividade conforme exibido no Anexo \ref{anexo:cronograma}. As atividades que compõe cada etapa do trabalho estão descritas no Anexo \ref{anexo:atividades}.
