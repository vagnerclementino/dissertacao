\chapter{Conclusão}
\label{ch:conclusao_trab_futuros}

\todobegin{A minha sugestão é articular o ``concluir'' com o ``introduzir'': na
    introdução você procura descrever o trabalho sem ter os elementos
    conceituais que só aparecem no capítulos 2 3 4. Na conclusão você retoma a
    descrição do trabalho e \textbf{traduz} a ``Introdução'' mais formalmente
    podendo se referir a elementos conceituais de todos capítulos.  A distinção
    principal da ``introdução'' e ``conclusão'' é construída pela descrição de
    trabalhos futuros *talvez*  balizados pelas considerações de complexidade e
    diversidade que a literatura provê. A complicação fica na decisão de os
    trabalhos futuros ficarem intercalados ou não no texto finalizador.  Em
    alguns trabalhos os ``trabalhos futuros''  ficam como sendo uma subseção.
    Quando você estava lendo algumas dissertações e teses eu te falei para ficar
    atento a isso para você poder decidir de forma mais tranquila.}

A Manutenção de Software é um processo complexo e caro e, portanto, merece
atenção da comunidade acadêmica e da indústria. Desta forma, emerge a
necessidade do desenvolvimento de técnicas, processo e ferramentas que reduzam o
custo e o esforço envolvidos nas atividades de manutenção e evolução de
software. Neste contexto, as Ferramentas de Gerenciamento de Requisição de
Mudança desempenham um papel fundamental que ultrapassa a simples função de
registrar falhas em software. Este estudo se propôs a avaliar as funcionalidades
das FGRMs de modo a melhorá-las.

Com base no estudo descrito na Seção~\ref{} verificamos que as FGRM dispõe de
funcionalidades para gerenciar a criação, consulta, atualização e destruição de
uma RM\@. Entretanto, em algumas plataformas, tais como o Github e o Gitlab, não
há clara distinção entre o gerenciamento das RMs e o controle de versão de
código. Uma possível desdobramento desta dissertação é avaliar o impacto desta
abordagem na Manutenção de Software.

Ao revisarmos a literatura sobre FGRMs foi possível recuperar estudos discutindo
diversos aspectos dos problemas e desafios do gerenciamento das RMs. Apesar de
constatarmos um foco em problemas como atribuição e RMs duplicadas, estudos
discutem que os desenvolvedores estariam mais interessados em trabalhos que
melhorem a qualidade do relato da RM\@. De maneira relacionada, em uma das
classificações, os estudos foram agrupados pelo tipo de papel desempenhado na
Manutenção de Software o qual visa dar suporte. Utilizamos uma classificação
proposta por Polo e outros~\cite{}, data de 1999. Entendemos que seria
importante uma nova realizar um novo trabalho avaliando os papéis realizados no
processo de manter um software. Este estudo poderia avaliar como as práticas
propostas pelos agilistas podem ter alterado a estrutura dos papéis na
Manutenção de Software.

O nível de satisfação dos profissionais envolvidos com Manutenção de Software
com as funcionalidades das FGRMs é bom. Esta percepção foi obtida mediante um
levantamento por questionário. Entretanto, o mesmo estudo demonstrou que os
participantes desconhecem o potencial deste tipo de software. Ao apresentarmos
algumas das propostas de melhorias discutidas na literatura o nível de aceitação
foi bastante elevado. Neste sentido, observamos um distanciamento entre o estado
da arte e o estado da prática das melhorias das FGRMs. Apesar deste
distanciamento ser natural em determinados contextos, estudos devem ser
realizados afim de diminuir esta distância. Entendemos que este estudo
contribuiu neste sentido ao apresentar para os profissionais algumas das
melhorias discutidas na literatura.

Verificamos que a literatura da área tem dedicado nesta melhoria, contudo, tais
avanços ainda não chegaram aos desenvolvedores. Apesar deles se mostrarem
satisfeitos com as funcionalidades oferecidas, ainda existem muito outras que
poderiam se acopladas a este tipo de software de modo a melhorar as atividades
diárias de quem dedicar à manter software.

Transversalmente as metodologias propostas pelos agilistas vêm sendo adotadas
por algumas equipes de manutenção de software. Neste contexto, as FGRM podem
implantar funcionalidade de modo a suportar algumas destas práticas. A
contribuição deste trabalho está na proposição de melhorias para este tipo de
sistema tomando como base a literatura em Engenharia e o estado da prática, com
base na opinião dos profissionais.

O gerenciamento das RMs formam as funcionalidades centrais de uma FGRM\@. Estas
funções podem ser agrupadas em um termo único denominado \textit{Operações de
    CRUD} (acrônimo de Create, Read, Update e Delete na língua Inglesa). As
principais categorias de funcionalidades que foram encontradas para a dimensão
de \textit{Gestão da RM} estão descritas a seguir.

Realmente utilizar a justificativa esforço necessário é fazer como você mesmo
disse aumenta superfície de ataque. Acho que o ideal é colocar que no momento de
decisão qual ferramenta analisar optamos por um número menor, não pelo esforço,
mas por entendermos que as escolhidas poderiam caracterizar o conjunto de
ferramentas disponíveis. Posteriormente, verificamos que a escolha não se
mostrou ideal já que no survey induziu a alguns erros na proposição das
melhorias.
