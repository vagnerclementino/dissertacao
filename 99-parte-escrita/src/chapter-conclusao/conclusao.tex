%%%%%%%%%%%%%%%%%%%%%%%%%%%%%%%%%%%%%%%%%%%%%%%%%%%%%%%%%%%%%%%%%%%%%%%%%%%%%%%%
%Objetivo: Descreve a conclusão da dissertação
%Autores: Vagner Clementino <vagnercs@dcc.ufmg.br>
%		  Rodolfo Resende <rodolfo@dcc.ufmg.br>
%Criação: dom fev 26 12:49:27 BRT 2017
%Modificação: dom mai 14 18:17:37 -03 2017
%Revisão: dom set 24 14:30:34 -03 2017
%%%%%%%%%%%%%%%%%%%%%%%%%%%%%%%%%%%%%%%%%%%%%%%%%%%%%%%%%%%%%%%%%%%%%%%%%%%%%%%%
\chapter{Conclusão}\label{ch:conclusao_trab_futuros}

A contribuição deste trabalho de dissertação está na proposição de melhorias
para as FGRMs tomando como base a literatura da área e a opinião de
profissionais envolvidos com Manutenção de Software. As atividades empregadas
para manter e evoluir software são complexas e caras e, portanto, merecem
atenção da comunidade científica e da indústria. Surge então a necessidade do
desenvolvimento de técnicas, processos e ferramentas que reduzam o custo e o
esforço envolvidos nas atividades de manutenção e evolução de software.
Conforme discutido, as FGRMs desempenham um papel fundamental que ultrapassa
apenas registrar as falhas e os pedidos de melhorias dos softwares.

Com base no estudo descrito na Seção~\ref{sec:caracterizacao_ferramentas}
verificamos que as FGRMs dispõem de funcionalidades para gerenciar a criação,
consulta, atualização e destruição de uma RM\@. Entretanto, em algumas
plataformas, tais como o Github e o Gitlab, não existe uma clara separação
entre o gerenciamento das RMs e o controle de versão do código. Um possível
desdobramento desta dissertação é avaliar o impacto deste tipo de abordagem no
processo de manutenção de software. Este tipo de análise poderia modificar o
desenvolvimento de futuras versões das FGRMs, especialmente para aquelas que
estão acopladas a repositórios de software.

No Capítulo~\ref{ch:mapeamento-sistematico}, ao revisarmos a literatura sobre
melhorias nas FGRMs, apresentamos estudos que discutem diversos aspectos dos
problemas e desafios do gerenciamento das RMs. A maior frequência de trabalhos
está relacionada com temas como atribuição automática e RMs duplicadas.
Conforme discutimos, alguns autores afirmam que, do ponto de vista dos
desenvolvedores, a duplicação dos pedidos de manutenção não seria o principal
problema a ser tratado. Em contrapartida, os profissionais estariam mais
interessados em propostas que melhorem a qualidade do relato. Esta situação
pode ser o sinal de um possível desacoplamento entre as necessidades dos
desenvolvedores e o que está sendo proposto na literatura. Apesar da qualidade
dos estudos sobre melhorias das funções das FGRMs, tais avanços, aparentemente,
não são de conhecimento das equipes de manutenção. Com base nos resultados da
amostra do levantamento realizado no Capítulo~\ref{ch:pesquisa-profissionais}
percebemos que os profissionais consultados estão satisfeitos com as
funcionalidades oferecidas. Todavia, a nossa visão é que existem muitos outros
comportamentos que poderiam ser acoplados a este tipo de software de modo a
melhorar as atividades de manter e evoluir um software. Da mesma forma, as
práticas propostas pelos agilistas vêm sendo adotadas por algumas equipes de
manutenção de software. Um trabalho futuro desta dissertação é aprofundar no
estudo da relação entre as práticas dos agilistas e a manutenção de software.
Em especial, entender como as FGRMs poderiam implantar funcionalidades com o
objetivo de suportar alguns dos métodos.

De maneira relacionada, em uma das classificações feitas no Mapeamento
conduzido, os estudos foram agrupados pelo tipo de papel desempenhado na
Manutenção de Software. Utilizamos uma classificação baseada na que foi
proposta por Polo e outros~\cite{Polo1999} apresentada em 1999. Em nossas
pesquisas na literatura não foi possível identificar uma discussão mais recente
sobre os papéis desempenhados na Manutenção de Software. Entendemos que seria
importante a condução de um novo trabalho com o objetivo de descrever e avaliar
os papéis presentes no processo de manter um software. Este estudo poderia
avaliar, por exemplo, como as práticas propostas pelos agilistas podem ter
alterado a estrutura de trabalho das equipes de manutenção.

O nível de satisfação dos profissionais com as funcionalidades das FGRMs pode
ser considerado alto. Esta percepção foi obtida mediante um levantamento por
questionário apresentado no Capítulo~\ref{ch:pesquisa-profissionais}.
Entretanto, o mesmo estudo demonstrou que os participantes desconhecem o
potencial deste tipo de software. Esta situação pode ser observada pelo fato
que, ao apresentarmos propostas de melhorias que eram discutidas na literatura,
o grau de aceitação foi elevado. Isto poderia ser considerado um distanciamento
entre o estado da arte e o estado da prática das melhorias para as FGRMs.
Entendemos que esta dissertação contribuiu neste sentido ao apresentar para os
profissionais algumas das melhorias discutidas na literatura. Este conhecimento
pode ser utilizado pelos desenvolvedores para exigir ferramentas que atendam às
suas demandas. Ao mesmo tempo, o nosso trabalho conseguiu coletar, discutir e
apresentar algumas das necessidades dos desenvolvedores. Da mesma forma,
pesquisadores podem usar esta informação para propor estudos propondo melhorias
para as FGRMs.

Esta dissertação contribuiu para o estudo das FGRMs com a apresentação de um
conjunto de melhorias no Capítulo~\ref{ch:sug_melhoria}. Considerando que a
maioria delas teve uma boa aceitação, as recomendações podem ser utilizadas por
pesquisadores, pelos responsáveis por desenvolver e manter projetos de FGRMs e
por os profissionais envolvidos com Manutenção de Software. Um possível
trabalho futuro é a avaliação do impacto da implantação destas sugestões. Esta
análise poderia ser realizada mediante um Estudo de Caso. Além disso, seria
importante conduzir uma análise qualitativa da relevância das sugestões
propostas. Um estudo com base em entrevistas poderia melhorar o entendimento da
opinião dos profissionais e esclarecer como tratar situações onde os critérios
definidos nas sugestões de melhorias não possam ser atendidos. Neste eventual
estudo seria importante realizar uma avaliação mais aprofundada sobre questões
que surgiram durante a avaliação das sugestões de melhorias e que não puderam
ser esclarecidas, como por exemplo, o ranqueamento pela reputação do
Reportador.

Na elaboração do modelo conceitual do contexto das FGRMs, que pudesse dar
suporte as discussões realizadas nesta dissertação, verificamos que o
tratamento da ambiguidade e inconsistência, entre os diversos artigos que
tratam do assunto, exigia maior esforço do que foi possível investir. Neste
sentido, entendemos que seria importante a realização de um estudo com o
objetivo de melhorar a organização dos conceitos da área de Manutenção de
Software, em especial sobre as RMs e FGRMs.

No Capítulo~\ref{ch:implemtacao_extensao} implementamos uma das sugestões como
uma extensão para o módulo de gerenciamento de RMs \textit{issues} da
plataforma Github. Por ser tratar de uma Prova de Conceito, não foi realizada
uma avaliação com membros dos projetos selecionados. Por esta razão não é
possível generalizar os resultados. Um trabalho futuro é conduzir um estudo em
uma configuração em que seja possível avaliar a opinião de alguém envolvido no
projeto estudado. Entendemos que seria importante conduzir um estudo com os
mesmos objetivos desta dissertação, entretanto, coletando a opinião daqueles
que reportam as RMs.

Conforme apresentado na Seção~\ref{sec:caracterizacao_ferramentas} as FGRMs
disponibilizam diversos métodos para o registro de uma RM\@: envio de e-mail,
formulários que podem ser disponibilizados em sítios da Web e etc. Entretanto,
a maneira mais comum é o formulário disponibilizado pelas FGRMs, conforme
exibido na Figura~\ref{fig:rm-exemplo}. Segundo o nosso entendimento, o
processo de criação de RMs poderia ser melhorado com a utilização de uma
interface que estimule e possibilite o \textit{Reportador} fornecer as
informações necessárias à análise da RM\@. Uma possível abordagem é a
utilização pelas FGRMs de uma interface de criação de RMs através de um
\textit{chatbot}~\cite{mauldin1994chatterbots,huang2007extracting}. Esta
maneira interativa de criação de uma RM, dentre outros aspectos, pode ajudar no
fornecimento das informações necessárias para a correspondente solução.

Os dados obtidos e os artefatos utilizados nesta dissertação (questionários dos
levantamentos, cartões de classificação, planilha e etc) estão disponíveis no
endereço eletrônico
\url{https://archive.org/details/dissertacao-vagner-clementino-um-estudo}.
